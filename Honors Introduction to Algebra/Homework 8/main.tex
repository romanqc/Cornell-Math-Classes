\documentclass[12pt]{article}
\usepackage[margin=1in]{geometry}

\usepackage{libertine}
\usepackage{parskip}
\usepackage{enumitem}
\usepackage{array}
\usepackage{graphicx}
\graphicspath{ {./images/} }

\usepackage{amsthm,amsmath,amssymb}
\usepackage{tikz}
\usetikzlibrary{arrows,automata,shapes.geometric}

\usepackage{pgfplots}
\pgfplotsset{compat=1.18}

\pagestyle{plain}
\thispagestyle{empty}

\definecolor{carnellian}{RGB}{190,20,20}

\theoremstyle{definition}
\newtheorem{problem}{Problem}

\newcounter{subq}[problem]
\newenvironment{subproblem}
{\refstepcounter{subq} \begin{itemize} \item[(\alph{subq})]}
{\end{itemize} \medskip}
\DeclareMathOperator{\Ima}{Im}
\DeclareMathOperator{\rank}{rank}
\DeclareMathOperator{\tr}{tr}
\DeclareMathOperator{\Hom}{Hom}
\DeclareMathOperator{\End}{End}
\DeclareMathOperator{\Span}{Span}
\DeclareMathOperator{\Aut}{Aut}
\DeclareMathOperator{\Ann}{Ann}
\DeclareMathOperator{\Mat}{Mat}
%\DeclareMathOperator{\deg}{deg}



\usepackage{environ}
\NewEnviron{solution}[1][\vfill]{
    \textcolor{blue}{\BODY}
}

\newcommand{\hwnum}{8}
\newcommand{\duedate}{4/8/2025}
\renewcommand{\title}{Groups, Modules, and Vector Spaces}

\begin{document}

\hspace{-10px}
\begin{tabular*}{\textwidth}{l @{\extracolsep{\fill}} r}
    \textbf{Honors Algebra} & \textbf{Spring 2025} \\
    \textbf{HW \hwnum : \title} &  \textbf{\duedate} \\
\end{tabular*}

\vspace{1cm}

\textit{Abstract Algebra: An Integrated Approach by J.H. Silverman.}\\
Page 320-325: 10.13, 10.16, 10.22, 10.23, 10.27\\
Page 357–370: 11.24, 11.32, 11.37\\
Page 393–396: 12.16\\
Do problem 11.32 for modules over non-commutative rings, i.e., show that 
if M is left R module then the annihiliator Ann(M) is two sided ideal in R.


\vspace{1cm}

%-----------TEMPLATE------------%
% \begin{problem}
%     \begin{enumerate}[label=(\alph*)]
%         \item 
%         \begin{solution}

%         \end{solution}

%         \item 
%         \begin{solution}

%         \end{solution}
%     \end{enumerate}
% \end{problem}

\begin{problem}[10.13]
    For each of the following linear operators, find eigenvectors and eigenvalues as we did in Example 10.27. 
    \begin{enumerate}[label=(\alph*)]
        \item $L(e_1) = -e_1 + 4e_2$ and $L(e_2) = 4e_1 - e_2$
        \begin{solution}

        \end{solution}

        \item $L(e_1) = e_2$ and $L(e_2) = -e_1$. (You may take $F = \mathbb{C}$ for this part.)
        \begin{solution}

        \end{solution}

        \item $L(e_1) = e_1 - e_3$, $L(e_2) = 4e_3$, and $L(e_3) = 2e_1 + e_2 + 2e_3$
        \begin{solution}

        \end{solution}
    \end{enumerate}
\end{problem}

\begin{problem}[10.16]
    Let $V$ be a finite-dimensional vector space. Let $L1, L2 \in \End_F{(V)}$ be linear operators on $V$ such that the following three statements are true: 
    \begin{enumerate}[label=(\arabic*)]
        \item There is a basis for $V$ consisting of eigenvectors of $L_1$
        \begin{solution}

        \end{solution}

        \item There is a basis for $V$ consisting of eigenvectors of $L_2$
        \begin{solution}

        \end{solution}

        \item $L_1L_2 = L_2L_1$
        \begin{solution}

        \end{solution}
    \end{enumerate}
    Prove that there is a basis for $V$ consisting of vectors that are simultaneously eigenvectors of $L1$ and eigenvectors of $L2$. 
    This result is often stated as follows: “commuting diagonalizable matrices are simultaneously diagonalizable.” 
    (Hint. If you’re not sure how to get started, first try the case that $L1$ has $\dim(V)$ distinct eigenvalues.)
\end{problem}

\begin{problem}[10.22]
    Let $L \in \End_F{(V)}$ be an invertible linear operator.
    \begin{enumerate}[label=(\alph*)]
        \item Prove that
        \[
            P_{L^{-1}}(T) = \det(L)^{-1} \cdot (-T)^{\dim{V}} \cdot P_L(T^{-1})
        \]
        \begin{solution}

        \end{solution}

        \item Let $n = \dim{V}$, and let the eigenvalues of $L$ be $\lambda_1, \ldots \lambda_n$ (repeated with appropriate multiplicity).
        Prove that the eigenvalues of $L^{-1}$ are $\lambda_1^{-1}, \ldots , \lambda_n^{-1}$.
        \begin{solution}

        \end{solution}

        \item If $L^d = I$, prove that the eigenvalues of $L$ are $d$th-roots of unity.
        \begin{solution}

        \end{solution}
    \end{enumerate}
\end{problem}

\begin{problem}[10.23]
    Let $V$ be an $n$-dimensional $F$ vector space, let $L \in \End_F(V)$, and let
    \[
        P_L(T) = T^n - c_1(L)T^{n-1} + c_2(L)T^{n-2} - \cdots + (-1)^{c}c_n(L)
    \]
    The \textit{trace of} $L$ is defined to be the quantity
    \[
        \tr(L) = c_1(L)
    \]
    \begin{enumerate}[label=(\alph*)]
        \item Let $J, L \in \End_F(V)$. Prove that
        \[
            \tr(JL) = \tr(LJ)
        \]
        In particular, if $J$ is invertible, prove that
        \[
            \tr(J^{-1}LJ) = \tr(L)
        \]
        \begin{solution}

        \end{solution}

        \item Prove that
        \[
            \tr(aL_1 + bL_2) = a\tr(L_1) + b\tr(L_2);
        \]
        i.e., prove that the map
        \[
            \tr : \End_F(V) \longrightarrow F
        \]
        is an $F$-linear transformation.
        \begin{solution}

        \end{solution}

        \item Suppose that $F$ is algebraically closed and that $\lambda_1, \ldots, \lambda_n$ are the eigenvalues of $L$, 
        repeated with appropriate multiplicities so that $P_L(T) = \prod(T - \lambda_i)$. Prove that
        \[
            \tr(L) = \lambda_1 + \cdots + \lambda_n
        \]
        \begin{solution}

        \end{solution}

        \item Let $\mathcal{B}$ be a basis for $V$, and let
        \begin{equation}
            \mathcal{M}_{\mathcal{L}, \mathcal{B}, \mathcal{B}} = 
            \begin{bmatrix}
                a_{11} & \cdots & a_{1n}\\ 
                \cdots & \ddots & \vdots\\   
                a_{n1}& \cdots & a_{nn}
            \end{bmatrix}
            \in \Mat_{n \times n}(F)
        \end{equation}
        be the matrix associated to $\mathcal{L}$ for the basis $\mathcal{B}$. Prove that
        \[
            \tr(L) = a_{11} + a_{22} + \cdots + a_{nn}
        \]
        is the sum of the diagonal elements of the martix $L$.
        \begin{solution}

        \end{solution}
    \end{enumerate}
\end{problem}

\begin{problem}[10.27]
    We defined the product of an infinite list of vector spaces $V1, V2, V3, \ldots$, but suppose that we want to 
    take the product of an uncountable number of vector spaces. In general, we take an arbitrary index set $I$, 
    and we suppose that for each $i in I$ we are given an $F$-vector space $V_i$. We can no longer talk about ordered 
    lists of infinite-tuples, since the index set $I$ is arbitrary, so we define the direct product of the $V_i$ over 
    all $i \in I$ to be a vector space of functions,

    \[
        \prod_{i \in I}V_i = \biggl\{ \text{functions }v : I \longrightarrow \bigcup_{i \in I} V_i \hspace{10px}\text{satisfying } v(i) \in V_i \hspace{10px}\text{for all } i \in I \biggr\}
    \]
    Addition and scalar multiplication in $\prod_{i \in I}V_i$ are defined by
    \[
        (v + w)(i) = v(i) + w(i) \hspace{10px} \text{and} \hspace{10px} (cv)(i) = cv(i)
    \]
    \begin{enumerate}[label=(\alph*)]
        \item Prove that $\prod_{i \in I} v_i$ is an $F$-vector space.
        \begin{solution}

        \end{solution}

        \item If $I = \mathbb{N}$, explain why the definition of $\prod_{i \in \mathbb{N}} V_i$ in this exercise is the same as the one given
        in Definition 10.48
        \begin{solution}

        \end{solution}

        \item Explain how you would define the direct sum of the $V_i$ for an arbitrary index set $I$.
        \begin{solution}

        \end{solution}
    \end{enumerate}
\end{problem}

\begin{problem}[11.24]
    An \textit{Artinian ring} is a ring in which every descending list of ideals
    \[
        I_1 \supseteq I_2 \supseteq I_3 \supseteq \cdots
    \]
    eventually stabilizes; i.e., there is a $k \geq 1$ so that $I_k = I_{k + i}$ for all $i \geq 0$. Although this resembles
    the definition of Noetherian ring, the Artinian condition is actually fare more restrictive.

    \begin{enumerate}[label=(\alph*)]
        \item Let $R$ be an Artinian ring, and let $I$ be an ideal in $R$. Prove that $R/I$ is an Artinian ring.
        \begin{solution}

        \end{solution}

        \item Let $R$ be an Artinian ring that is an integral domain. Prove that $R$ is a field. (\textit{Hint.} Let $a \in R$
        and consider the ideals $aR \supseteq a^2R \supseteq a^3R \supseteq \cdots$)
        \begin{solution}

        \end{solution}

        \item Let $R$ be an Artinian ring. Prove that every prime ideal in $R$ is a maximal ideal. (\textit{Hint.} Use (a)
        and (b)).
        \begin{solution}

        \end{solution}

        \item Let $R$ be an Artinian ring. Prove that $R$ has only finitely many maximal ideals.
        \begin{solution}

        \end{solution}
    \end{enumerate}
\end{problem}

\begin{problem}[11.32] Do problem 11.32 for modules over non-commutative rings, i.e., show that 
    if M is left R module then the annihiliator Ann(M) is two sided ideal in R. Let $M$ be an 
    $R$-module. Prove that the annihilator
    \[
        \Ann(M) = \{ a \in R : am = 0 \hspace{10px} \text{for all} \hspace{10px} m \in M \}
    \]
    is an ideal of $R$.
    \begin{solution}

    \end{solution}
\end{problem}

\begin{problem}[11.37]
    Let $R$ be a commutative ring.
    \begin{enumerate}[label=(\alph*)]
        \item Suppose that $a, b \in R$ have the property that $aR + bR = R$. Prove that for all $m, n \geq 1$
        we have 
        \[
            a^mR + b^nR = R
        \]
        \begin{solution}

        \end{solution}

        \item More generally, let $a_1, \ldots, a_t \in R$, and let $e_1, \ldots, e_t \geq 1$ be positive integers.
        Prove that 
        \[
            a_1R + a_2R + \cdots + a_tR = R \hspace{10px} \iff \hspace{10px} a_1^{e_1}R + a_2^{e_2}R + \cdots + a_t^{e_t}R = R.
        \]
        \begin{solution}

        \end{solution}
    \end{enumerate}
\end{problem}

\begin{problem}[12.16]
    This exercise give examples showing that the list of composition quotients of a finite group $G$
    are not enough to determine $G$.
    \begin{enumerate}[label=(\alph*)]
        \item Prove that composition series for the cyclic group $\mathcal{C}_4$ and the product group $\mathcal{C}_2 \times \mathcal{C}_2$
        have the same length and the same composition quotients.
        \begin{solution}

        \end{solution}

        \item Prove that composition series for the cyclic group $\mathcal{C}_6$ and the symmetric group $\mathcal{S}_3$ have the
        same length and the same composition quotients.
        \begin{solution}

        \end{solution}
    \end{enumerate}
\end{problem}

\end{document}