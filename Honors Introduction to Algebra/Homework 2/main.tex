\documentclass[12pt]{article}
\usepackage[margin=1in]{geometry}

\usepackage{libertine}
\usepackage{parskip}
\usepackage{enumitem}
\usepackage{array}
\usepackage{graphicx}
\graphicspath{ {./images/} }

\usepackage{amsthm,amsmath,amssymb}
\usepackage{tikz}
\usetikzlibrary{arrows,automata,shapes.geometric}

\usepackage{pgfplots}
\pgfplotsset{compat=1.18}

\pagestyle{plain}
\thispagestyle{empty}

\definecolor{carnellian}{RGB}{190,20,20}

\theoremstyle{definition}
\newtheorem{problem}{Problem}

\newcounter{subq}[problem]
\newenvironment{subproblem}
{\refstepcounter{subq} \begin{itemize} \item[(\alph{subq})]}
{\end{itemize} \medskip}
\DeclareMathOperator{\Ima}{Im}
\DeclareMathOperator{\rank}{rank}
\DeclareMathOperator{\tr}{tr}
% \DeclareMathOperator{\span}{span}



\usepackage{environ}
\NewEnviron{solution}[1][\vfill]{
    \textcolor{blue}{\BODY}
}

\newcommand{\hwnum}{2}
\newcommand{\duedate}{2/4/2025}
\renewcommand{\title}{Groups}

\begin{document}

\hspace{-10px}
\begin{tabular*}{\textwidth}{l @{\extracolsep{\fill}} r}
    \textbf{Honors Algebra} & \textbf{Spring 2025} \\
    \textbf{HW \hwnum : \title} &  \textbf{\duedate} \\
\end{tabular*}

\vspace{1cm}

\textit{Abstract Algebra: An Integrated Approach by J.H. Silverman.}\\
Page 53-62: 2.2, 2.6, 2.8, 2.15, 2.19, 2.21, 2.25, 2.28, 2.30, 2.33, 2.36, 2.39, 2.44, 2.45

\vspace{1cm}

\begin{problem}[2.2]
    Let $n$ be a positive integer, and let $S_n$ be the group of permutations of the set $\{1, 2,\ldots n\}$ as described
    in Example 2.19. Prove that $S_n$ is a finite group, and give a formula for the order of $S_n$.

    \begin{solution}
        The symmetric group $S_n$ consists of all possible bijections (permutations) of the set $\{1, 2, \dots, n\}$. Since there are only finitely many elements in this set, the number of possible permutations is also finite. Thus, $S_n$ is a finite group.
        
        The order of $S_n$ is determined by the number of permutations, the rearrangements, of the $n$ elements. Thus,
        \[
            |S_n| = n!
        \]
        
        is the order of $S_n$.
    \end{solution}
\end{problem}

\begin{problem}[2.6]
    Let $G$ be a group, let $g$ and $h$ be elements of $G$, and suppose that $g$ has order $n$ and that $h$ has order $m$.
    \begin{enumerate}[label=(\alph*)]
        \item If $G$ is an abelian group and if $\gcd(m, n) = 1$, prove that the order of $gh$ is $mn$.
        
        \begin{solution}
            Since $G$ is abelian, we have $(gh)^k = g^k h^k$ for any integer $k$. To find the order of $gh$, we need to find the smallest positive integer $k$ such that $(gh)^k = e$.
            
            Given that $g$ has order $n$, we have $g^n = e$, and since $h$ has order $m$, we have $h^m = e$. If we raise $gh$ to the power $mn$, we get:
            \[
                (gh)^{mn} = g^{mn} h^{mn} = (g^n)^m (h^m)^n = e^m e^n = e.
            \]
            
            To show that $mn$ is the smallest such exponent, assume there exists some $k < mn$ such that $(gh)^k = e$. Then:
            \[
                g^k h^k = e.
            \]
            
            Since the orders of $g$ and $h$ are $n$ and $m$, respectively, and $\gcd(m, n) = 1$, it follows from number theory that the least common multiple of $n$ and $m$ is $mn$. Thus, the order of $gh$ must be $mn$.
        \end{solution}
        \item Give an example showing that (a) need not be true if we allow $\gcd(m, n) > 1$.
        
        \begin{solution}
            Consider the abelian group $\mathbb{Z}/6\mathbb{Z}$ and let $g = 2 + 6\mathbb{Z}$ and $h = 3 + 6\mathbb{Z}$. Here, $g$ has order 3 since $2^3 = 6 \equiv 0 \pmod{6}$, and $h$ has order 2 since $3^2 = 6 \equiv 0 \pmod{6}$. However, $gh = (2 + 6\mathbb{Z}) + (3 + 6\mathbb{Z}) = 5 + 6\mathbb{Z}$, and $5$ has order 6 in $\mathbb{Z}/6\mathbb{Z}$, not $3 \cdot 2 = 6$. The actual order of $gh$ is 2, which is less than 6.
        \end{solution}
        \item Give an example of a nonabelian group showing that (a) need not be true even if we retian the requirement that $\gcd(m, n) = 1$.
        
        \begin{solution}
            Consider the symmetric group $S_3$, which is nonabelian. Let $g = (1\ 2)$ and $h = (1\ 2\ 3)$. The order of $g$ is 2, and the order of $h$ is 3. Since $\gcd(2,3) = 1$, we check the order of $gh$:
            \[
                gh = (1\ 2)(1\ 2\ 3) = (1\ 3).
            \]
            The order of $(1\ 3)$ is 2, not $2 \cdot 3 = 6$. Hence, (a) does not necessarily hold in a nonabelian group.
        
        \end{solution}
        \item Again assume that $G$ is an abelian group, and let $l = mn/\gcd(m, n)$. Porve that $G$ has an element of order $l$. (\textit{Hint:} The element
              $gh$ might not have order $l$, so in general you'll need to take a power of $g$ times a power of $h$.) 

        \begin{solution}
            Let $d = \gcd(m, n)$. Then we can write $m = d m'$ and $n = d n'$, where $\gcd(m', n') = 1$. Define the element:
            \[
                x = g^{m'} h^{n'}.
            \]
            We will show that $x$ has order $l = \frac{mn}{d} = m' n' d$.
            
            First, compute $x^l$:
            \[
                x^l = (g^{m'} h^{n'})^l = g^{m' l} h^{n' l}.
            \]
            Since $l = m' n' d$, we substitute:
            \[
                g^{m' l} = g^{m' m' n' d} = (g^n)^{m' n'} = e^{m' n'} = e,
            \]
            and similarly,
            \[
                h^{n' l} = h^{n' m' n' d} = (h^m)^{m' n'} = e^{m' n'} = e.
            \]
            Thus, $x^l = e$, meaning that the order of $x$ divides $l$.
            
            To show that $x$ has order exactly $l$, assume $x^k = e$ for some $k < l$. This would imply that $g^{m' k} = e$ and $h^{n' k} = e$. But the least common multiple of $m'$ and $n'$ is $m' n'$, and so $k$ must be at least $l$. Thus, the order of $x$ is $l$, completing the proof.
        \end{solution}
    \end{enumerate}
\end{problem}

\begin{problem}[2.8]
    There are other sorts of algebraic structures that are similar to groups in that they are sets $S$ that have a composition law
    \[S \times S \longrightarrow S, \hspace{10px} (s_1, s_2) \longmapsto s_1 \cdot s_2,\]
    but they have fewer or different axioms than a group. In this exercise we explore two of these structures.
    \begin{itemize}
        \item The set $S$ with its composition law is a \textit{monoid} if it has an identity element $e \in S$ and satisfies the associative law,
              but elements are not required to have inverses. (See Exercise 2.19 for another example of a monoid.)
        \item The set $S$ with its composition law is a \textit{semigroup} if its composition law is associative, but it need not have an identity
              element or inverses. (Indeed, without an identity element, the definition of inverse doesn't even make sense.)
    \end{itemize}
    For each of the following sets $S$ and composition laws $\cdot$, determine if $(S, \cdot)$ is a group, a monoid, or a semigroup.
    \begin{enumerate}[label=(\alph*)]
        \item The set of natural numbers $\mathbb{N} = \{1, 2, 3, \ldots\}$ with the composition law being addition.
        
        \begin{solution}
        The operation of addition is associative, meaning $(a + b) + c = a + (b + c)$ for all $a, b, c \in \mathbb{N}$. However, there is no identity element in $\mathbb{N}$ since $0 \notin \mathbb{N}$. Also inverses don't exist because there is no $b \in \mathbb{N}$ such that $a + b = e$. Therefore, $(\mathbb{N}, +)$ is a semigroup but not a monoid or a group.
        \end{solution}
        \item The set of extended natural numbers $\mathbb{N}_0 = \{0, 1, 2, 3, \ldots\}$ with the composition law being addition.
        
        \begin{solution}
            The operation of addition is associative, and now $0$ acts as an identity element since $a + 0 = a$ for all $a \in \mathbb{N}_0$. In this case, additive inverses don't exist because for example, there is no $b \in \mathbb{N}_0$ such that $1 + b = 0$. Thus, $(\mathbb{N}_0, +)$ is a monoid but not a group.
        \end{solution}
        \item The set of integers $\mathbb{Z} = \{\ldots, -3, -2, -1, 0, 1, 2, 3, \ldots\}$ with the composition law being addition.
        
        \begin{solution}
            The operation of addition is associative, and the identity element is $0$. Furthermore, every element $a \in \mathbb{Z}$ has an additive inverse $-a$ such that $a + (-a) = 0$. Hence, $(\mathbb{Z}, +)$ is a group.
        \end{solution}
        \item The set of natural numbers $\mathbb{N}$ with the composition law being multiplication.
        
        \begin{solution}
            Multiplication is associative, and the identity element is $1$ since $a \cdot 1 = a$ for all $a \in \mathbb{N}$. However, there are no multiplicative inverses in $\mathbb{N}$ since $\frac{1}{a}$ is not necessarily in $\mathbb{N}$. Thus, $(\mathbb{N}, \cdot)$ is a monoid but not a group.
        \end{solution}
        \item The set of extended natural numbers $\mathbb{N}_0$ with the composition law being multiplication.
        
        \begin{solution}
            Multiplication is associative, and the identity element is still $1$. However, $0$ is also included, and while it does not affect the identity, it ensures the presence of absorbing elements. Since inverses don't exist for general elements, $(\mathbb{N}_0, \cdot)$ is a monoid but not a group.
        \end{solution}
        \item The set of integers $\mathbb{Z}$ with the composition law being multiplication.
        
        \begin{solution}
            Multiplication is associative, and the identity element is $1$. However, not all integers have multiplicative inverses in $\mathbb{Z}$ (only $1$ and $-1$ do). Since inverses are required for a group, $(\mathbb{Z}, \cdot)$ is a monoid but not a group.
        \end{solution}
        \item The set of integers $\mathbb{Z}$ with the composition law $m \cdot n = \max\{m, n\}$.
        
        \begin{solution}
            The operation $\max(m, n)$ is associative, meaning $\max(\max(a, b), c) = \max(a, \max(b, c))$. However, there is no identity element because no single integer $e$ satisfies $\max(m, e) = m$ for all $m \in \mathbb{Z}$. Thus, $(\mathbb{Z}, \max)$ is a semigroup but not a monoid or a group.
        \end{solution}
        \item The set of natural numbers $\mathbb{N}$ with composition law $m \cdot n = \max\{m, n\}$
        
        \begin{solution}
            Similar to the previous case, $\max(m, n)$ is associative. Here, the identity element is $1$ since $\max(m, 1) = m$ for all $m \in \mathbb{N}$. However, inverses don't exist, so $(\mathbb{N}, \max)$ is a monoid but not a group.
        \end{solution}
        \item The set of natural numbers $\mathbb{N}$ with composition law $m \cdot n = \min\{m, n\}$
        
        \begin{solution}
            The operation $\min(m, n)$ is associative, but there is no identity element that satisfies \\ $\min(m, e) = m$ for all $m \in \mathbb{N}$. Therefore, $(\mathbb{N}, \min)$ is a semigroup but not a monoid or a group.
        \end{solution}
        \item The set natural numbers $\mathbb{N}$ with composition law $m \cdot n = mn^2$.
        
        \begin{solution}
            This operation is not associative in general since $(a \cdot b) \cdot c = (ab^2) \cdot c = (ab^2)c^2 = abc^4$ is not necessarily equal to $a \cdot (b \cdot c) = a \cdot (bc^2) = a(bc^2)^2 = ab^2c^4$. Since associativity fails, $(\mathbb{N}, \cdot)$ is not even a semigroup.
        \end{solution}
    \end{enumerate}
\end{problem}

\begin{problem}[2.15]
    \begin{enumerate}[label=(\alph*)]
        \item Let
            \begin{equation*}
                \text{GL}_2(\mathbb{R}) = 
                \biggl\{ \begin{pmatrix} a & b \\ c & d \end{pmatrix} : a, b, c, d \in \mathbb{R}, ad - bc \neq 0 \biggr\}
            \end{equation*}
        be the indicated set of 2-by-2 matrices, with composition law being matrix multiplication, as
        described in Example 2.21. Prove that $\text{GL}_2(\mathbb{R})$ is a group.
        \begin{solution}
            \begin{enumerate}[label=(\roman*)]
                \item Closure 
                
                If $A, B \in \text{GL}_2(\mathbb{R})$, then their determinant is nonzero, meaning their product $AB$ also has a nonzero determinant:
                \[
                \det(AB) = \det(A) \det(B) \neq 0.
                \]
                Thus, $AB \in \text{GL}_2(\mathbb{R})$.
        
                \item Associativity
                
                Matrix multiplication is always associative, so for all $A, B, C \in \text{GL}_2(\mathbb{R})$,
                \[
                (AB)C = A(BC).
                \]
        
                \item Identity Element
                
                The identity matrix $I = \begin{pmatrix}1 & 0 \\ 0 & 1\end{pmatrix}$ is in $\text{GL}_2(\mathbb{R})$ and satisfies $AI = IA = A$ for all $A \in \text{GL}_2(\mathbb{R})$.
        
                \item Inverses
                
                For any $A = \begin{pmatrix} a & b \\ c & d \end{pmatrix} \in \text{GL}_2(\mathbb{R})$, the determinant is nonzero, so its inverse exists and is given by:
                \[
                A^{-1} = \frac{1}{ad - bc} \begin{pmatrix} d & -b \\ -c & a \end{pmatrix}.
                \]
                Since $\det(A) \neq 0$, $A^{-1} \in \text{GL}_2(\mathbb{R})$.
                \end{enumerate}
            Since all group axioms hold, $\text{GL}_2(\mathbb{R})$ is a group under matrix multiplication.
        \end{solution}
        \item Let $\text{SL}_2(\mathbb{R})$ be the set of 2-by-2 matrices
            \begin{equation*}
                \text{SL}_2(\mathbb{R}) = 
                \biggl\{ \begin{pmatrix} a & b \\ c & d \end{pmatrix} : a, b, c, d \in \mathbb{R}, ad - bc = 1 \biggr\}
            \end{equation*}
        Prove that $\text{SL}_2(\mathbb{R})$ is a group, where the group law is again matrix multiplication.
        \begin{solution}
            \begin{enumerate}[label=(\roman*)]
                \item Closure
                
                If $A, B \in \text{SL}_2(\mathbb{R})$, then their determinants are both $1$. The determinant of their product is:
                \[
                \det(AB) = \det(A) \det(B) = 1 \cdot 1 = 1.
                \]
                Thus, $AB \in \text{SL}_2(\mathbb{R})$.

                \item Associativity
                
                Since matrix multiplication is associative, for all $A, B, C \in \text{SL}_2(\mathbb{R})$,
                \[
                (AB)C = A(BC).
                \]

                \item Identity Element
                
                The identity matrix 
                \[
                I = \begin{pmatrix}1 & 0 \\ 0 & 1\end{pmatrix}
                \]
                has determinant $\det(I) = 1$ and belongs to $\text{SL}_2(\mathbb{R})$. It satisfies $AI = IA = A$ for all $A \in \text{SL}_2(\mathbb{R})$.

                \item Inverses
                
                If $A = \begin{pmatrix} a & b \\ c & d \end{pmatrix} \in \text{SL}_2(\mathbb{R})$, then $ad - bc = 1$. The inverse of $A$ is given by:
                \[
                A^{-1} = \begin{pmatrix} d & -b \\ -c & a \end{pmatrix}.
                \]
                Computing its determinant:
                \[
                \det(A^{-1}) = (d)(a) - (-b)(-c) = ad - bc = 1.
                \]
                Since $A^{-1} \in \text{SL}_2(\mathbb{R})$, every element has an inverse.
            \end{enumerate}
        Since all group axioms hold, $\text{SL}_2(\mathbb{R})$ is a group under matrix multiplication.
        \end{solution}
    \end{enumerate}
\end{problem}

\begin{problem}[2.19]
    Let $X$ be a set. We recall from Example 2.19 that the collection of all permutations (bijective functions)
    $\pi : X \rightarrow X$ forms the symmetry group $\mathcal{S}_X$ of $X$, where the group law is composition
    of functions. Suppose that we instead look at the set
    \[
        \epsilon_X = \{\text{functions} \phi : X \longrightarrow X\},
    \]
    so we no longer require that $\phi$ be bijective. We can define a composition law on $\epsilon_X$ using composition of functions. 
    \begin{enumerate}[label=(\alph*)]
        \item Prove that $\epsilon_X$ is a monoid. (See Exercise 2.8 for the definition of monoid.)
        
            \begin{solution}
                To prove that $\epsilon_X$ is a monoid, we must verify two properties:
                \begin{itemize}
                    \item Associativity
                    
                        Function composition is always associative. That is, for any $\phi, \psi, \theta \in \epsilon_X$,
                        \[
                        (\phi \circ \psi) \circ \theta = \phi \circ (\psi \circ \theta).
                        \]
                    \item Identity element
                        
                        The identity function $\text{id}_X: X \to X$ defined by $\text{id}_X(x) = x$ for all $x \in X$ is an element of $\epsilon_X$ and satisfies
                        \[
                        \text{id}_X \circ \phi = \phi \circ \text{id}_X = \phi
                        \]
                        for all $\phi \in \epsilon_X$.
                \end{itemize}
            Since $\epsilon_X$ is associative and has an identity element, it forms a monoid.
            
            \end{solution}
        \item If $X$ is a finite set with $n$ elements, prove that $\epsilon_X$ is a finite monoid and compute how many elements it has.
            
            \begin{solution}
                If $X$ has $n$ elements, then each function $\phi: X \to X$ assigns to each of the $n$ elements one of $n$ possible values in $X$. Thus, the number of possible functions is:
                \[
                \text{Total elements in } \epsilon_X = n^n.
                \]
                Since $\epsilon_X$ is finite and satisfies the monoid properties, it forms a finite monoid of order $n^n$.
            \end{solution}
        \item If $\#X\geq 3$, prove that $\epsilon_X$ is not commutative; i.e., show that there are elements $\phi, \psi \in \epsilon_X$ satisfying $\phi\circ\psi \neq \psi\circ\phi$
            
        \begin{solution}
            To show that $\epsilon_X$ is not commutative when $\#X \geq 3$, we provide a counterexample.

            Let $X = \{a, b, c\}$, and define two functions $\phi, \psi : X \to X$ as follows:
            \[
            \phi(a) = b, \quad \phi(b) = a, \quad \phi(c) = c
            \]
            \[
            \psi(a) = a, \quad \psi(b) = c, \quad \psi(c) = b.
            \]
            Now, we compute their compositions:
            \[
            (\phi \circ \psi)(a) = \phi(\psi(a)) = \phi(a) = b, 
            \]
            \[
            (\phi \circ \psi)(b) = \phi(\psi(b)) = \phi(c) = c, 
            \]
            \[
            (\phi \circ \psi)(c) = \phi(\psi(c)) = \phi(b) = a.
            \]
            On the other hand,
            \[
            (\psi \circ \phi)(a) = \psi(\phi(a)) = \psi(b) = c, 
            \]
            \[
            (\psi \circ \phi)(b) = \psi(\phi(b)) = \psi(a) = a, 
            \]
            \[
            (\psi \circ \phi)(c) = \psi(\phi(c)) = \psi(c) = b.
            \]
            Since $\phi \circ \psi \neq \psi \circ \phi$, $\epsilon_X$ is not commutative for $\#X \geq 3$.
            
        \end{solution}
    \end{enumerate}
\end{problem}

\begin{problem}[2.21]
    Let $G$ be a group, and consider the function
    \[\phi: G \longrightarrow G, \hspace{15px} \phi(g) = g^{-1}\]
    \begin{enumerate}[label=(\alph*)]
        \item Prove that $\phi(\phi(g)) = g$ for all $g \in G$.
        
        \begin{solution}
            By definition, $\phi(g) = g^{-1}$ for all $g \in G$. Applying $\phi$ again, we obtain:
            \[
            \phi(\phi(g)) = \phi(g^{-1}).
            \]
            Since $\phi$ takes an element to its inverse, we get:
            \[
            \phi(g^{-1}) = (g^{-1})^{-1}.
            \]
            By the group axioms, the inverse of $g^{-1}$ is just $g$. Hence,
            \[
            \phi(\phi(g)) = g.
            \]
            This proves that $\phi$ is an involution, meaning applying it twice returns the original element.
        
        \end{solution}
        \item Prove that $\phi$ is a bijection.
        
        \begin{solution}
            Gonna show that $\phi$ is both injective and surjective.

            \textbf{Injectivity}: Suppose $\phi(g_1) = \phi(g_2)$ for some $g_1, g_2 \in G$. Then:
            \[
            g_1^{-1} = g_2^{-1}.
            \]
            Applying the inverse operation to both sides, we obtain:
            \[
            (g_1^{-1})^{-1} = (g_2^{-1})^{-1},
            \]
            which simplifies to $g_1 = g_2$. Thus, $\phi$ is injective.
    
            \textbf{Surjectivity}: For any $h \in G$, we need to find some $g \in G$ such that $\phi(g) = h$. By definition, 
            \[
            \phi(g) = g^{-1}.
            \]
            Choosing $g = h^{-1}$, we see that:
            \[
            \phi(h^{-1}) = (h^{-1})^{-1} = h.
            \]
            Since such a $g$ exists for every $h \in G$, $\phi$ is surjective.
    
            Since $\phi$ is both injective and surjective, it is a bijection.
            
        \end{solution}
        \item Prove that $\phi$ is a group homomorphism if and only if $G$ is an abelian group.
        
        \begin{solution}
            The function $\phi$ is a group homomorphism if for all $g_1, g_2 \in G$, we have:
            \[
            \phi(g_1 g_2) = \phi(g_1) \phi(g_2).
            \]
            Expanding both sides using the definition of $\phi$:
            \[
            (g_1 g_2)^{-1} = g_2^{-1} g_1^{-1}.
            \]
            For this to be equal to $\phi(g_1) \phi(g_2) = g_1^{-1} g_2^{-1}$, we require:
            \[
            g_2^{-1} g_1^{-1} = g_1^{-1} g_2^{-1}.
            \]
            This equality holds if and only if $G$ is abelian, meaning $g_1 g_2 = g_2 g_1$ for all $g_1, g_2 \in G$.
    
            Thus, $\phi$ is a homomorphism if and only if $G$ is abelian.
            
        \end{solution}
    \end{enumerate}
\end{problem}

\begin{problem}[2.25]
    Let $\text{SL}_2(\mathbb{Z}/2\mathbb{Z})$ be the group that we defined in Exercise 2.18.
    \begin{enumerate}[label=(\alph*)]
        \item Prove that $\#\text{SL}_2(\mathbb{Z}/2\mathbb{Z}) = 6$
        
        \begin{solution}
            The special linear group $\text{SL}_2(\mathbb{Z}/2\mathbb{Z})$ consists of all $2 \times 2$ matrices with entries in $\mathbb{Z}/2\mathbb{Z}$ that have determinant equal to $1$ modulo $2$.

            The set $\mathbb{Z}/2\mathbb{Z}$ has only two elements, $\{0,1\}$, so the general form of a matrix in $\text{SL}_2(\mathbb{Z}/2\mathbb{Z})$ is:
            \[
            A = \begin{pmatrix} a & b \\ c & d \end{pmatrix}
            \]
            where $a, b, c, d \in \mathbb{Z}/2\mathbb{Z}$ and $\det(A) = ad - bc \equiv 1 \mod 2$.
    
            We enumerate all possible matrices satisfying this determinant condition:
            
            \begin{enumerate}[label=(\roman*)]
            \item If $a = 1, d = 1$, then $\det(A) = 1 - bc \equiv 1 \mod 2$ forces $bc = 0$. This gives the matrices:
              \[
              \begin{pmatrix} 1 & 0 \\ 0 & 1 \end{pmatrix}, \quad
              \begin{pmatrix} 1 & 1 \\ 0 & 1 \end{pmatrix}, \quad
              \begin{pmatrix} 1 & 0 \\ 1 & 1 \end{pmatrix}.
              \]
    
            \item If $a = 0, d = 0$, then $\det(A) = -bc \equiv 1 \mod 2$ is not possible since $bc$ must be $1$, but at least one of $a$ or $d$ is $0$.
    
            \item If $a = 1, d = 0$ or $a = 0, d = 1$, then $\det(A) = ad - bc \equiv 1 \mod 2$ simplifies to $bc \equiv 0 \mod 2$. This gives the matrices:
              \[
              \begin{pmatrix} 1 & 1 \\ 1 & 0 \end{pmatrix}, \quad
              \begin{pmatrix} 0 & 1 \\ 1 & 1 \end{pmatrix}, \quad
              \begin{pmatrix} 0 & 1 \\ 1 & 0 \end{pmatrix}.
              \]
            \end{enumerate}
    
            Listing all valid matrices, we find exactly six elements:
            \[
            \begin{pmatrix} 1 & 0 \\ 0 & 1 \end{pmatrix}, \quad
            \begin{pmatrix} 1 & 1 \\ 0 & 1 \end{pmatrix}, \quad
            \begin{pmatrix} 1 & 0 \\ 1 & 1 \end{pmatrix}, \quad
            \begin{pmatrix} 1 & 1 \\ 1 & 0 \end{pmatrix}, \quad
            \begin{pmatrix} 0 & 1 \\ 1 & 1 \end{pmatrix}, \quad
            \begin{pmatrix} 0 & 1 \\ 1 & 0 \end{pmatrix}.
            \]
    
            Thus, $\#\text{SL}_2(\mathbb{Z}/2\mathbb{Z}) = 6$.
            
        \end{solution}
        \item Prove that $\text{SL}_2(\mathbb{Z}/2\mathbb{Z})$ is isomorphic to the symmetric group $\mathcal{S}_3$. (\textit{Hint:} Show that the matrices in $\text{SL}_2(\mathbb{Z}/2\mathbb{Z})$) permute
              the vectors in the set $\{(1, 0), (0, 1), (1, 1)\}$, where the coordinates of the vectors are viewed as numbers modulo 2.)
        
        \begin{solution}
            Consider the set of column vectors:
            \[
            v_1 = \begin{pmatrix} 1 \\ 0 \end{pmatrix}, \quad
            v_2 = \begin{pmatrix} 0 \\ 1 \end{pmatrix}, \quad
            v_3 = \begin{pmatrix} 1 \\ 1 \end{pmatrix}.
            \]
            The matrices in $\text{SL}_2(\mathbb{Z}/2\mathbb{Z})$ act on this set by permutation.
    
            Computing the action of each matrix on the basis vectors:
            
            \begin{enumerate}[label=(\roman*)]
                \item The identity matrix $\begin{pmatrix} 1 & 0 \\ 0 & 1 \end{pmatrix}$ fixes all vectors.
                \item The matrix $\begin{pmatrix} 1 & 1 \\ 0 & 1 \end{pmatrix}$ sends:
                \[
                v_1 \mapsto v_3, \quad v_2 \mapsto v_2, \quad v_3 \mapsto v_1.
                \]
                This corresponds to the transposition $(13)$.
                \item The matrix $\begin{pmatrix} 1 & 0 \\ 1 & 1 \end{pmatrix}$ sends:
                \[
                v_1 \mapsto v_1, \quad v_2 \mapsto v_3, \quad v_3 \mapsto v_2.
                \]
                This corresponds to the transposition $(23)$.
                \item The matrix $\begin{pmatrix} 1 & 1 \\ 1 & 0 \end{pmatrix}$ sends:
                \[
                v_1 \mapsto v_2, \quad v_2 \mapsto v_1, \quad v_3 \mapsto v_3.
                \]
                This corresponds to the transposition $(12)$.
                \item The matrix $\begin{pmatrix} 0 & 1 \\ 1 & 1 \end{pmatrix}$ sends:
                \[
                v_1 \mapsto v_3, \quad v_2 \mapsto v_1, \quad v_3 \mapsto v_2.
                \]
                This corresponds to the 3-cycle $(132)$.
                \item The matrix $\begin{pmatrix} 0 & 1 \\ 1 & 0 \end{pmatrix}$ sends:
                \[
                v_1 \mapsto v_2, \quad v_2 \mapsto v_3, \quad v_3 \mapsto v_1.
                \]
                This corresponds to the 3-cycle $(123)$.
            \end{enumerate}
    
            Since these matrices permute the three vectors exactly as the elements of the symmetric group $\mathcal{S}_3$ do, we have a group homomorphism from $\text{SL}_2(\mathbb{Z}/2\mathbb{Z})$ to $\mathcal{S}_3$. Because both groups have exactly six elements, and the mapping is bijective, it is an isomorphism.
    
            Thus,
            \[
            \text{SL}_2(\mathbb{Z}/2\mathbb{Z}) \cong \mathcal{S}_3.
            \]
        \end{solution}
    \end{enumerate}
\end{problem}

\begin{problem}[2.28]
    Let $G$ be a group, and let $H \subset G$ be a subset of $G$. Prove that $H$ is a subgroup if and only if it has the following two properties:
    \begin{enumerate}
        \item $H \neq \emptyset$
        \item For every $h_1, h_2 \in H$, the product $h_1 \cdot h_2^{-1}$ is in $H$.
    \end{enumerate}

    \begin{solution}
        We need to prove that $H$ is a subgroup of $G$ if and only if it satisfies the given conditions.

        \textbf{($\Rightarrow$) Suppose $H$ is a subgroup of $G$}.  
        \begin{itemize}
            \item Since $H$ is a subgroup, it contains the identity element $e$ of $G$. Thus, $H$ is nonempty.  
            \item Since $H$ is closed under multiplication and contains inverses, for every $h_1, h_2 \in H$, we have $h_2^{-1} \in H$.  
            \item Since subgroups are closed under multiplication, the element $h_1 \cdot h_2^{-1}$ must also be in $H$.  
        \end{itemize}
        Hence, $H$ satisfies both conditions.
    
        \textbf{($\Leftarrow$) Suppose $H$ satisfies the given conditions.}  
        \begin{itemize}
            \item Since $H \neq \emptyset$, there exists at least one element $h \in H$.  
            \item Choosing $h_1 = h$ and $h_2 = h$, the second condition implies $h \cdot h^{-1} = e \in H$, so $H$ contains the identity.  
            \item To show closure under inverses, let $h \in H$. Taking $h_1 = e$ and $h_2 = h$, we get $e \cdot h^{-1} = h^{-1} \in H$. Thus, $H$ is closed under taking inverses.  
            \item Finally, for closure under multiplication, let $h_1, h_2 \in H$. Since $H$ contains inverses, we can replace $h_2$ with $h_2^{-1}$ in the given condition, giving us $h_1 \cdot (h_2^{-1})^{-1} = h_1 \cdot h_2 \in H$.  
        \end{itemize}
        Since $H$ contains the identity, is closed under inverses, and is closed under multiplication, it is a subgroup of $G$.
    
        The two given conditions are necessary and sufficient for $H$ to be a subgroup of $G$.
        
    \end{solution}
\end{problem}

\begin{problem}[2.30]
    This exercise generalizes the notion of the cyclic subgroup generated by an element of a group as described in Example 2.37.
    Let $G$ be a group, and let $S \subseteq G$ be a subset of $G$. The \textit{subgroup of} $G$ \textit{generated by} $S$, which
    we denote by $\langle S \rangle$, is the intersection of all of the subgroups of $G$ that contain $S$; i.e.,
    \[
        \langle S \rangle = \bigcap_{\substack{S \subseteq H \subset G\\ H \text{ is a subgroup of } G}} H.
    \]
    \begin{enumerate}[label=(\alph*)]
        \item Prove that $\langle S \rangle$ is not the empty set.
        
            \begin{solution}
                Note that $G$ is a group, and hence contains the identity element $e$. Every subgroup of $G$ must contain $e$, including all subgroups that contain $S$. Since $\langle S \rangle$ is defined as the intersection of all such subgroups, it must also contain $e$. Thus, $\langle S \rangle \neq \emptyset$.

            \end{solution}
        \item Prove that $\langle S \rangle$ is a subgroup of $G$.
        
            \begin{solution}
                Recall that a subset $H$ of $G$ is a subgroup if it satisfies the following:
                \begin{enumerate}[label=(\roman*)]
                    \item It contains the identity element.
                    \item It is closed under the group operation.
                    \item It is closed under inverses.
                \end{enumerate}
                Since each subgroup containing $S$ is a subgroup of $G$, it must satisfy these properties. The intersection of any collection of subgroups is itself a subgroup, since all properties are preserved under intersection. Since $\langle S \rangle$ is defined as such an intersection, it must also be a subgroup of $G$.
            
            \end{solution}
        \item Suppose that $K \subseteq G$ is a subgroup of $K$ and that $S \subseteq K$. Prove that $K \subset \langle S \rangle$. 
              Thus $\langle S \rangle$ is often described as being the smallest subgroup of $G$ that contains the set $S$.

            \begin{solution}
                Suppose that $K$ is a subgroup of $G$ containing $S$. By definition, $\langle S \rangle$ is the intersection of all subgroups containing $S$. Since $K$ is one of these subgroups, it follows that $\langle S \rangle \subseteq K$. Therefore, $\langle S \rangle$ is the smallest subgroup of $G$ containing $S$.

            \end{solution}
        \item Let $T$ be the set of inverses of the elements in $S$; i.e.,
              \[T = \{g^{-1}:g \in S\}\]
              Prove that $\langle S \rangle$ is equal to the following set of products:
              \[\langle S \rangle = \{g_1g_2 \cdots g_n : n \geq 0 \text{ and } g_1,\ldots,g_n \in S \cup T\}\]
              If $G = \langle S \rangle$, then we say that $S$ generates $G$, or that $S$ is a generating set for $G$.
            
            \begin{solution}
                Define 
                \[
                    H = \{ g_1 g_2 \cdots g_n : n \geq 0 \text{ and } g_1, g_2, \dots, g_n \in S \cup T \}.
                \]
                We will show that $H = \langle S \rangle$.
                \begin{itemize}
                    \item First, note that $H$ is a subgroup: It contains the identity element ($n=0$ gives the identity), it is closed under multiplication by construction, and since $T$ contains the inverses of elements in $S$, $H$ is closed under taking inverses.
                    \item Since $H$ contains $S$ and is a subgroup of $G$, it must contain $\langle S \rangle$, the smallest subgroup containing $S$, so $\langle S \rangle \subseteq H$.
                    \item Conversely, $\langle S \rangle$ must be closed under multiplication and taking inverses, so it contains all finite products of elements from $S \cup T$, meaning $H \subseteq \langle S \rangle$.
                \end{itemize}
                Since we have both $\langle S \rangle \subseteq H$ and $H \subseteq \langle S \rangle$, it follows that $H = \langle S \rangle$, completing the proof.
            
            \end{solution}
    \end{enumerate}
\end{problem}

\begin{problem}[2.33]
    Let $G$ be a group, let $A \subseteq G$ and $B \subseteq G$ be subgroups of $G$, and let $\phi$ be the map
    \[\phi : A \times B \longrightarrow G, \hspace{15px} \phi(a, b) = ab\]
    \begin{enumerate}[label=(\alph*)]
        \item Prove that $A \cap B = \{e\}$ if and only if the map $\phi$ is an injective map of sets.
        
        \begin{solution}
            Let's look at both directions:
            \begin{itemize}
                \item[$\impliedby$] Suppose $\phi$ is injective. If there exists some $x \in A \cap B$ with $x \neq e$, then $\phi(x, e) = x$ and $\phi(e, x) = x$, contradicting injectivity since $(x, e) \neq (e, x)$. Thus, $A \cap B = \{e\}$.
                \item[$\implies$] Conversely, assume $A \cap B = \{e\}$ and suppose $\phi(a_1, b_1) = \phi(a_2, b_2)$. Then $a_1 b_1 = a_2 b_2$, which rearranges to $a_1 a_2^{-1} = b_2 b_1^{-1}$. Since $a_1 a_2^{-1} \in A$ and $b_2 b_1^{-1} \in B$, and the only element common to both is $e$, we conclude $a_1 = a_2$ and $b_1 = b_2$, proving injectivity.
            \end{itemize}
        \end{solution}
        \item We can turn $A \times B$ into a group by using the group operation
              \[(a_1, b_1) \cdot (a_2, b_2) = (a_1 \cdot a_2, b_1 \cdot b_2)\]
              (See Section 2.6 for further details on product groups.) Prove that the map $\phi$ is a homomorphism of groups 
              if and only if every element of $A$ commutes with every element of $B$.
        
        \begin{solution}
            The map $\phi$ is a homomorphism if for all $(a_1, b_1), (a_2, b_2) \in A \times B$, we have:
            \[ \phi((a_1, b_1) \cdot (a_2, b_2)) = \phi(a_1, b_1) \cdot \phi(a_2, b_2). \]
            Expanding both sides:
            \[ \phi(a_1 a_2, b_1 b_2) = (a_1 a_2)(b_1 b_2). \]
            On the other hand,
            \[ \phi(a_1, b_1) \cdot \phi(a_2, b_2) = (a_1 b_1)(a_2 b_2). \]
            For these to be equal, we require that multiplication can be rearranged as:
            \[ (a_1 b_1)(a_2 b_2) = (a_1 a_2)(b_1 b_2). \]
            This holds if and only if $a_2 b_1 = b_1 a_2$ for all $a_2 \in A$ and $b_1 \in B$, which means every element of $A$ commutes with every element of $B$.
        
        \end{solution}
    \end{enumerate}
\end{problem}

\begin{problem}[2.36]
    Let $G$ be a group, and let $g \in G$. The \textit{centralizer of g}, denoted $Z_G(g)$, is the set of elements of $G$ that commute with $g$; i.e.,
    \[Z_G(g) = \{g' \in G: gg' = g'g\}\]
    \begin{enumerate}[label=(\alph*)]
    \item Prove that $Z_G(g)$ is a subgroup of $G$.
    
    \begin{solution}
        We prove that $Z_G(g)$ is a subgroup of $G$ by checking the subgroup criteria:
        \begin{itemize}
            \item The identity element $e$ satisfies $eg = ge$, so $e \in Z_G(g)$.
            \item If $g_1, g_2 \in Z_G(g)$, then $gg_1 = g_1g$ and $gg_2 = g_2g$. Multiplying these gives 
            \[ g(g_1 g_2) = (gg_1) g_2 = (g_1 g) g_2 = g_1 (g g_2) = g_1 (g_2 g) = (g_1 g_2) g, \]
            so $g_1 g_2 \in Z_G(g)$.
            \item If $g' \in Z_G(g)$, then $gg' = g'g$. Taking inverses, we get 
            \[ g'^{-1} g^{-1} = g^{-1} g'^{-1}. \]
            Since inverses exist in $G$, we conclude $g'^{-1} \in Z_G(g)$.
        \end{itemize}
        Thus, $Z_G(g)$ is a subgroup of $G$.
    \end{solution}
    \item Compute the centralizer $Z_G(g)$ for the following groups and elements:
        \begin{enumerate}[label=(\roman*)]
            \item $G = \mathcal{D}_4$ and $g$ is rotation by $90^{\circ}$
            
            \begin{solution}
                $Z_G(g)$ consists of elements that commute with $g$. These elements are the rotations (0, 90, 180, 270 degrees), forming the cyclic subgroup $\{e, g, g^2, g^3\}$.
            \end{solution}
            \item $G = \mathcal{D}_4$ and $f$ is a flip fixing two of the vertices of the square.
            
            \begin{solution}
                $Z_G(f)$ consists of elements that commute with $f$. These include $e, f$, the 180-degree rotation, and the other flip along the perpendicular axis.

            \end{solution}
            \item $G = \text{GL}_2(\mathbb{R})$ and $g = \begin{pmatrix} a & 0 \\ 0 & d \end{pmatrix}$
            
            \begin{solution}
                $Z_G(g)$ consists of all matrices that commute with $g$. These are precisely the diagonal and block diagonal matrices of the form \[ \begin{pmatrix} x & 0 \\ 0 & y \end{pmatrix}, \] meaning $Z_G(g)$ is the set of all such matrices in $\text{GL}_2(\mathbb{R})$.

            \end{solution}
        \end{enumerate}
    \end{enumerate}
\end{problem}

\begin{problem}[2.39]
    Let $G$ be a group, and let $K \subseteq H \subseteq G$ be subgroups. We may thus view $K$ as a subgroup of $G$ or as a subgroup of $H$.
    We also recall that the index of a subgroup is its number of distinct cosets; see Definition 2.49.
    \begin{enumerate}[label=(\alph*)]
        \item If $G$ is finite, prove the \textit{Index Multiplication Rule}
        \[(G:K) = (G:H)(H:K)\]
        (\textit{Hint.} Use (2.11) in Definition 2.49.)

        \begin{solution}
            To prove the Index Multiplication Rule, we analyze the number of cosets:
            \begin{itemize}
                \item By definition, $(G:H)$ is the number of left cosets of $H$ in $G$, and $(H:K)$ is the number of left cosets of $K$ in $H$.
                \item Each left coset of $H$ in $G$ can be written as $gH$ for some $g \in G$.
                \item Each left coset of $K$ in $H$ can be written as $hK$ for some $h \in H$.
                \item Consider the collection of left cosets of $K$ in $G$, which are of the form $gK$ for some $g \in G$.
                \item Since each coset $gH$ can be subdivided into $(H:K)$ cosets of $K$ in $G$, we obtain:
                \[
                (G:K) = (G:H)(H:K).
                \]
            \end{itemize}
        
        \end{solution}
        \item \textit{Challenge Problem.} Prove that the Index Multiplication Rule (2.13) is true even if $G, H,$ and $K$ are allowed to be infinite groups,
              provided that we assume that $(G : K)$ is finite. (\textit{Hint.} Take cosets for $H$ in $G$ and cosets for $K$ in $H$, and use them to build cosets for $K$ in $G$.)

        \begin{solution}
            We extend the proof to the case where $G, H$, and $K$ are infinite, assuming that $(G:K)$ is finite.
            \begin{itemize}
                \item Define the set of left cosets of $H$ in $G$ as $\{g_i H : i \in I\}$, where $I$ is an index set of size $(G:H)$.
                \item Similarly, define the set of left cosets of $K$ in $H$ as $\{h_j K : j \in J\}$, where $J$ is an index set of size $(H:K)$.
                \item Each element of $G$ can be written uniquely as $g_i h_j k$ for some $g_i H$, $h_j K$, and $k \in K$.
                \item Therefore, the number of left cosets of $K$ in $G$ is the product of the number of left cosets of $H$ in $G$ and the number of left cosets of $K$ in $H$:
                \[
                (G:K) = (G:H)(H:K).
                \]
            \end{itemize}
            This holds even when $G, H, K$ are infinite, provided $(G:K)$ is finite.
        
        \end{solution}
    \end{enumerate}
\end{problem}

\begin{problem}[2.44]
    Let $G_1$ and $G_2$ be groups, and let $p_1, p_2, \iota_1, \iota_2$ be the projection and inclusion maps described in Section 2.6.
    \begin{enumerate}[label=(\alph*)]
        \item Prove that $p_1$ and $p_2$ are homomorphisms.
        
        \begin{solution}
            The projection maps $p_1 : G_1 \times G_2 \to G_1$ and $p_2 : G_1 \times G_2 \to G_2$ are homomorphisms since for any $(a_1, b_1), (a_2, b_2) \in G_1 \times G_2$, we have:
            \[
            p_1((a_1, b_1)(a_2, b_2)) = p_1(a_1 a_2, b_1 b_2) = a_1 a_2 = p_1(a_1, b_1) p_1(a_2, b_2).
            \]
            A similar argument holds for $p_2$.
        \end{solution}
        \item Prove that $\iota_1$ and $\iota_2$ are homomorphisms.
        
        \begin{solution}
            The inclusion maps $\iota_1 : G_1 \to G_1 \times G_2$ and $\iota_2 : G_2 \to G_1 \times G_2$ are homomorphisms since for any $a_1, a_2 \in G_1$, we have:
            \[
            \iota_1(a_1 a_2) = (a_1 a_2, e) = (a_1, e)(a_2, e) = \iota_1(a_1) \iota_1(a_2).
            \]
            A similar argument holds for $\iota_2$.
        \end{solution}
        \item Compute the following compositions of these maps:
        
        \renewcommand{\arraystretch}{1} % Adjust row height for better spacing\
        \begin{tabular}{m{3.4cm} m{3.4cm} m{3.4cm} m{3.4cm}}
            $p_1 \circ \iota_1(a)$ & $p_2 \circ \iota_1(a)$ & $p_1 \circ \iota_2(b)$ & $p_2 \circ \iota_2(b)$ \\ 
        \end{tabular}

        \begin{solution} 
            \hspace{25px}$a$ \hspace{100px}$e$ \hspace{100px}$e$ \hspace{100px}$b$ 
        \end{solution}
            
        \begin{tabular}{m{3.4cm} m{3.4cm} m{3.4cm} m{3.4cm}}
            $\iota_1 \circ p_1(a, b)$ & $\iota_2 \circ p_1(a, b)$ & $\iota_1 \circ p_2(a, b)$ & $\iota_2 \circ p_2(a, b)$ \\ 
        \end{tabular}

        \begin{solution} 
            \hspace{18px}$(a, e)$ \hspace{80px}$(e, e)$ \hspace{80px}$(e, e)$ \hspace{83px}$(e, b)$ 
        \end{solution}
    \end{enumerate}
\end{problem}

\begin{problem}[2.45]
    Let $G_1$ and $G_2$ be groups, and let $p_1, p_2, \iota_1, \iota_2$ be the projection and inclusion maps described in Section 2.6.
    \begin{enumerate}[label=(\alph*)]
        \item Suppose that $G$ is some other group and that we are given group homomorphisms
              \[ \psi_1 : G \longrightarrow G_1 \hspace{15px} \psi_2 : G \longrightarrow G_2 \]
              Prove that there exists a unique group homomorphism
              \[ \phi : G \longrightarrow G_1 \times G_2 \]
              with the property that
              \[ p_1(\phi(g)) = \psi_1(g) \hspace{15px} p_2(\phi(g)) = \psi_2(g) \hspace{10px} \text{for all } g \in G.\]
            
              \begin{solution}
                Define $\phi: G \to G_1 \times G_2$ by
                \[ \phi(g) = (\psi_1(g), \psi_2(g)). \]
                We verify that $\phi$ is a group homomorphism. For any $g, h \in G$,
                \[ \phi(gh) = (\psi_1(gh), \psi_2(gh)) = (\psi_1(g)\psi_1(h), \psi_2(g)\psi_2(h)) = \phi(g)\phi(h), \]
                showing that $\phi$ is a homomorphism. Uniqueness follows because any such $\phi$ must satisfy $p_1(\phi(g)) = \psi_1(g)$ and $p_2(\phi(g)) = \psi_2(g)$, uniquely determining $\phi$.

            \end{solution}
        \item Suppose that $G$ is an abelian group, and suppose that we are given group homomorphisms
              \[ \lambda_1 : G_1 \longrightarrow G \hspace{10px}\text{and}\hspace{10px} \lambda_2 : G_2 \longrightarrow G.\]
              Prove that there exists a unique group homomorphism
              \[ \mu : G_1 \times G_2 \longrightarrow G \]
              with the property that for all $g_1 \in G_1$ and all $g_2 \in G_2$ we have
              \[ \mu(\iota_1(g_1)) = \lambda_1(g_1) \hspace{10px}\text{and}\hspace{10px} \mu(\iota_2(g_2)) = \lambda_2(g_2). \]
              (\textit{Hint.} First show that (2.14) uniquely determines $\mu$ as a map of sets, and then verify that $\mu$ is a homomorphism.)

              \begin{solution}
                Define $\mu: G_1 \times G_2 \to G$ by setting
                \[ \mu(g_1, g_2) = \lambda_1(g_1) \lambda_2(g_2). \]
                This function is uniquely determined because for any $(g_1, g_2) \in G_1 \times G_2$,
                \[ \mu(\iota_1(g_1)) = \lambda_1(g_1), \quad \mu(\iota_2(g_2)) = \lambda_2(g_2). \]
                To show that $\mu$ is a homomorphism, let $(g_1, g_2), (h_1, h_2) \in G_1 \times G_2$. Since $G$ is abelian,
                \[ \mu((g_1, g_2)(h_1, h_2)) = \mu(g_1h_1, g_2h_2) = \lambda_1(g_1h_1)\lambda_2(g_2h_2). \]
                Using the homomorphism property of $\lambda_1$ and $\lambda_2$,
                \[ \lambda_1(g_1h_1) = \lambda_1(g_1)\lambda_1(h_1), \quad \lambda_2(g_2h_2) = \lambda_2(g_2)\lambda_2(h_2). \]
                Since $G$ is abelian, we get
                \[ \mu(g_1, g_2)\mu(h_1, h_2) = \lambda_1(g_1)\lambda_2(g_2) \lambda_1(h_1)\lambda_2(h_2) = \lambda_1(g_1)\lambda_1(h_1)\lambda_2(g_2)\lambda_2(h_2), \]
                which equals $\mu((g_1, g_2)(h_1, h_2))$, proving that $\mu$ is a homomorphism.

              \end{solution}
        \item Continuing with the notation from (b), let $\Gamma$ be a non-abelian group, let
              \[G_1 = G_2 = G = \Gamma,\]
              and let
              \[ \lambda_1 : G_1 \longrightarrow G \hspace{5px}\text{and}\hspace{5px} \lambda_2 : G_2 \longrightarrow G \hspace{5px}\text{both be the identity map}\]
              Prove that there does \textit{not} exist a homomorphism $\mu : G_1 \times G_2 \longrightarrow G$ satisfying (2.14). (\textit{Hint.} As noted in (b),
              the map $\mu$ exists as a map of sets, so your task is to show that $\mu$ is not a group homomorphism.)

              \textit{Bonus:} Generalize Exercise 2.45 to a product $G_1 \times G_2 \times \cdots \times G_n$ of more than two groups.

              \begin{solution}
                Suppose $G_1 = G_2 = G = \Gamma$ and $\lambda_1, \lambda_2$ are identity maps. Then, for all $g_1, g_2 \in \Gamma$,
                \[ \mu(g_1, g_2) = g_1 g_2. \]
                If $\mu$ were a homomorphism, then for all $g_1, g_2, h_1, h_2 \in \Gamma$,
                \[ \mu((g_1, g_2)(h_1, h_2)) = \mu(g_1 h_1, g_2 h_2) = g_1 h_1 g_2 h_2. \]
                However, applying $\mu$ separately,
                \[ \mu(g_1, g_2) \mu(h_1, h_2) = (g_1 g_2)(h_1 h_2). \]
                Since $\Gamma$ is non-abelian, in general $g_2 h_1 \neq h_1 g_2$, meaning the two expressions do not always match. Thus, $\mu$ is not a homomorphism.

              \end{solution}
    \end{enumerate}

\end{problem}





\end{document}