\documentclass[12pt]{article}
\usepackage[margin=1in]{geometry}

\usepackage{libertine}
\usepackage{parskip}
\usepackage{enumitem}
\usepackage{array}
\usepackage{graphicx}
\graphicspath{ {./images/} }

\usepackage{amsthm,amsmath,amssymb}
\usepackage{tikz}
\usetikzlibrary{arrows,automata,shapes.geometric}

\usepackage{pgfplots}
\pgfplotsset{compat=1.18}

\pagestyle{plain}
\thispagestyle{empty}

\definecolor{carnellian}{RGB}{190,20,20}

\theoremstyle{definition}
\newtheorem{problem}{Problem}

\newcounter{subq}[problem]
\newenvironment{subproblem}
{\refstepcounter{subq} \begin{itemize} \item[(\alph{subq})]}
{\end{itemize} \medskip}
\DeclareMathOperator{\Ima}{Im}
\DeclareMathOperator{\rank}{rank}
\DeclareMathOperator{\tr}{tr}
\DeclareMathOperator{\Hom}{Hom}
\DeclareMathOperator{\End}{End}
\DeclareMathOperator{\Span}{Span}
\DeclareMathOperator{\Aut}{Aut}
\DeclareMathOperator{\Ann}{Ann}
\DeclareMathOperator{\Mat}{Mat}
%\DeclareMathOperator{\deg}{deg}



\usepackage{environ}
\NewEnviron{solution}[1][\vfill]{
    \textcolor{blue}{\BODY}
}

\newcommand{\hwnum}{6}
\newcommand{\duedate}{3/18/2025}
\renewcommand{\title}{Fields}

\begin{document}

\hspace{-10px}
\begin{tabular*}{\textwidth}{l @{\extracolsep{\fill}} r}
    \textbf{Honors Algebra} & \textbf{Spring 2025} \\
    \textbf{HW \hwnum : \title} &  \textbf{\duedate} \\
\end{tabular*}

\vspace{1cm}

\textit{Abstract Algebra: An Integrated Approach by J.H. Silverman.}\\
Page 180-186: 7.14, 7.22, 7.29\\
Page 214-220: 8.3, 8.8, 8.21, 8.23\\
Page 320-325: 10.6, 10.12\\
Page 357-370: 11.2, 11.7, 11.8


\vspace{1cm}

%-----------TEMPLATE------------%
% \begin{problem}
%     \begin{enumerate}[label=(\alph*)]
%         \item 
%         \begin{solution}

%         \end{solution}

%         \item 
%         \begin{solution}

%         \end{solution}
%     \end{enumerate}
% \end{problem}

\begin{problem}[7.14]
    Let $R$ be a commutative ring.
    \begin{enumerate}[label=(\alph*)]
        \item Suppose that $a, b \in R$ have the property that $aR + bR = R$. 
              Prove that for all $m, n \geq 1$ we have 
              \[
                a^m R + b^n R = R
              \]
        \begin{solution}

        \end{solution}

        \item More generally, let $a_1, \ldots, a_t \in R$, and let $e_1, \ldots, e_t \geq 1$
              be positive integers. Prove that
              \[
                a_1 R + a_2 R + \cdots + a_t R = R \hspace{5px} \iff \hspace{5px} a_1^{e_1}R + a_2^{e_2}R + \cdots + a_t^{e_t}R = R
              \]
        \begin{solution}

        \end{solution}
    \end{enumerate}
\end{problem}

\begin{problem}[7.22]
    Let $R$ be a ring, let $P \subset R$ be a prime ideal, let $S = R \ P$ be the complement of $P$, let $R_S$
    be the localization ring as described in Exercise 7.21, and let
    \[
        Q = \{ (a, b) \in R_S : a \in P \}
    \]
    Prove that $Q$ is the unique maximal ideal of $R_S$. (A ring with a unique maximal ideal is called a local
    ring; see Exercise 3.53).

    \begin{solution}

    \end{solution}
\end{problem}

\begin{problem}[7.29]
    A polynomial $f(X_1, \ldots, X_n) \in F[X_1, \ldots, X_n]$ is said to be homogeneous of degree k if
    \[
        f(aX_1, \ldots, aX_n) = a^kf(X_1, \ldots, X_n) \hspace{5px} \text{for all } a \in F
    \]
    \begin{enumerate}[label=(\alph*)]
        \item Prove that $f$ is a homogeneous polynomial of degree $k$ if and only if $f$ is a sum of the form
              \[
                f(X_1, \ldots, X_n) = \sum_{\substack{i_1, i_2, \ldots, i_n \geq 0\\ i_1 + i_2 + \cdots + i_n = k}} c_{i_1, i_2, \ldots, i_n} X_1^{i_1} X_2^{i_2} \cdots X_n^{i_n}
              \]
        \begin{solution}

        \end{solution}

        \item Prove that the elementary symmetric polynomials $s_k (X_1, \ldots, X_n)$ described in Definition 7.40
              is a homogeneous polynomial of degree $k$.
        \begin{solution}

        \end{solution}

        \item Let $f(X_1, \ldots, X_n) \in F(X_1, \ldots, X_n)$ be homogeneous of degree $k$. Prove that
              \[
                X_1 \frac{\partial{f}}{X_1} + X_2 \frac{\partial{f}}{X_2} + \cdots + X_n \frac{\partial{f}}{X_n} = kf
              \]
              (\textit{Hint.} If you view
              \[
                f(TX_1, \ldots, TX_n) = T^k f(X_1, \ldots, X_n)
              \]
              as being a relation in the polynomial ring $F[T, X_1, \ldots, X_n]$, then you can differentiate it with
              respect to $T$. Then set $T = 1$.)
        \begin{solution}

        \end{solution}
    \end{enumerate}
\end{problem}

\begin{problem}[8.3]
    This exercise sketches a proof of the following result, which says that if a number is the root of 
    a polynomial in $Q[x]$, then it cannot be too closely approximated by rational numbers.

    \textbf{Theorem 8.46} Let $f(x) \in Q[x]$ be a polynomial ofdegree $d \geq 1$. There is a positive 
    constant $C_f > 0$ such that if $\alpha \in mathbb{C} \ \mathbb{Q}$ is a non-rational root of $f(x)$, then
    \[
        \Bigg| \frac{p}{q} - \alpha \Bigg| \geq \frac{C_f}{q^d} \hspace{5px} \text{for all } \frac{p}{q} \in \mathbb{Q}
    \]
    \begin{enumerate}[label=(\alph*)]
        \item Prove that every $p/q \in mathbb{Q}$ satisfies either
        \[
            f\bigg( \frac{p}{q} \bigg) = 0 \hspace{5px} \text{or} \hspace{5px} \Bigg| f\bigg( \frac{p}{q} \bigg) \Bigg| \geq \frac{1}{q^d}
        \]
        \begin{solution}

        \end{solution}

        \item Let $g(x) \in \mathbb{C}[x]$ be a polynomial of degree $e$, and let $\alpha \in \mathbb{C}$. Prove that there
              is a constant $A_{g, \alpha}$ so that
              \[
                |g(\beta)| \leq A_{g, \alpha} \max\{1, |\beta - \alpha|^e\} \hspace{5px} \beta \in \mathbb{C}
              \]
            (\textit{Hint.} Expand $g(x)$ as a sum of powers of $x - \alpha$)
        \begin{solution}

        \end{solution}

        \item Use (a) and (b) to prove Theorem 8.46. (\textit{Hint.} Since we are given that $f(\alpha) = 0$),
              we can factor $f(x)$ as $f(x) = (x - \alpha)g(x)$ for some $g(x) \in \mathbb{C}[x]$.)
        \begin{solution}

        \end{solution}
    \end{enumerate}
\end{problem}

\begin{problem}[8.8]
    Let $F$ be a finite field of order $q$, and assume that $q$ is odd.
    \begin{enumerate}[label=(\alph*)]
        \item Let $a, b \in F*$. If $a^2 = b^2$, prove that either $a = b$ or $a = -b$.
        \begin{solution}

        \end{solution}

        \item Show by way of an example that (a) is not true for the rings $\mathbb{Z}/8\mathbb{Z}$
              and $\mathbb{Z}/15\mathbb{Z}$.
        \begin{solution}

        \end{solution}

        \item Let 
        \[
            \mathcal{R} = \{ a^2 : a \in F* \} \hspace{5px} \text{and} \hspace{5px} \mathcal{N} = \{ b \in F* : b \not\in \mathcal{R} \}
        \]
        be, respectively, the set of squares and non-squares in $F*$. Prove that $\mathcal{R}$ and $\mathcal{N}$ each contain
        exactly $(q - 1)/2$ distinct elements.
        \begin{solution}

        \end{solution}

        \item Let $f(x)$ be the polynomial
              \[
                f(x) = x^{\frac{q-1}{2}} - 1
              \]
              Prove that $\mathcal{R}$ is exactly the set of roots of $f(x)$ in $F$. (\textit{Hint.} Use Lagrange
              to prove that the elements of $\mathcal{R}$ are roots. Then use (c) and Theorem 8.8(c).)
        \begin{solution}

        \end{solution}

        \item Let $c \in F*$. Prove that
              \[ 
                c^{\frac{q-1}{2}} \equiv 
                \begin{cases} 
                    1 & \text{if } c \in \mathcal{Q}\\
                    -1 & \text{if } c \in \mathcal{N}
                \end{cases}
              \]
              (\textit{Hint.} Lagrange says that every element of $F*$ is a root of $x^{q-1} - 1$.
              Factor this polynomial as $f(x)g(x)$ and use (d).)
        \begin{solution}

        \end{solution}

        \item Let $a_1, a_2 \in \mathcal{R}$ and $b_1, b_2 \in \mathcal{N}$. Prove that
              \[
                a_1a_2 \in \mathcal{R} \hspace{5px} \text{and} \hspace{5px} b_1b_2 \in \mathcal{R}
              \]
              The first of these facts is hardly surprising, since indeed, the product of two squares is
              a square in any commutative ring. But the second fact is surprising, since in most rings, 
              most products of non-square won't be squares. 
        \begin{solution}

        \end{solution}
    \end{enumerate}
\end{problem}

\begin{problem}[8.21]
    \begin{enumerate}[label=(\alph*)]
        \item 
        \begin{solution}

        \end{solution}

        \item 
        \begin{solution}

        \end{solution}
    \end{enumerate}
\end{problem}

\begin{problem}[8.23]
    \begin{enumerate}[label=(\alph*)]
        \item 
        \begin{solution}

        \end{solution}

        \item 
        \begin{solution}

        \end{solution}
    \end{enumerate}
\end{problem}

\begin{problem}[10.6]
    \begin{enumerate}[label=(\alph*)]
        \item 
        \begin{solution}

        \end{solution}

        \item 
        \begin{solution}

        \end{solution}
    \end{enumerate}
\end{problem}

\begin{problem}[10.12]
    \begin{enumerate}[label=(\alph*)]
        \item 
        \begin{solution}

        \end{solution}

        \item 
        \begin{solution}

        \end{solution}
    \end{enumerate}
\end{problem}

\begin{problem}[11.2]
    \begin{enumerate}[label=(\alph*)]
        \item 
        \begin{solution}

        \end{solution}

        \item 
        \begin{solution}

        \end{solution}
    \end{enumerate}
\end{problem}

\begin{problem}[11.7]
    \begin{enumerate}[label=(\alph*)]
        \item 
        \begin{solution}

        \end{solution}

        \item 
        \begin{solution}

        \end{solution}
    \end{enumerate}
\end{problem}

\begin{problem}[11.8]
    \begin{enumerate}[label=(\alph*)]
        \item 
        \begin{solution}

        \end{solution}

        \item 
        \begin{solution}

        \end{solution}
    \end{enumerate}
\end{problem}

\end{document}