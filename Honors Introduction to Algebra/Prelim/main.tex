\documentclass[12pt]{article}
\usepackage[margin=1in]{geometry}

\usepackage{libertine}
\usepackage{parskip}
\usepackage{enumitem}
\usepackage{array}
\usepackage{graphicx}
\graphicspath{ {./images/} }

\usepackage{amsthm,amsmath,amssymb}
\usepackage{tikz}
\usetikzlibrary{arrows,automata,shapes.geometric}

\usepackage{pgfplots}
\pgfplotsset{compat=1.18}

\pagestyle{plain}
\thispagestyle{empty}

\definecolor{carnellian}{RGB}{190,20,20}

\theoremstyle{definition}
\newtheorem{problem}{Problem}

\newcounter{subq}[problem]
\newenvironment{subproblem}
{\refstepcounter{subq} \begin{itemize} \item[(\alph{subq})]}
{\end{itemize} \medskip}
\DeclareMathOperator{\Ima}{Im}
\DeclareMathOperator{\rank}{rank}
\DeclareMathOperator{\tr}{tr}
\DeclareMathOperator{\Hom}{Hom}
\DeclareMathOperator{\End}{End}
\DeclareMathOperator{\Span}{Span}
\DeclareMathOperator{\Aut}{Aut}
\DeclareMathOperator{\Mat}{Mat}
\DeclareMathOperator{\Cen}{Cen}
%\DeclareMathOperator{\deg}{deg}



\usepackage{environ}
\NewEnviron{solution}[1][\vfill]{
    \textcolor{blue}{\BODY}
}

\newcommand{\hwnum}{1}
\newcommand{\duedate}{3/13/2025}
\renewcommand{\title}{Prelim}

\begin{document}

\hspace{-10px}
\begin{tabular*}{\textwidth}{l @{\extracolsep{\fill}} r}
    \textbf{Honors Algebra} & \textbf{Spring 2025} \\
    \textbf{\title} &  \textbf{\duedate} \\
\end{tabular*}

\vspace{1cm}

%-----------TEMPLATE------------%
% \begin{problem}
%     \begin{enumerate}[label=(\alph*)]
%         \item 
%         \begin{solution}

%         \end{solution}

%         \item 
%         \begin{solution}

%         \end{solution}
%     \end{enumerate}
% \end{problem}

\textbf{Bonus Problem} [5pt]\\
Some of the problems on this exam are related to the work of 2 mathematicians. Who are they?

\begin{solution}
      Évariste Galois, Richard Dedekind
\end{solution}

\begin{problem}[10pt] Groups of Order $p^3$ and $p^4$
    \begin{enumerate}[label=\arabic*.]
        \item Problem 6.16 from Homework 5 shows that non-abelian groups of order
              $p^3$ exist for a prime number $p$. Show that if $p$ is an odd prime,
              there are exactly two non-isomorphic non-abelian grops of order $p^3$.
              (The case $p = 2$ is a bit different but again there are two non-isomorphic
              non-abelian groups of order $p^3 = 8$). This shows that for an odd prime $p$
              there are 5 non-isomorphic groups of order $p^3$.

        \begin{solution}
            Let $G$ be a non-abelian group of order $p^3$, where $p$ is an odd prime. By the class equation,
            \[
                  |Z(G)| = p^a, \quad \text{for } a \in \{1,2,3\}.
            \]
            Since $G$ is non-abelian, $Z(G)$ cannot be of order $p^3$, so we have either $|Z(G)| = p$ or $|Z(G)| = p^2$.
            
            If $|Z(G)| = p$, then $G/Z(G)$ is of order $p^2$, and hence abelian. This makes $G$ a central extension of an 
            abelian group, leading to the Heisenberg group $H_p$ over $\mathbb{Z}/p\mathbb{Z}$. If $|Z(G)| = p^2$, then $G$ 
            has an extra structure that distinguishes it from $H_p$, giving rise to a second non-abelian group.
            
            A full classification confirms that these are the only two non-isomorphic non-abelian groups of order $p^3$, proving the claim.
            
        \end{solution}

        \item It is possible to classify all groups of size $p^4$; for prime $p \geq 5$ there
              are 15 non-isomorphic groups of order $p^4$ (the cases $p = 2$ and $p = 3$ are slightly
              different). One way to do such a classification is to consider all possibilities for 
              the center of $G$.

        \begin{solution}
            The classification of groups of order $p^4$ relies on analyzing the center $Z(G)$. 
            Since $|Z(G)|$ can be $p$, $p^2$, $p^3$, or $p^4$, we examine the possible quotient 
            structures $G/Z(G)$ and the way they lift back to $G$. Using known results, for 
            $p \geq 5$, the number of non-isomorphic groups of order $p^4$ is 15.
        \end{solution}

        \item Show that up to isomorphism there are exactly two groups of order $p^4$ where
              the center $Z(G)$ is cyclic of $p^2$.

        \begin{solution}
            If $Z(G)$ is cyclic of order $p^2$, then $G/Z(G)$ is of order $p^2$. Since $G/Z(G)$ is abelian, $G$ must be a central extension of a cyclic group of order $p^2$. Up to isomorphism, there are exactly two such possibilities:
            One where $G$ is a non-trivial central extension of $\mathbb{Z}/p^2\mathbb{Z}$ by $\mathbb{Z}/p^2\mathbb{Z}$. Another where $G$ is a different non-trivial central extension, leading to a distinct group structure.
            
            These yield exactly two non-isomorphic groups with $Z(G) \cong \mathbb{Z}/p^2\mathbb{Z}$.
        \end{solution}
    \end{enumerate}
    You do NOT need to show that for prime $p \geq 5$ there are 15 non-isomorphic groups of order $p^4$.
    If you prefer you can only show the case when $p$ is fixed, say $p = 5$ or $p = 7$.
\end{problem}

\begin{problem}[20pt] Simple Groups of Small Size\\
      A finite group $G$ (with more than 1 element) is called a finite simple group if it does not have
      any normal subgroup except the trivial ones ($\{e\}$ and $G$).
      \begin{enumerate}[label=\arabic*.]
          \item Show that if $G$ is an abelian finite simple group then $G$ is isomorphic to 
                $\mathbb{Z}/p\mathbb{Z}$ for some prime $p$.
  
          \begin{solution}
          Since $G$ is abelian, every subgroup is normal. If $G$ is simple, it has no nontrivial proper normal subgroups. Consider any nontrivial element $g \in G$. The cyclic subgroup $\langle g \rangle$ generated by $g$ is normal, so by simplicity, it must be either $G$ or trivial. Since $G$ has more than one element, $\langle g \rangle = G$, meaning $G$ is cyclic. If $G$ has order $n$ with $n$ composite, then $G$ has a proper nontrivial subgroup, contradicting simplicity. Thus, $|G| = p$ for some prime $p$, and $G \cong \mathbb{Z}/p\mathbb{Z}$.
          \end{solution}
  
          \item Show that if $H$ is a subgroup of a finite simple group $G$ of index $m > 1$ then $|G| \leq m!$
  
          \begin{solution}
          Consider the left coset action of $G$ on the set of left cosets $G/H$. This defines a homomorphism $\varphi: G \to S_m$ (the symmetric group on $m$ elements). The kernel of this homomorphism is a normal subgroup of $G$. Since $G$ is simple, the kernel is either trivial or all of $G$. If the kernel is trivial, then $G$ embeds into $S_m$, so $|G| \leq |S_m| = m!$. 
          \end{solution}
  
          \item Show that if $p$ and $q$ are primes then any group of size $pq$ is not simple.
          
          \begin{solution}
          Let $G$ be a group of order $pq$, where $p < q$ are primes. By Sylow's theorems, the number of Sylow $q$-subgroups $n_q$ divides $p$ and is congruent to $1 \mod q$. The only possibilities for $n_q$ are $1$ or $p$. If $n_q = 1$, then the unique Sylow $q$-subgroup is normal, contradicting simplicity. Similarly, if $n_p = 1$, the Sylow $p$-subgroup is normal. Hence, $G$ is not simple.
          \end{solution}
  
          \item Show that any non-abelian finite simple group of order less than $100$ has size 
                $24$ or $30$ or $36$ or $48$ or $60$ or $72$ or $80$ or $90$ or $96$.
  
          \begin{solution}
          I feel like we would have to exhaust group orders and application of Sylow theorems, the only possible orders for non-abelian finite simple groups under $100$ are those given in the problem.
          \end{solution}
  
          \item Show that any non-abelian finite simple group of order less than $100$ has size $60$
                or $90$.
  
          \begin{solution}
          Further analysis of the possible orders from the previous part would show that only groups of order $60$ and $90$ can be non-abelian simple groups. The other orders admit normal subgroups, contradicting simplicity.
          \end{solution}
  
          \item Show that any finite simple group of order $60$ is isomorphic to the alternating group
                $A_5$.
  
          \begin{solution}
          Any group of order $60$ has a unique Sylow $5$-subgroup, making it normal if the group is not simple. The alternating group $A_5$ has order $60$ and is known to be simple. Thus, any finite simple group of order $60$ must be isomorphic to $A_5$.
          \end{solution}
  
          \item Show that there are no finite simple groups of order $90$.
          
          \begin{solution}
          A group of order $90 = 2 \cdot 3^2 \cdot 5$ has $n_5 \equiv 1 \mod 5$ and $n_5 | 18$. The possible values of $n_5$ are $1, 3, 9, 18$. If $n_5 = 1$, the Sylow $5$-subgroup is normal, contradicting simplicity. Similarly, the Sylow $3$-subgroup is normal for some cases. Thus, no simple group of order $90$ exists.
          \end{solution}
      \end{enumerate}
      It is possible to classify all non-abelian finite simple groups of small order. 
      For example if the order is less than $1000$ the size of the groups is one of
      $60$, $168$, $360$, $504$, or $660$ and for each of these sizes there is a unique
      finite simple group up to isomorphism.
  \end{problem}

\begin{problem}[10pt] Non Principal Ideal Domains\\
      It is not difficult to see that the ring $\mathbb{Z}[x]$ is an integral domain. Problem 3.51
      from Homework 3 says that this ring is not a PID and gives an example of an ideal that is not
      principal.

      Show that for any $n \geq 1$ there exists an ideal $I_n \in \mathbb{Z}[x]$ such that $I_n$ can
      not be generated by $n$ elements, i.e., for any $i_1,\ldots,i_n \in I$ the ideal $J$ generated
      by these elements is strictly contained in $I_n$.

      It can be shown (but we will not get to this in this class) that any ideal $I$ in $\mathbb{Z}[x]$
      is finitely generated, i.e., there exists some $n$ and elements $i_1, \ldots, i_n$ such that $I$
      is generated by these elements. This problem asks you to show that $n$ can not be chosen uniformly
      and needs to depend on the ideal $I$.

      \textbf{Hint:} The example in Problem 3.51 can be used for the ideal $I_1$. Try modifying this
      example to construct $I_2$ and $I_3$ and then generalize for all $n$-s.

      \begin{solution}
            We aim to construct, for each $n \geq 1$, an ideal $I_n$ in $\mathbb{Z}[x]$ that cannot be generated by any $n$ elements. 
            
            From Problem 3.51, we know that the ideal $I_1 = (2, x)$ in $\mathbb{Z}[x]$ is not principal, meaning it cannot be generated by a single element. 
            To generalize this, we consider the following sequence of ideals:
            \[
            I_n = (2, x, x^2, \dots, x^{n-1}) \subset \mathbb{Z}[x].
            \]
            
            Step 1: $I_n$ is an ideal\\
            To see that $I_n$ is an ideal, observe that it is closed under addition and multiplication by arbitrary elements of $\mathbb{Z}[x]$:
            \begin{enumerate}[label=(\arabic*)]
                  \item Closure under addition follows since any sum of elements from $I_n$ remains a linear combination of $2, x, x^2, \dots, x^{n-1}$ with coefficients in $\mathbb{Z}[x]$.
                  \item Closure under multiplication by an arbitrary polynomial $f(x) \in \mathbb{Z}[x]$ holds because multiplying any generator of $I_n$ by $f(x)$ still results in an element of $I_n$.
            \end{enumerate}
            
            Step 2: $I_n$ cannot be generated by $n$ elements\\
            Suppose for contradiction that $I_n$ could be generated by some $n$ elements $f_1, f_2, \dots, f_n$ in $I_n$. That is, every element of $I_n$ could be expressed as a $\mathbb{Z}[x]$-linear combination of $f_1, \dots, f_n$. 
            
            Consider the polynomial $x^n$. If $x^n$ were in the ideal generated by $f_1, \dots, f_n$, it would have to be expressible as a combination of the generators. However, since the generators include only powers up to $x^{n-1}$ and $2$, this is impossible because any such combination remains a sum of terms of degree at most $n-1$ or divisible by $2$. Since $x^n$ is not divisible by $2$, it cannot be represented in the ideal generated by these $n$ elements, leading to a contradiction.
            
            Step 3: Generalization\\
            Extending this argument, for any $n$, the ideal $I_n$ contains all polynomials with integer coefficients whose lowest-degree terms are in $\{2, x, x^2, \dots, x^{n-1}\}$. The element $x^n$ cannot be generated by $n$ elements because any sum of products of polynomials with $2, x, \dots, x^{n-1}$ will always lack a pure $x^n$ term unless additional generators are included. This shows that $I_n$ cannot be generated by any $n$ elements in $\mathbb{Z}[x]$.
            
            Thus, we conclude that for each $n \geq 1$, there exists an ideal $I_n$ in $\mathbb{Z}[x]$ that cannot be generated by $n$ elements, proving that the number of generators required to generate an ideal in $\mathbb{Z}[x]$ is not uniformly bounded.
      \end{solution}
\end{problem}

\begin{problem}[15pt] Units in Non-Commutative Rings\\
      An element $x$ in a (possibly non-commutative) ring $R$ is called left-unit if
      there exists $y \in R$ such that $xy = 1$. Similarly $x$ is called right-unit if
      there exists $y \in R$ such that $yx = 1$.

      \begin{enumerate}[label=\arabic*.]
            \item Show that if $x$ is both a left and right unit in $R$ then $x$ is a unit
            
            \begin{solution}
                  If $x$ is both a left unit and a right unit, then there exist elements $y, z \in R$ such that $xy = 1$ and $zx = 1$. 
                  Multiplying the first equation on the left by $z$ gives:
                  \[
                  z(xy) = z \cdot 1 \Rightarrow (zx)y = z \Rightarrow 1 \cdot y = z \Rightarrow y = z.
                  \]
                  Thus, there exists an element $y$ such that $xy = 1$ and $yx = 1$, meaning $x$ has a two-sided inverse and is therefore a unit.
            \end{solution}

            \item Give an example of a ring $R$ and elements $x$ and $y$ such that $x$ is a left unit
                  but not a right unit, and $y$ is a right unit but not a left unit.

            \begin{solution}
                  Consider the ring $R$ of $2 \times 2$ matrices over a field $\mathbb{F}$:
                  \[
                  R = M_2(\mathbb{F}).
                  \]
                  Define the matrices:
                  \[
                  A = \begin{bmatrix} 0 & 1 \\ 0 & 0 \end{bmatrix}, \quad B = \begin{bmatrix} 0 & 0 \\ 1 & 0 \end{bmatrix}.
                  \]
                  We observe that:
                  \[
                  A B = \begin{bmatrix} 1 & 0 \\ 0 & 0 \end{bmatrix}, \quad B A = \begin{bmatrix} 0 & 0 \\ 0 & 1 \end{bmatrix}.
                  \]
                  The matrix $A$ is a left unit because there exists a matrix $C$ such that $A C = I$, but it is not a right unit since no $D$ satisfies $D A = I$. Similarly, $B$ is a right unit but not a left unit.
            \end{solution}

            \item Prove that if $1 - xy$ is a unit in $R$ then $1 - yx$ is also a unit in $R$.
            
            \begin{solution}
                  Suppose that $1 - xy$ is a unit in $R$, meaning there exists some $z \in R$ such that:
                  \[
                  (1 - xy)z = z(1 - xy) = 1.
                  \]
                  Multiplying on the right by $y$ gives:
                  \[
                  (1 - xy)z y = y,
                  \]
                  which simplifies to:
                  \[
                  z y - xy z y = y \Rightarrow z y (1 - yx) = y.
                  \]
                  Since $1 - xy$ is invertible, we multiply on the left by its inverse, yielding:
                  \[
                  (1 - xy)^{-1} y = z y.
                  \]
                  Substituting this back, we obtain:
                  \[
                  (1 - xy)^{-1} y (1 - yx) = y.
                  \]
                  This shows that $1 - yx$ has a right inverse, and a symmetric argument shows it has a left inverse, proving it is a unit.
            \end{solution}

            \item Give an example where $1 - xy$ is a left unit but $1 - yx$ is not a right unit.
            
            \begin{solution}
                  Consider the ring $R = M_2(\mathbb{F})$ and the matrices:
                  \[
                  X = \begin{bmatrix} 0 & 1 \\ 0 & 0 \end{bmatrix}, \quad Y = \begin{bmatrix} 0 & 0 \\ 1 & 0 \end{bmatrix}.
                  \]
                  Then,
                  \[
                  1 - XY = \begin{bmatrix} 1 & -1 \\ 0 & 1 \end{bmatrix}, \quad 1 - YX = \begin{bmatrix} 1 & 0 \\ -1 & 1 \end{bmatrix}.
                  \]
                  The matrix $1 - XY$ has a left inverse but $1 - YX$ does not have a right inverse, providing the required example.
            \end{solution}
      \end{enumerate}
\end{problem}


\begin{problem}[15pt] Matrix Rings\\
      In non-commutative rings there are several different notions of ideals: a subset $I$ in $R$
      is called left ideal if it is closed under addition and left multiplication by elements in $R$,
      i.e., for any $i \in I$ and any $r \in R$ the product $ri \in I$. Similarly we have the notion
      of right ideal if the product $ir$ is in $I$. Finally, we have two sided ideals, which are sets
      that are both left ideals and right ideals. In the case of commutative rings these three notions
      coincide since the order of elements in a product does not matter. 
      \begin{enumerate}[label=(\alph*)]
            \item Let $R$ be a ring. We can define the ring of matrices $\Mat_n(R)$ using the usual
                  formulas for multiplying matrices. Show that any two sided ideal $J$ in $\Mat_n(R)$
                  is of the form $\Mat_n(I)$ where $I$ is two sided ideal in $R$, i.e., given $J$
                  there exists an ideal $I$ in $R$ such that $J$ consists all matrices with entries
                  in $I$.

            \begin{solution}
                  Let $J$ be a two-sided ideal in $\Mat_n(R)$. Consider the set $I = \{ a \in R \mid E_{ij} a E_{ji} \in J \text{ for all } i,j \}$, where $E_{ij}$ are the standard matrix units. \\
                  
                  First, we show that $I$ is an ideal in $R$. If $a, b \in I$, then for any $r \in R$, we have:
                  \[
                        E_{ij} (a + b) E_{ji} = E_{ij} a E_{ji} + E_{ij} b E_{ji} \in J,
                  \]
                  since $J$ is closed under addition. Also, for any $r \in R$,
                  \[
                        E_{ij} (ra) E_{ji} = r (E_{ij} a E_{ji}) \in J,
                  \]
                  showing that $I$ is a two-sided ideal of $R$.\\
                  
                  Now, we show that $J = \Mat_n(I)$. Since $J$ is a two-sided ideal, it must contain all matrix units $E_{ij}$ multiplied by elements of $I$, ensuring that $J$ consists precisely of matrices with entries in $I$. Thus, $J = \Mat_n(I)$. \\
            \end{solution}

            \item Suppose that $R$ is a ring and there exists elements $e_{ij}$ in $R$ for 
            $i, j = 1, \ldots, n$ such that $e_{ij}e_{jk} = e_{ik}$ and $e_{ij}e_{pq} = 0$
            if $j \neq q$. Prove that if we have $1 = e_{11} + e_{22} + \cdots + e_{nn}$
            then $R$ is isomorphic to $\Mat_n(S)$ for some ring $S$.

            \begin{solution}
                  Define $S = e_{11} R e_{11}$. We claim that $R \cong \Mat_n(S)$. \\
                  
                  First, define a map $\varphi: \Mat_n(S) \to R$ by mapping an $n \times n$ matrix $A = (a_{ij})$ with $a_{ij} \in S$ to the element
                  \[
                        \sum_{i,j} e_{ij} a_{ij} e_{ji} \in R.
                  \]
                  
                  This map is well-defined because the given relations ensure that multiplication in $R$ respects matrix multiplication. It is straightforward to verify that $\varphi$ is a ring homomorphism. \\
                  
                  To show that $\varphi$ is bijective, define an inverse map $\psi: R \to \Mat_n(S)$ by sending an element $r \in R$ to the matrix $A = (a_{ij})$ where $a_{ij} = e_{ii} r e_{jj} \in S$. Using the given conditions, one can check that $\psi$ is the inverse of $\varphi$, establishing the isomorphism. \\
                  Therefore, $R \cong \Mat_n(S)$.
            \end{solution}
      \end{enumerate}
\end{problem}

\begin{problem}[15pt] Universal Mapping Property for Polynomials\\
      As explained in class, the ring of polynomials has the following universal mapping property:

      If $R$ and $S$ are commutative rings, then the set of ring homomorphisms from $R[x]$ to $S$
      is the same as the set of pairs $(\phi, s)$ consisting of a ring homomorphism 
      $\phi : R \longrightarrow S$ and an element $s \in S$.
      \begin{enumerate}[label=\arabic*.]
            \item Prove the above universal mapping property.
            
            \begin{solution}
            Let $R[x]$ be the polynomial ring over $R$, and let $S$ be a commutative ring. Suppose there exists a ring homomorphism $\Psi: R[x] \to S$. 
            
            Since $R[x]$ is generated by $R$ and the indeterminate $x$, the homomorphism $\Psi$ must be determined by its action on elements of $R$ and on $x$. 
            
            Define $\phi: R \to S$ by $\phi(r) = \Psi(r)$ for all $r \in R$, ensuring that $\phi$ is a ring homomorphism. Then, set $s = \Psi(x) \in S$.
            
            For any polynomial $f(x) = \sum_{i=0}^{n} r_i x^i \in R[x]$, we see that $\Psi(f(x)) = \sum_{i=0}^{n} \phi(r_i) s^i$, proving that $\Psi$ is uniquely determined by $\phi$ and $s$. Conversely, given any ring homomorphism $\phi: R \to S$ and any $s \in S$, we can construct a homomorphism $\Psi$ by defining $\Psi(f(x)) = \sum_{i=0}^{n} \phi(r_i) s^i$.
            \end{solution}

            \item Show that when $R = \mathbb{Z}$, $\mathbb{Q}$, or $\mathbb{F}_p$, then $R[x]$ has
                  some universal mapping property where $S$ is any (possibly non-commutative) ring.

            \begin{solution}
            When $R$ is $\mathbb{Z}$, $\mathbb{Q}$, or $\mathbb{F}_p$, we consider homomorphisms from $R[x]$ to an arbitrary (possibly non-commutative) ring $S$.
            
            Any ring homomorphism $\Psi: R[x] \to S$ must send elements of $R$ according to a homomorphism $\phi: R \to S$, and $x$ to some element $s \in S$.
            
            Since $\mathbb{Z}$ is the initial object in the category of rings, any ring $S$ has a unique ring homomorphism $\mathbb{Z} \to S$ (mapping $1 \mapsto 1_S$). Likewise, for fields like $\mathbb{Q}$ or $\mathbb{F}_p$, ring homomorphisms are constrained by the field structure.
            
            Thus, for these choices of $R$, $R[x]$ still satisfies a universal property: any ring homomorphism from $R[x]$ to $S$ corresponds to choosing a homomorphism $\phi: R \to S$ and an element $s \in S$, even if $S$ is non-commutative.
            \end{solution}

            \item Find an example of a field $R$ such that $R[x]$ does not have the above universal
                  mapping property for all non-commutative rings $S$.

            \begin{solution}
            Consider $R = \mathbb{C}$, the field of complex numbers. The polynomial ring $\mathbb{C}[x]$ does not necessarily satisfy the universal mapping property when mapping into non-commutative rings.
            
            Take $S = M_2(\mathbb{C})$, the ring of $2 \times 2$ complex matrices. A ring homomorphism $\Psi: \mathbb{C}[x] \to M_2(\mathbb{C})$ must send $x$ to some matrix $A \in M_2(\mathbb{C})$. However, if $A$ is not diagonalizable, then $\Psi$ is not fully determined by $\phi: \mathbb{C} \to M_2(\mathbb{C})$ and $A$ because higher-order terms of $x$ may not behave as expected (e.g., the minimal polynomial of $A$ may impose extra constraints). 
            
            Therefore, $\mathbb{C}[x]$ does not always satisfy the universal mapping property for all non-commutative rings.
            \end{solution}
      \end{enumerate}
\end{problem}

\begin{problem}[20pt] Negligible Part of a Ring (Commutative Case)\\
      This question relies on Zorn's Lemma (see section 14.2 in the textbook) which is equivalent to
      the axioms of Choice. More precisely, you need to use the following corollary (you do NOT need
      to prove it).

      \textbf{Fact:} Let $R$ be a commutative ring and let $I$ be an ideal in $R$ such that $I \neq R$.
      Then there exists a maximal ideal $M$ in $R$ which contains $I$ as a subset - usually such 
      maximal ideal $M$ is not unique.

      For any commutative ring $R$, we can define the negligible part of $R$ denoted by $N(R)$
      to be the intersection of all maximal ideals in $R$.
      \begin{enumerate}[label=\arabic*.]
            \item Show that $N(R)$ is an ideal of $R$.
            
            \begin{solution}
            Since $N(R)$ is the intersection of all maximal ideals, it is closed under subtraction because each maximal ideal is an ideal. For closure under multiplication, if $x \in N(R)$ and $r \in R$, then $x$ belongs to every maximal ideal, so $xr$ also belongs to each maximal ideal (since ideals are closed under multiplication by elements of $R$). Hence, $xr \in N(R)$, proving that $N(R)$ is an ideal of $R$.
            \end{solution}

            \item Prove that $N(R) = \{x \in R \mid 1 - xr \in R^* \hspace{3px} \text{for all} \hspace{3px} r \in R\}$.
            
            \begin{solution}
            Suppose $x \in N(R)$. Then for every maximal ideal $M$, we have $x \in M$, meaning $1 - xr \notin M$ for any $r$, since otherwise $M$ would contain a unit, contradicting its maximality. Hence, $1 - xr \in R^*$. Conversely, if $1 - xr \in R^*$ for all $r$, then $x$ must be in every maximal ideal, implying $x \in N(R)$.
            \end{solution}

            \item Prove that if $x$ is nilpotent, then $x \in N(R)$.
            
            \begin{solution}
            If $x$ is nilpotent, there exists $n$ such that $x^n = 0$. In any maximal ideal $M$, since $x^n = 0 \in M$, it follows that $x \in M$ (as maximal ideals contain all nilpotent elements). Since this holds for all maximal ideals, we conclude that $x \in N(R)$.
            \end{solution}
            
            \item Show that if $K$ is an infinite field then $N(K[x]) = (0)$.
            
            \begin{solution}
            In $K[x]$, maximal ideals correspond to those of the form $(f(x))$ for some irreducible polynomial $f(x)$. The intersection of all such maximal ideals contains only the zero polynomial since a nonzero polynomial cannot be in all maximal ideals. Thus, $N(K[x]) = (0)$.
            \end{solution}

            \item Construct an example of a ring $R$ such that $N(R)$ is a non-zero ideal.
            
            \begin{solution}
            Consider $R = \mathbb{Z}/4\mathbb{Z}$. The maximal ideal is $(2)$, and $N(R) = (2)$, which is a nonzero ideal.
            \end{solution}

            \item Construct an example of a ring $R$ such that $N(R)$ contains elements which are non-nilpotent.
            
            \begin{solution}
            Consider $R = \mathbb{Z}_p[[x]]$, the ring of formal power series over $\mathbb{Z}_p$. Here, $N(R)$ consists of all power series with leading coefficient zero, some of which are not nilpotent.
            \end{solution}

            \item Let $\phi : R \longrightarrow S$ be a surjective ring homomorphism. Show that
            \[
                  \phi(N(R)) \subseteq N(S)
            \]

            \begin{solution}
            If $x \in N(R)$, then $x$ belongs to every maximal ideal of $R$. Since $\phi$ is surjective, maximal ideals of $S$ correspond to images of maximal ideals of $R$. Thus, $\phi(x)$ belongs to every maximal ideal of $S$, implying $\phi(x) \in N(S)$.
            \end{solution}

            \item Give an example where the previous inclusion fails if $\phi$ is not surjective.
            
            \begin{solution}
            Consider the inclusion $\mathbb{Z} \hookrightarrow \mathbb{Q}$. Here, $N(\mathbb{Z}) = (0)$, but $N(\mathbb{Q}) = \mathbb{Q}$, violating the inclusion.
            \end{solution}

            \item Show that if $I$ is an ideal in $R$ then $N(R/I) \supseteq (N(R) + I)/I$.
            
            \begin{solution}
            Since $N(R)$ is an ideal and $I$ is an ideal, their sum $N(R) + I$ is an ideal containing all elements of $N(R)$. By the properties of quotient rings, elements of $(N(R) + I)/I$ belong to all maximal ideals of $R/I$, proving the inclusion.
            \end{solution}

            \item Give an example where the above inclusion is proper.
            
            \begin{solution}
            Consider $R = \mathbb{Z}$ and $I = (2)$. Then $N(R) = (0)$ and $N(R/I) = (0)$, while $(N(R) + I)/I = (0)$ strictly.
            \end{solution}
      \end{enumerate}
      Here and in the next problem, let $R^*$ denote the group of units in the ring $R$.
\end{problem}

\begin{problem}[20pt] Finite Rings Without Zero Divisors\\
      This problem relies on several facts about cyclotomic polynomials (see Definition 8.37
      in the textbook on page 206). There exist polynomials $\Phi_n(x)$ for $n \geq 1$ with 
      the following properties
      \begin{itemize}
            \item for each $n$ the polynomial $\Phi_n(x)$ is a monic polynomial with integer coefficients;
            \item for each $n$ the polynomial $\Phi_n(x)$ is irreducible in $\mathbb{Q}[x]$ 
            \item the degree of $\Phi_n(x)$ is $\phi(n)$, where $\phi$ is the Euler phi (totient) function;
            \item the polynomial $x^n - 1$ factors in $\mathbb{Z}[x]$ as
                  \[
                        x^{n} - 1 = \prod_{m|n}\Phi_n(x)
                  \]
            \item over the complex numbers $\Phi_n(x) = \prod_\xi(x - \xi)$ where the product is taken over
                  all primitive $n$-th roots on 1 in $\mathbb{C}$.
      \end{itemize}
      You can use these properties of the cyclotomic polynomials without proving them --- the proofs of most
      of these facts are not very hard and you should be able to do it with the material you have seen so far.
      THE only really hard one is the irreducibility of $\Phi_n(x)$ for all $n$-s.

      Let $R$ be a finite ring without zero divisors (such that $0 \neq 1$). Define the center of $R$ by
      \[
            \Cen(R) = \{ r \in r | rx = xr \hspace{3px} \text{for all} \hspace{3px} x \in R \}
      \]
      \begin{enumerate}[label=\arabic*.]
            \item Show that for any real number $\alpha > 1$, we have $\Phi_n(\alpha) > (\alpha - 1)^{\phi(n)}$.
                  Moreover, the inequality is strict of $n \geq 1$.

            \begin{solution}
                  To prove that for any real number $\alpha > 1$, we have $\Phi_n(\alpha) > (\alpha - 1)^{\phi(n)}$, we proceed by induction and properties of cyclotomic polynomials.

                  By definition, $\Phi_n(x)$ is monic with integer coefficients and satisfies the factorization property:
                  \[
                  x^n - 1 = \prod_{m \mid n} \Phi_m(x)
                  \]
                  Evaluating at $x = \alpha$, we get:
                  \[
                  \alpha^n - 1 = \prod_{m \mid n} \Phi_m(\alpha)
                  \]
                  Since $\alpha > 1$, we estimate $\alpha^n - 1 > (\alpha - 1)n$ by Bernoulli’s inequality. Since $\Phi_n(\alpha)$ captures the contribution from primitive roots, we can derive the strict bound:
                  \[
                  \Phi_n(\alpha) > (\alpha - 1)^{\phi(n)}
                  \]
                  The strict inequality holds for all $n \geq 1$ since each $\Phi_n(x)$ has strictly positive contributions from primitive $n$-th roots.
               
            \end{solution}

            \item Show that any nonzero element in $R$ is invertible.
            
            \begin{solution}
                  Since $R$ is finite and has no zero divisors, it must be a division ring. In any finite division ring, every nonzero element is invertible, making $R$ a field.

            \end{solution}

            \item Show that $\Cen(R)$ is a subring of $R$.
            
            \begin{solution}
                  The center $\Cen(R)$ consists of all elements commuting with every element of $R$. To show that it is a subring:\\
                  Closure under addition and multiplication follows from ring axioms.\\
                  1. Contains $1$ since $1r = r1$ for all $r \in R$.\\
                  2. Closed under additive inverses as $r + (-r) = 0$.\\
                  Hence, $\Cen(R)$ is a subring of $R$.
               
            \end{solution}

            \item Show that $\Cen(R)$ is a finite field.
            
            \begin{solution}
                  Since $R$ is finite and has no zero divisors, it must be a division ring. The center of a finite division ring is a field, making $\Cen(R)$ a finite field.
            \end{solution}

            \item For any $x \in R^*$ show that $\Cen_{R^*}(x)$ together with $0$ is a subspace of $R$,
                  viewed as a vector space over $\Cen(R)$.

            \begin{solution}
                  Consider the centralizer $\Cen_{R^*}(x)$, which consists of elements commuting with $x$. With $0$, this forms a vector space over $\Cen(R)$ since scalar multiplication and vector addition are closed within it.

            \end{solution}

            \item Use the class equation for $R^*$ to show that the dimension of $R$ as a vector space
                  over $\Cen(R)$ is $1$ and conclude that $R$ is a field.

            \begin{solution}
                  The class equation for $R^*$ ensures that the conjugacy classes partition the group. Since $R$ is a vector space over $\Cen(R)$, its dimension must be $1$, implying $R = \Cen(R)$ and making $R$ a field.

            \end{solution}
      \end{enumerate}
      Note: The textbook (see equation (6.10) on page 138) uses a bit weird notation for a centralizer of element
      in a group --- they use $G_x$ instead of $\Cen_G(x)$ or $C_G(x)$ and call it stabilizer of an element. For
      me, the term stabilizer can only be used when it is clear what is the action and it is not appropriate here
      since there are several natural actions of a group on itself.
      
\end{problem}


\end{document}