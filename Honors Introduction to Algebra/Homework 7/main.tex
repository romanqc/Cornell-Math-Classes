\documentclass[12pt]{article}
\usepackage[margin=1in]{geometry}

\usepackage{libertine}
\usepackage{parskip}
\usepackage{enumitem}
\usepackage{array}
\usepackage{graphicx}
\graphicspath{ {./images/} }

\usepackage{amsthm,amsmath,amssymb}
\usepackage{tikz}
\usetikzlibrary{arrows,automata,shapes.geometric}

\usepackage{pgfplots}
\pgfplotsset{compat=1.18}

\pagestyle{plain}
\thispagestyle{empty}

\definecolor{carnellian}{RGB}{190,20,20}

\theoremstyle{definition}
\newtheorem{problem}{Problem}

\newcounter{subq}[problem]
\newenvironment{subproblem}
{\refstepcounter{subq} \begin{itemize} \item[(\alph{subq})]}
{\end{itemize} \medskip}
\DeclareMathOperator{\Ima}{Im}
\DeclareMathOperator{\rank}{rank}
\DeclareMathOperator{\tr}{tr}
\DeclareMathOperator{\Hom}{Hom}
\DeclareMathOperator{\End}{End}
\DeclareMathOperator{\Span}{Span}
\DeclareMathOperator{\Aut}{Aut}
\DeclareMathOperator{\Ann}{Ann}
\DeclareMathOperator{\Mat}{Mat}
\DeclareMathOperator{\GL}{GL}
%\DeclareMathOperator{\deg}{deg}



\usepackage{environ}
\NewEnviron{solution}[1][\vfill]{
    \textcolor{blue}{\BODY}
}

\newcommand{\hwnum}{7}
\newcommand{\duedate}{3/25/2025}
\renewcommand{\title}{Groups, Structures of Abelian Groups, Automorphisms}

\begin{document}

\hspace{-10px}
\begin{tabular*}{\textwidth}{l @{\extracolsep{\fill}} r}
    \textbf{Honors Algebra} & \textbf{Spring 2025} \\
    \textbf{HW \hwnum : \title} &  \textbf{\duedate} \\
\end{tabular*}

\vspace{1cm}

\textit{Abstract Algebra: An Integrated Approach by J.H. Silverman.}\\
Page 393-396: 12.10, 12.11, 12.12, 12.14, 12.15, 12.20, 12.26


\vspace{1cm}

%-----------TEMPLATE------------%
% \begin{problem}
%     \begin{enumerate}[label=(\alph*)]
%         \item 
%         \begin{solution}

%         \end{solution}

%         \item 
%         \begin{solution}

%         \end{solution}
%     \end{enumerate}
% \end{problem}

\begin{problem}[12.10]
    For each $pi \in \mathcal{S}_n$, we define a linear transformation $\rho(\pi): \mathbb{R}^n \longrightarrow \mathbb{R}^n$ by
    \[
        \rho(\pi)(e_i) = e_{\pi(i)}
    \]
    By abuse of notation, we write $\rho(\pi)$ for the $n$-by-$n$ matrix of $\rho(\pi)$ relative to the basis $\{e_1, \ldots, e_n\}$.
    The matrix $\rho(\pi)$, whose entries are all equal to $0$ or $1$, is called the permutation matrix associated to $\pi$.
    \begin{enumerate}[label=(\alph*)]
        \item Prove that the matrix $\rho(\pi)$ can also be described by the following formula:
        \[ 
            (i, j)\text{-entry of } \rho(\pi) = 
            \begin{cases} 
                1 & \text{if } \pi(j) = i,   \\
                0 & \text{if } \pi(j) \neq i
            \end{cases}
        \]
        \begin{solution}

        \end{solution}

        \item Write down the six 3-by-3 matrices corresponding to the six elements of $\mathcal{S}_3$.
        \begin{solution}

        \end{solution}

        \item Prove that every row of $\rho(\pi)$ has exactly one entry equal to 1 and similarly that every column of $\rho(\pi)$
              has exactly one entry equal to 1.
        \begin{solution}

        \end{solution}

        \item Prove that the map
        \[
            \rho : \mathcal{S}_n \longrightarrow \GL_n{\mathbb{Z}}
        \]
        is an injective group homomorphism; i.e., prove that $\rho$ is injective and satisfies
        \[
            \rho(\pi_1 \pi_2) = \rho(\pi_i)\rho(\pi_2) \hspace{4px} \text{for all } \pi_1, \pi_2 \in \mathcal{S}_n
        \]
        \begin{solution}

        \end{solution}

        \item Let $\pi \in \mathcal{S}_n$. Prove that
              \[
                \text{sign}(\pi) = \det{\rho(\pi)}
              \]
        \begin{solution}

        \end{solution}

        \item Prove that the eigenvalues of $\rho(\pi)$ are roots of unity.
        \begin{solution}

        \end{solution}
    \end{enumerate}
\end{problem}

\begin{problem}[12.11]
    Let $G$ be a group, let $H \subset G$ be a subgroup of $G$, and let $G/H$ be the collection of cosets of $H$.
    \begin{enumerate}[label=(\alph*)]
        \item Prove that there is a well-defined group homomorphism
        \[
            \pi_H : G \longrightarrow \mathcal{S}_{G/H}, \hspace{5px} \pi_H(g)(a H) = gaH
        \]
        (\textit{Hint.} Generalize the proof of Cayley's Theorem, which was the case $H = \{e\}$.)
        \begin{solution}

        \end{solution}

        \item Prove that the kernel of $\pi_H$ is contained in $H$.
        \begin{solution}

        \end{solution}

        \item Let $K$ be a normal subgroup of $G$, and suppose that $K$ is contained in $H$. 
              Prove that $K \subseteq \ker(\pi_H)$. Thus we may describe $\ker(\pi_H)$ as the
              largest normal subgroup of $G$ that is contained in $H$.
        \begin{solution}

        \end{solution}
    \end{enumerate}
\end{problem}

\begin{problem}[12.12]
    Let $G$ be a group of order $n$, and let $H \not\subseteq G$ be a subgroup of order $m$, and suppose
    that $n$ does nto divide $(n/m)!$. Prove that $H$ contains a non-trivial normal subgroup of $G$. 
    (\textit{Hint.} Use Exercise 12.11.)
    \begin{solution}

    \end{solution}
\end{problem}

\begin{problem}[12.14]
    Let $G$ be a group. We say that a normal subgroup $N \not\subseteq G$ is a maximal normal subgroup of $G$
    if it has the following property:
    \[
        N \subseteq N' \subseteq G \hspace{5px} \text{and} \hspace{5px} N' \text{ normal in } G \hspace{5px} \implies \hspace{5px} N' = N \hspace{5px} \text{or} \hspace{5px} N' = G
    \]
    Prove that
    \[
        N \text{ is a maximal normal subgroup} \hspace{5px} \iff \hspace{5px} G/N \text{is a simple group}
    \]
    \begin{solution}

    \end{solution}
\end{problem}

\begin{problem}[12.15]
    For this exercise, you may use the first part of Sylow's Theorem that says that $p$-Sylow subgroups exist
    but do not use the second and third parts.
    \begin{enumerate}[label=(\alph*)]
        \item Let $p$ and $q$ be distinct primes, and let $G$ be a group of order $pq$. Prove that $G$ is not
              a simple group.
        \begin{solution}

        \end{solution}

        \item Prove that a group of order 36 is not a simple group.
        \begin{solution}

        \end{solution}

        \item Let $p$ and $q$ be distinct primes, let $i, j \geq 1$, and suppose that $p^i < q$.
              Prove that a group of order $p^i < q^j$ is not a simple group. 
        \begin{solution}

        \end{solution}
    \end{enumerate}
    (\textit{Hint.} Use Exercise 12.12. We mention that we proved (a) using all three 
    parts of Sylow's Theorem in Example 6.37.)
\end{problem}

\begin{problem}[12.20] \phantom{.}
    \begin{enumerate}[label=(\alph*)]
        \item Prove that $\Aut(\mathbb{Z})$ is a cyclic group of order 2.
        \begin{solution}

        \end{solution}

        \item More generally, prove that $\Aut(\mathbb{Z^n}) \cong \GL_n(\mathbb{Z})$.
        \begin{solution}

        \end{solution}
    \end{enumerate}
\end{problem}

\begin{problem}[12.26]
    Let $G$ be the set of affine linear maps on $\mathbb{R}$, which by definition is the group
    \[
        G = \{ \text{maps } \phi : \mathbb{R} \longrightarrow \mathbb{R} \text{ of the form } \phi(x) = ax + b \text{ for some } a \in \mathbb{R}^* \text{ and } b \in \mathbb{R} \}
    \]
    \begin{enumerate}[label=(\alph*)]
        \item Prove that $G$ is group, where the group law is composition of maps.
        \begin{solution}

        \end{solution}

        \item Prove that $G$ is isomorphic to the semidirect product $\mathbb{R} \rtimes \mathbb{R}^*$.
              Be sure to describe the homomorphism from $\mathbb{R}^*$ to $\Aut(\mathbb{R})$ used to define
              the semidirect product.
        \begin{solution}

        \end{solution}
    \end{enumerate}
\end{problem}


\end{document}