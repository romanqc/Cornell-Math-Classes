\documentclass[12pt]{article}
\usepackage[margin=1in]{geometry}

\usepackage{libertine}
\usepackage{parskip}
\usepackage{enumitem}
\usepackage{array}
\usepackage{graphicx}
\graphicspath{ {./images/} }

\usepackage{amsthm,amsmath,amssymb}
\usepackage{tikz}
\usetikzlibrary{arrows,automata,shapes.geometric}

\usepackage{pgfplots}
\pgfplotsset{compat=1.18}

\pagestyle{plain}
\thispagestyle{empty}

\definecolor{carnellian}{RGB}{190,20,20}

\theoremstyle{definition}
\newtheorem{problem}{Problem}

\newcounter{subq}[problem]
\newenvironment{subproblem}
{\refstepcounter{subq} \begin{itemize} \item[(\alph{subq})]}
{\end{itemize} \medskip}
\DeclareMathOperator{\Ima}{Im}
\DeclareMathOperator{\rank}{rank}
\DeclareMathOperator{\tr}{tr}
\DeclareMathOperator{\Hom}{Hom}
\DeclareMathOperator{\End}{End}
\DeclareMathOperator{\Span}{Span}
\DeclareMathOperator{\Aut}{Aut}
\DeclareMathOperator{\Ann}{Ann}
\DeclareMathOperator{\Mat}{Mat}
\DeclareMathOperator{\modd}{mod}
%\DeclareMathOperator{\deg}{deg}



\usepackage{environ}
\NewEnviron{solution}[1][\vfill]{
    \textcolor{blue}{\BODY}
}

\newcommand{\hwnum}{10}
\newcommand{\duedate}{4/22/2025}
\renewcommand{\title}{Galois Theory, Automorphisms of Fields, Splitting Fields}

\begin{document}

\hspace{-10px}
\begin{tabular*}{\textwidth}{l @{\extracolsep{\fill}} r}
    \textbf{Honors Algebra} & \textbf{Spring 2025} \\
    \textbf{HW \hwnum : \title} &  \textbf{\duedate} \\
\end{tabular*}

\vspace{1cm}

\textit{Abstract Algebra: An Integrated Approach by J.H. Silverman.}\\
Page 285-294: 9.22, 9.23, 9.25, 9.26, 9.29, 9.32, 9.32, 9.39, 9.40, 9.41, 9.44, 9.48, 9.53


\vspace{1cm}

%-----------TEMPLATE------------%
% \begin{problem}
%     \begin{enumerate}[label=(\alph*)]
%         \item 
%         \begin{solution}

%         \end{solution}

%         \item 
%         \begin{solution}

%         \end{solution}
%     \end{enumerate}
% \end{problem}

\begin{problem}[9.22]
    Let $F$ be a field, let $f(x) \in F[x]$ be a seperable polynomial of degree $n \geq 1$,
    and let $K/F$ be a splitting field for $f(x)$ over $F$. Prove the following implications:
    \[
        \#G(K/F) = n! \hspace{5px} \iff \hspace{5px} G(K/F) \cong \mathcal{S}_n \hspace{5px} \implies \hspace{5px} f(x) \text{ is irreducible in } F[x]
    \]
    Note that the first implication is an ``if and only if,'' but the second only goes in one direction. 
    Show that the second implication cannot be reversed, by writing down an example for which it is false.
    \begin{solution}

    \end{solution}
\end{problem}

\begin{problem}[9.23]
    Let $K/F$ be a Galois extension, let $\alpha \in K$, and let
    \[
        f(X) = \prod_{\sigma \in G(K/F)} (X - \sigma(\alpha))
    \]
    \begin{enumerate}[label=(\alph*)]
        \item Prove that $f(x) \in F[x]$; i.e., prove that the coefficients of $f(x)$ are in $F$.
        \begin{solution}

        \end{solution}

        \item Let $\Phi_{\alpha, F} \in F[x]$ be the minimal polynomial of $f(X)$ over $F$. Prove that
              \[ 
                f(X) = \Phi_{\alpha, F}(X) \hspace{5px} \iff \hspace{5px} K = F(\alpha) 
              \]
        \begin{solution}

        \end{solution}

        \item In general, even if $K \neq F(\alpha)$, prove that there is an integer $d \geq 1$ such that
              \[
                f(X) = \Phi_{\alpha, F}(X)^d
              \]
              (\textit{Hint.} Use the $G(K/F)$-orbit of $\alpha$ to split the factors of $f(X)$ into subsets,
              and relate them to the $G(K/F)$-stabilizer subgroup of $\alpha$.)
        \begin{solution}

        \end{solution}
    \end{enumerate}
\end{problem}

\begin{problem}[9.25]
    \phantom{.}
    \begin{enumerate}[label=(\alph*)]
        \item Let $f(x) = x^4 - 6x^2 + 2$, and let $K$ be the splitting field of $f(x)$ over $\mathbb{Q}$.
              \begin{enumerate}[label=(\arabic*)]
                  \item Compute the Galois group $G(K/\mathbb{Q})$.
                  \begin{solution}

                  \end{solution}

                  \item Make a list of the subgroups $H$ of $G(K/\mathbb{Q})$ and the corresponding intermediate fields $K^H$; cf. Figure 29 on page 249.
                  \begin{solution}

                  \end{solution}

                  \item Make a field diagram and a group diagram illustrating the Galois correspondence that you found in (2); cf Figure 30 on page 251.
                  \begin{solution}

                  \end{solution}
              \end{enumerate}

        \item Same as (a) for the polynomial $f(x) = x^4 - 10x^2 + 20$.
        \begin{solution}

        \end{solution}

        \item Same as (a) for the polynomial $f(x) = x^4 - 5x^2 + 6$
        \begin{solution}

        \end{solution}
    \end{enumerate}
\end{problem}

\begin{problem}[9.26]
    Let $K/F$ be a Galois extension.
    \begin{enumerate}[label=(\alph*)]
        \item Let $\sigma \in G(K/F)$, and let $E$ be an intermediate field of $K/F$. Prove that
              \[ 
                 G(K/\sigma(E)) = \sigma G(K/E)\sigma^{-1}
              \]
        \begin{solution}

        \end{solution}

        \item Let $\sigma \in G(K/F)$, and let $H \subseteq G(K/F)$ be a subgroup. Prove that
              \[
                    \sigma(K^H) = K^{\sigma H \sigma^-1}
              \]
        \begin{solution}

        \end{solution}

        \item We say that intermediate fields $E_1$ and $E_2$ are conjugate subfields of $K/F$ if
              \[
                    \sigma(E_1) = E_2 \quad \text{for some} \quad \sigma \in G(K/F)
              \]
              Prove that $E_1$ and $E_2$ are conjugate subfields of $K/F$ if and only if the groups $G(K/E_1)$ and $G(K/E_2)$ are conjugate subgroups of $G(K/F)$.
        \begin{solution}

        \end{solution}

        \item Let $H_1, H_2 \subseteq G(K/F)$ be subgroups. Prove that their fixed fields $K^{H_1}$ and $K^{H_2}$
              are conjugate subfields of $K/F$ if and only if $H_1$ and $H_2$ are conjugate subgroups of $G(K/F)$.
    \end{enumerate}
\end{problem}

\begin{problem}[9.29]
    Let $f(X) = X^5 - iX^2 + 1 - i \in \mathbb{C}[X]$. Find a polynomial $g(X) \in \mathbb{R}[X]$
    of degree 10 with the property that every complex root of $f(X)$ is also a root of $g(X)$.
    \begin{solution}

    \end{solution}
\end{problem}

\begin{problem}[9.32]
    Let $q$ be a prime power, let $N \in \mathbb{Z}$ be a prime satisfying $\gcd(q, N) = 1$, and let $\zeta$ be a primitive $N$th-root of unity in some extension field of $\mathbb{F}_q$.
    \begin{enumerate}[label=(\alph*)]
        \item Prove that $\mathbb{F}_q(\zeta)/\mathbb{F}_q$ is a seperable extension.
        \begin{solution}

        \end{solution}

        \item Let $e$ be the order of $q$ in the group $(\mathbb{Z}/N\mathbb{Z})^*$; i.e.,
              \[
                  q^e \equiv 1 (\modd{N}) \quad \text{and} \quad q^i \not\equiv 1 (\modd{N}) \quad \text{for } 1 \leq i < e
              \]
              Prove that
              \[
                    [\mathbb{F}_q(\zeta) : \mathbb{F}_q] = e
              \]
        \begin{solution}

        \end{solution}

        \item Prove that the polynomial
              \[
                  X^{N-1} + X^{N-2} + \cdots + X + 1
              \]
            is irreducible in $\mathbb{F}_q[X]$ if and only if $q$ generates the multiplicative group $(\mathbb{Z}/N\mathbb{Z})^*$.
        \begin{solution}

        \end{solution}
    \end{enumerate}
\end{problem}

\begin{problem}[9.39]
    Let $K/F$ be a Galois extension of fields. For subgroups $H_1, H_2 \subset G(K/F)$, we define
    \[
        H_1 \star H_2 = \bigcap_{\substack{\text{subgroups } H \subset G(K/F)\\\text{with } H_1, H_2 \subseteq H}} H 
    \]
    to be the smallest subgroup of $G(K/F)$ that contains both $H_1$ and $H_2$. Similarly, for intermediate fields $E_1, E_2$ of $K/F$, we define
    \[
        E_1 \star E_2 = \bigcap_{\substack{\text{subgroups } E \subseteq K\\\text{with } E_1, E_2 \subseteq E}} E
    \]
    to be the smallest intermediate field containing both $E_1$ and $E_2$. Prove that
    \[
        K^{H_1 \star H_2} = K^{H_1} \cap K^{H_2} \quad \text{and} \quad K^{H_1 \cap H_2} = K^{H_1} \star K^{H_2}
    \]
    \begin{solution}

    \end{solution}
\end{problem}

\begin{problem}[9.40]
    Let $K/F$ be a Galois extension, and let $E_1$ and $E_2$ be intermediate fields such that $E_1/F$ and $E_2/F$
    are Galois extensions. Let $E_1 \star E_2$ be the compositum of $E_1$ and $E_2$ as defined in Exercise 9.39.
    \begin{enumerate}[label=(\alph*)]
        \item Prove that $E_1 \star E_2$ is a Galois extension of $F$.
        \begin{solution}

        \end{solution}

        \item What is the kernel of the homomorphism
              \[
                  G(K/F) \longrightarrow G(E_1/F) \times G(E_2/F)
              \]
              Describe the kernel explicitly as a subgroup of $G(K/F)$.
        \begin{solution}

        \end{solution}

        \item Prove that
              \[
                    \gcd([E_1 : F], [E_2 : F]) = 1 \quad \implies \quad \text{the map (9.44) in (b) is surjective.}
              \]
        \begin{solution}

        \end{solution}
    \end{enumerate}
\end{problem}

\begin{problem}[9.41]
    Let $K/F$ be a finite extension of fields. For each $\alpha \in K$, we define a multiplication by $\alpha$ map by
    \[
        \mu_\alpha : K \longrightarrow K, \quad \mu_\alpha(\beta) = \alpha\beta
    \]
    \begin{enumerate}[label=(\alph*)]
        \item If we view $K$ as an $F$-vector space, prove that $\mu_\alpha$ is an $F$-linear transformation of $K$ to itself.
        \begin{solution}

        \end{solution}

        \item We recall that any $F$-linear operator on a finite-dimensional $F$-vector space has a well-defined
              trace and determinant; see Section 10.6 and Exercise 10.23. For $\alpha \in K$, we define the $K/F$-trace
              and the $K/F$-norm of $\alpha$ by
              \[
                  \tr_{K/F}(\alpha) = \tr(\mu_\alpha) \quad \text{and} \quad N_{K/F}(\alpha) = \det(\mu_\alpha)
              \]
              Prove that for all $\alpha_1, \alpha_2 \in K$ we have
              \[
                    \tr_{K/F}(\alpha_1 + \alpha_2) = \tr_{K/F}(\alpha_1) + \tr_{K/F}(\alpha_2)
              \]
              \[
                    N_{K/F}(\alpha_1 \cdot \alpha_2) = N_{K/F}(\alpha_1) \cdot N_{K/F}(\alpha_2)
              \]
        \begin{solution}

        \end{solution}

        \item For each of the following $K/F$ and $\alpha \in K$, compute $\tr_{K/F}(\alpha)$ and $N_{K/F}(\alpha)$
              directly from the definition (9.45) by choosing a basis for $K/F$ and writing down the matrix associated to $\mu_\alpha$:
              \begin{enumerate}[label=(\arabic*)]
                \item $F = \mathbb{R}, \quad K = \mathbb{C},              \hspace{38px} \alpha = a + bi$ with $a,b \in \mathbb{R}$
                \item $F = \mathbb{Q}, \quad K = \mathbb{Q}(\sqrt{3}),    \quad \alpha = a + b\sqrt{3}$ with $a,b \in \mathbb{Q}$ 
                \item $F = \mathbb{Q}, \quad K = \mathbb{Q}(\sqrt[3]{2}), \quad \alpha = 2 - 3\sqrt[3]{2} + \sqrt[3]{4}$ 
              \end{enumerate}
        \begin{solution}

        \end{solution}
        
        \item Let $K/F$ be a Galois extension, and let $\alpha \in K$. Prove that
        \[
            \tr_{K/F}(\alpha) = \sum_{\sigma \in G(K/F)} \sigma(\alpha) \quad \text{and} \quad N_{K/F}(\alpha) = \prod_{\sigma \in G(K/F)} \sigma(\alpha)
        \]
        (\textit{Hint.} First do the case that $\#\{ \sigma(\alpha) : \sigma \in G(K/F) \} = [K : F]$ and use the Cayley-Hamilton
        theorem from linear algebra. See Exercise 11.47 on page 368 for a statement of the Cayley-Hamilton theorem.)
        \begin{solution}
            
        \end{solution}
    \end{enumerate}
\end{problem}

\begin{problem}[9.44]
    For this problem, we write $\zeta_n$ for a primite $n$th-root of unity. Kronecker's Theorem (Theorem 9.66) implies in particular
    that every square root lies in a cyclotomic extension of $\mathbb{Q}$. This problem asks you to verify this special case.
    \begin{enumerate}[label=(\alph*)]
        \item Prove that $\sqrt{2} \in \mathbb{Q}(\zeta_8)$. 
        \begin{solution}

        \end{solution}

        \item Prove that $\sqrt{3} \in \mathbb{Q}(\zeta_12)$
        \begin{solution}

        \end{solution}

        \item More generally, prove that if $p$ is an odd prime, then $\sqrt{p} \in \mathbb{Q}(\zeta_{4p})$.
        (\textit{Hint.} We haven't really developed the tools yet to prove (c), but it's fun to think about.
        One way to prove is to show that $p = \Phi_{\zeta_{p, \mathbb{Q}}}(1)$ is (almost) a square in $\mathbb{Q}(\zeta_{4p})$
        by using the factorization of $\Phi_{\zeta_{p, \mathbb{Q}}}(x)$ into linear factors in $\mathbb{Q}(\zeta_p)[x]$.)
        \begin{solution}

        \end{solution}
    \end{enumerate}
\end{problem}

\begin{problem}[9.48]
    Consider the cyclotomic polynomial
    \[
        \Phi_17(x) = \frac{x^17 - 1}{x - 1} = x^16 + x^15 + \cdots + x + 1
    \]
    which we know is irreducible from Example 8.36.
    \begin{enumerate}[label=(\alph*)]
        \item Prove that $f(x)$ is solvable in radicals over $\mathbb{Q}$.
        \begin{solution}

        \end{solution}

        \item Let $K$ be the splitting field over $\mathbb{Q}$ of $\Phi_17(x)$. Prove that there is a sequence of fields
              \[
                  \mathbb{Q} = E_0 \subset E_1 \subset E_2 \subset E_3 \subset E_4 = K
              \]
              with each $[E_{i+1} : E_i] = 2$. Conclude that the roots of $f(x)$ require taking only square roots.
        \begin{solution}

        \end{solution}
        
        \item Recall that we proved that a number is constructible if and only if it lives in a field obtained by taking successive square roots.
              use (b) to prove that it is possible to draw a regular 17-gon using ruler and compass.
        \begin{solution}

        \end{solution}
    \end{enumerate}
\end{problem}

\begin{problem}[9.53]
    Let $K$ be a field, let $\sigma_1 : K \longrightarrow K$ and $\sigma_2 : K \longrightarrow K$ be field automorphisms,
    let $c_1, c_2 \in K$, and define a function
    \[
        \phi : K \longrightarrow K, \quad \phi(\alpha) = c_1\sigma_1(\alpha) + c_2\sigma_2(\alpha)
    \]
    \begin{enumerate}[label=(\alph*)]
        \item Give an example of a field $K$, distinct field automorphisms $\sigma_1, \sigma_2$ of $K$, and non-zero field
              elements $c_1, c_2 \in K$, so that the map $\phi = c_1\sigma_1 + c_2\sigma_2$ is neither injective nor sujective.
        \begin{solution}

        \end{solution}

        \item Let $\alpha, \beta \in K$. Expand and simplify the difference
              \[
                    \phi(\alpha) \cdot \phi(\beta) - \phi(\alpha \cdot \beta)
              \]
              to find an expression that makes it clear that the difference is unlikely to be 0.
        \begin{solution}

        \end{solution}
    \end{enumerate}
\end{problem}

\end{document}