\documentclass[12pt]{article}
\usepackage[margin=1in]{geometry}

\usepackage{libertine}
\usepackage{parskip}
\usepackage{enumitem}
\usepackage{array}
\usepackage{graphicx}
\graphicspath{ {./images/} }

\usepackage{amsthm,amsmath,amssymb}
\usepackage{tikz}
\usetikzlibrary{arrows,automata,shapes.geometric}

\usepackage{pgfplots}
\pgfplotsset{compat=1.18}

\pagestyle{plain}
\thispagestyle{empty}

\definecolor{carnellian}{RGB}{190,20,20}

\theoremstyle{definition}
\newtheorem{problem}{Problem}

\newcounter{subq}[problem]
\newenvironment{subproblem}
{\refstepcounter{subq} \begin{itemize} \item[(\alph{subq})]}
{\end{itemize} \medskip}
\DeclareMathOperator{\Ima}{Im}
\DeclareMathOperator{\rank}{rank}
\DeclareMathOperator{\tr}{tr}
\DeclareMathOperator{\Hom}{Hom}
\DeclareMathOperator{\End}{End}
\DeclareMathOperator{\Span}{Span}
\DeclareMathOperator{\Aut}{Aut}
%\DeclareMathOperator{\deg}{deg}



\usepackage{environ}
\NewEnviron{solution}[1][\vfill]{
    \textcolor{blue}{\BODY}
}

\newcommand{\hwnum}{5}
\newcommand{\duedate}{2/25/2025}
\renewcommand{\title}{Fields}

\begin{document}

\hspace{-10px}
\begin{tabular*}{\textwidth}{l @{\extracolsep{\fill}} r}
    \textbf{Honors Algebra} & \textbf{Spring 2025} \\
    \textbf{HW \hwnum : \title} &  \textbf{\duedate} \\
\end{tabular*}

\vspace{1cm}

\textit{Abstract Algebra: An Integrated Approach by J.H. Silverman.}\\
Page 151-155: 6.1, 6.7, 6.8, 6.12, 6.16, 6.21, 6.22, 6.26, 6.30, 6.31

\vspace{1cm}

%-----------TEMPLATE------------%
% \begin{problem}
%     \begin{enumerate}[label=(\alph*)]
%         \item 
%         \begin{solution}

%         \end{solution}

%         \item 
%         \begin{solution}

%         \end{solution}
%     \end{enumerate}
% \end{problem}

\begin{problem}[6.1]
    Let $\psi : G \longrightarrow G'$ be a homomorphism of groups.
    \begin{enumerate}[label=(\alph*)]
        \item Prove that the image $\psi(G) = \{ \psi(g) : g \in G \}$ is a subgroup of $G'$.
        
        \begin{solution}
            Let's verify the subgroup criteria: Closure, Identity, and Inverses.
            
            Let $a, b \in \psi(G)$. Then there exist $g_1, g_2 \in G$ such that $\psi(g_1) = a$ and $\psi(g_2) = b$. Since $G$ is closed under multiplication, $g_1 g_2 \in G$, and thus
               \[
               \psi(g_1 g_2) = \psi(g_1) \psi(g_2) = ab \in \psi(G).
               \]
            
            Since $G$ has an identity element $e_G$, applying $\psi$ gives $\psi(e_G)$, which is the identity in $G'$. Thus, $\psi(G)$ contains the identity of $G'$.
            
            Let $a \in \psi(G)$. Then $a = \psi(g)$ for some $g \in G$. Since $G$ contains inverses, $g^{-1} \in G$, and applying $\psi$, we obtain $\psi(g^{-1}) = \psi(g)^{-1} = a^{-1} \in \psi(G)$.
            
            Since all subgroup criteria are satisfied, $\psi(G)$ is a subgroup of $G'$.
        \end{solution}

        \item Suppose that $G$ is a finite group. Prove that
              \[
                  \#G = \#\psi(G) \cdot \#\ker(\psi)
              \]

        \begin{solution}
            Consider the kernel of $\psi$, $\ker(\psi) = \{ g \in G \mid \psi(g) = e' \}$, which is a normal subgroup of $G$. By the First Isomorphism Theorem, $G/\ker(\psi) \cong \psi(G)$, so $\#G/\ker(\psi) = \#\psi(G)$.
            
            Since each coset of $\ker(\psi)$ has $\#\ker(\psi)$ elements, the number of cosets is $\#G / \#\ker(\psi)$, which equals $\#\psi(G)$. Rearranging gives the desired result:
            \[
            \#G = \#\psi(G) \cdot \#\ker(\psi).
            \]
        \end{solution}
    \end{enumerate}
\end{problem}

\begin{problem}[6.17]
    Let $G$ be a group, let $K \subseteq H \subseteq G$ be subgroups, and assume that $K$ is a normal subgroup of $G$.
    \begin{enumerate}[label=(\alph*)]
        \item Prove that $H/K$ is naturally a subgroup of $G/K$; more precisely, show that there is a natural injective homomorphism $H/K \hookrightarrow G/K$.
        
        \begin{solution}
            Define $\phi: H/K \to G/K$ by $\phi(hK) = hK$ for all $h \in H$. This function is well-defined, as coset multiplication in $G/K$ respects multiplication in $G$. It is a homomorphism because for $h_1, h_2 \in H$, we have
            \[
            \phi(h_1K h_2K) = (h_1 h_2)K = h_1K h_2K = \phi(h_1K) \phi(h_2K).
            \]
            Injectivity follows because if $hK = K$, then $h \in K$, meaning the kernel of $\phi$ is trivial. Hence, $H/K$ is isomorphic to its image in $G/K$ and is thus a subgroup.
        \end{solution}

        \item Conversely, prove that every subgroup of $G/K$ looks like $H/K$ for some subgroup $H$ satisfying $K \subseteq H \subseteq G$.
        
        \begin{solution}
            Let $S \leq G/K$. Define $H = \{ g \in G \mid gK \in S \}$. Then $H$ is a subgroup of $G$, since for $g_1, g_2 \in H$, we have $g_1K, g_2K \in S$, so $g_1 g_2 K \in S$, implying $g_1 g_2 \in H$. Clearly, $K \subseteq H$. Moreover, $S$ corresponds precisely to $H/K$, proving the claim.
        \end{solution}

        \item Prove that $H$ is a normal subgroup of $G$ if and only if $H/K$ is a normal subgroup of $G/K$.
        
        \begin{solution}
            ($\Rightarrow$) If $H$ is normal in $G$, then for all $g \in G$, we have $gH g^{-1} = H$. Taking cosets modulo $K$, we get $gK H/K g^{-1}K = H/K$, proving normality in $G/K$.
            
            ($\Leftarrow$) If $H/K$ is normal in $G/K$, then for all $gK \in G/K$, we have $gK H/K g^{-1}K = H/K$, meaning $gHg^{-1} \subseteq H$. Thus, $H$ is normal in $G$.
        \end{solution}

        \item If $H$ is a normal subgroup of $G$, prove that
              \[
                  \frac{G/K}{H/K} \equiv G/H.
              \]
              (\textit{Hint.} Prove that there is a well-defined surjective homomorphism $G/K \longrightarrow G/H$. What is its kernel?)
       
        \begin{solution}
            Define $\phi: G/K \to G/H$ by $\phi(gK) = gH$. This is well-defined since if $gK = g'K$, then $g^{-1}g' \in K \subseteq H$, implying $gH = g'H$.
            
            This map is a homomorphism because for any $g_1, g_2 \in G$,
            \[
            \phi(g_1K g_2K) = \phi(g_1g_2K) = g_1g_2H = g_1H g_2H = \phi(g_1K) \phi(g_2K).
            \]
            
            The kernel of $\phi$ is $H/K$ since $\phi(gK) = H$ if and only if $gH = H$, meaning $g \in H$. The First Isomorphism Theorem then gives
            \[
            (G/K) / (H/K) \cong G/H.
            \]
        \end{solution}
    \end{enumerate}
\end{problem}


\begin{problem}[6.8]
    Let $G$ be a group, let $K \subseteq G$ be a normal subgroup of $G$, and let $H \subseteq H' \subseteq G$ be subgroups of $G$.
    \begin{enumerate}[label=(\alph*)]
        \item Prove that $H \cap K$ is a normal subgroup of $H$ and similarly that $H' \cap K$ is a normal subgroup of $H'$.
              
        \begin{solution}
            Since $K$ is normal in $G$, for all $h \in H$ and $k \in H \cap K$, we have $hkh^{-1} \in K$. 
            Moreover, since $hkh^{-1} \in H$ because $h, k, h^{-1} \in H$, it follows that $hkh^{-1} \in H \cap K$. 
            Thus, $H \cap K$ is normal in $H$. The same argument applies to $H' \cap K$ in $H'$. 
        \end{solution}

        \item Prove that $H/(H \cap K)$ is naturally a subgroup of $H'/(H' \cap K)$.
        
        \begin{solution}
            The inclusion $H \subseteq H'$ induces a natural homomorphism \[ H \to H'/(H' \cap K) \] given by $h \mapsto h(H' \cap K)$. 
            The kernel of this map is precisely $H \cap K$, so the First Isomorphism Theorem gives an injection \[ H/(H \cap K) \hookrightarrow H'/(H' \cap K). \] 
            Thus, $H/(H \cap K)$ is naturally a subgroup of $H'/(H' \cap K)$. 
        \end{solution}

        \item Suppose further that $H$ is a normal subgroup of $H'$. Prove that $H/(H \cap K)$ is a normal subgroup of $H'/(H' \cap K)$.
        
        \begin{solution}
            Since $H$ is normal in $H'$, conjugation by any element of $H'$ sends $H$ to itself. 
            Given any $h' \in H'$ and coset $h(H \cap K) \in H/(H \cap K)$, we have \[ h' h (H \cap K) h'^{-1} = (h' h h'^{-1}) (H \cap K). \]
            Since $h' h h'^{-1} \in H$ (as $H$ is normal in $H'$), it follows that $h(H \cap K)$ is mapped within $H/(H \cap K)$ under conjugation by elements of $H'/(H' \cap K)$. 
            Thus, $H/(H \cap K)$ is normal in $H'/(H' \cap K)$. 
        \end{solution}
    \end{enumerate}
\end{problem}

\begin{problem}[6.12]
    Let $G$ be a group that acts on a set $X$. We say that the action is \textit{doubly transitive} if it has the following property:
    
    For all $x_1, x_2, y_1, y_2 \in X$ with $x_1 \neq x_2$ and $y_1 \neq y_2$, there exists an element $g \in G$ of the group satisfying $gx_1 = y_1$ and $gx_2 = y_2$.
    \begin{enumerate}[label=(\alph*)]
        \item Let $Z$ be the following set of ordered pairs:
              \[
                  Z = \{ (z_1, z_2) \in X \times X : z_1 \neq z_2 \}
              \]
              Let $G$ act on $Z$ by the rule
              \[
                  g(z_1, z_2) = (gz_1, gz_2)
              \]
              Prove that the action of $G$ on $X$ is doubly transitive if and only if the action of $G$ on $Z$ is transitive.
              
              \begin{solution}
                  Suppose $G$ acts doubly transitively on $X$. Then, given any two pairs $(x_1, x_2), (y_1, y_2) \in Z$, there exists $g \in G$ such that $gx_1 = y_1$ and $gx_2 = y_2$, showing transitivity on $Z$.

                  Conversely, if $G$ acts transitively on $Z$, then for any distinct elements $x_1, x_2$ and $y_1, y_2$, there exists $g \in G$ such that $(gx_1, gx_2) = (y_1, y_2)$. This implies doubly transitive action on $X$. 
              \end{solution}

        \item For each of the following groups and group actions, determine whether the action is transitive, and also whether the action is doubly transitive:
              \begin{enumerate}[label=(\arabic*)]
                  \item The symmetric group $S_n$ acting on the set $\{1, 2, \ldots, n\}$.
                        
                        \begin{solution}
                            The symmetric group $S_n$ acts transitively and doubly transitively on $\{1,2,\dots,n\}$ because it can send any pair $(x_1, x_2)$ to any other pair $(y_1, y_2)$ by permutation. 
                        \end{solution}
                  \item The dihedral group $\mathcal{D}_n$ acting on the vertices of a regular $n$-gon.
                  
                        \begin{solution}
                            The dihedral group $\mathcal{D}_n$ acts transitively on the vertices of the polygon, but it is not doubly transitive for $n > 3$, as reflections preserve orientation. 
                        \end{solution}

                  \item A group $G$ acts on itself via left multiplication; i.e., take $X$ to be another copy of $G$, and let $g \in G$ send $x \in X$ to $gx$.
                  
                        \begin{solution}
                            The left multiplication action is transitive but not doubly transitive, since left multiplication preserves group structure and does not allow arbitrary swaps of two elements.
                        \end{solution}
              \end{enumerate}
    \end{enumerate}
\end{problem}


\begin{problem}[6.16]
    Let $p$ be a prime. We proved in Corollary 6.26 that a group with $p^2$ elements must be abelian.
    Let $G$ be a group with $p^3$ elements.
    \begin{enumerate}[label=(\alph*)]
        \item Mimic the proof of Corollary 6.26 to try to prove that $G$ is abelian. Where does the proof go wrong?
        
        \begin{solution}
        The proof of Corollary 6.26 relies on the fact that a group of order $p^2$ has a nontrivial center and that the quotient by this center is cyclic, which forces the group to be abelian. When we attempt to extend this argument to a group of order $p^3$, we still get a nontrivial center $Z(G)$. If $|Z(G)| = p^2$, then $G/Z(G)$ has order $p$, which is cyclic, forcing $G$ to be abelian. However, if $|Z(G)| = p$, then $G/Z(G)$ has order $p^2$, which is not necessarily cyclic. This is where the proof fails.
        \end{solution}

        \item Give two examples of non-abelian groups with $2^3$ elements. (This shows that the proof in (a) can't work).
        
        \begin{solution}
        Two examples of non-abelian groups of order $2^3$ are:
        \begin{itemize}
            \item The dihedral group $D_4$, which consists of the symmetries of a square.
            \item The quaternion group $Q_8$, which consists of elements $\{ \pm 1, \pm i, \pm j, \pm k \}$ with multiplication rules defined by $i^2 = j^2 = k^2 = ijk = -1$.
        \end{itemize}
        Both of these groups have nontrivial centers of order $2$ but remain non-abelian.
        \end{solution}

        \item What sort of information about $G$ can you deduce from the proof in (a) that failed?
        
        \begin{solution}
        Even though the proof does not show that $G$ is abelian, it does establish that $Z(G)$ is nontrivial and that $G/Z(G)$ has order $p^2$. Since we know that groups of order $p^2$ are abelian, this means that $G$ is at most a central extension of an abelian group, making it close to being abelian in structure.
        \end{solution}

        \item \textbf{Challenge Problem}. Construct a non-abelian group of order $p^3$ for every prime $p$.
        
        \begin{solution}
        One general construction for a non-abelian group of order $p^3$ is given by the Heisenberg group over $\mathbb{Z}/p\mathbb{Z}$:
        \[
        H_p = \left\{ \begin{bmatrix} 1 & a & b \\ 0 & 1 & c \\ 0 & 0 & 1 \end{bmatrix} \mid a, b, c \in \mathbb{Z}/p\mathbb{Z} \right\}.
        \]
        This group has order $p^3$ and is non-abelian whenever $p > 1$ because matrix multiplication does not always commute.
        \end{solution}
    \end{enumerate}
\end{problem}

\begin{problem}[6.21]
    Let $p$ be prime, and let $G$ be a group of order $p^n$. Prove that for every $0 \leq r \leq n$, there is a subgroup
    $H$ of $G$ of order $p^r$. (\textit{Hint.} Give a proof by induction on $n$. Use Theorem 6.25, which says that $G$
    has a non-trivial center $Z(G)$, and apply the induction hypothesis to $G/N$, where $N$ is an appropriately
    chosen subgroup of $Z(G)$).

    \begin{solution}
    We prove the result by induction on $n$.
    
    \textbf{Base case:} When $n = 0$, the trivial group is the only group of order $p^0 = 1$, and it has a subgroup of order $p^0 = 1$.
    
    \textbf{Inductive step:} Assume the statement holds for groups of order $p^k$ for some $k \geq 0$. Let $G$ be a group of order $p^{k+1}$. By Theorem 6.25, $G$ has a non-trivial center $Z(G)$, which contains a subgroup $N$ of order $p$. Consider the quotient group $G/N$, which has order $p^k$. By the induction hypothesis, for every $0 \leq r \leq k$, there exists a subgroup of $G/N$ of order $p^r$. The preimage of this subgroup in $G$ under the natural projection map is a subgroup of $G$ of order $p^{r+1}$. Since $N$ itself is of order $p$, it is also a valid subgroup. Thus, subgroups of all required orders exist in $G$.
    \end{solution}
\end{problem}


\begin{problem}[6.22]
    This exercise asks you to give two different proofs of the following stronger version of the first part of Sylow's Theorem.

    \textit{Theorem.} Let $G$ be a finite group, let $p$ be a prime, and suppose that $\#G$ is divisible by $p^r$. Prove that $G$
    has a subgroup of order $p^r$. (Note that $p^r$ is not required to be the largest power of $p$ that divides $G$).
    \begin{enumerate}[label=(\alph*)]
        \item Give a proof that directly mimics the proof of Theorem 6.29 by considering the set of all subsets of $G$ that contain $p^r$ elements.
              But note that if $n > r$, then ${p^nm}\choose{p^r}$ is divisible by $p$, so you'll need to make some changes in the proof.

        \begin{solution}
        Consider the set $S$ of all subsets of $G$ of size $p^r$. Let's analyze the number of ways to form such subsets.
        
        Define an equivalence relation on $S$ where two subsets are equivalent if they are conjugate under the action of $G$ by left multiplication. The number of such subsets is given by the binomial coefficient $\binom{p^n m}{p^r}$. Since $p^n m$ is divisible by $p^r$, it follows that for sufficiently large $n$, this binomial coefficient is divisible by $p$.
        
        Now, consider the subset stabilizers. By Sylow's counting arguments, the number of such subsets modulo $p$ must be 1, ensuring the existence of a subgroup of order $p^r$. This construction follows a similar argument to the proof of Theorem 6.29 while modifying it to account for divisibility.
        \end{solution}

        \item Combine the version of Sylow's Theorem that we did prove with Exercise 6.21. (If you haven't already done Exercise 6.21, now
              would be a good time to do it!)

        \begin{solution}
        From Sylow's Theorem, we know that if $p^n$ is the highest power of $p$ dividing $\#G$, then there exists a Sylow $p$-subgroup of order $p^n$.
        
        Exercise 6.21 establishes that if a group $G$ has a normal subgroup of order $p^k$, then $G$ contains a subgroup of order $p^r$ for any $r \leq k$.
        
        Applying this, we see that $G$ must have a subgroup of order $p^r$, even if $p^r$ is not the highest power of $p$ dividing $\#G$. Thus, by combining Sylow's Theorem with the result from Exercise 6.21, we conclude the existence of a subgroup of order $p^r$.
        \end{solution}
    \end{enumerate}
\end{problem}


\begin{problem}[6.26]
    This exercise describes a way to create new groups from known groups. Let $G$ be a group. An isomorphism from $G$ to itself
    is called an \textit{automorphism} of $G$. The set of automorphisms is denoted
    \[
        \Aut(G) = \{ \text{group isomorphisms } G \longrightarrow G \}
    \]
    We define a composition law on $\Aut(G)$ as follows: for $\alpha, \beta \in \Aut(G)$, we define $\alpha\beta$ to be the map
    from $G$ to $G$ give by $(\alpha\beta)(g) = \alpha(\beta(g))$.
    \begin{enumerate}[label=(\alph*)]
        \item Prove that this composition law makes $\Aut(G)$ into a group.
        
        \begin{solution}
            Let's verify the group axioms for this:
            
            \textit{Closure}: If $\alpha, \beta \in \Aut(G)$, then $\alpha\beta$ is a composition of two isomorphisms, which is itself an isomorphism. Hence, $\alpha\beta \in \Aut(G)$.

            \textit{Associativity}: For $\alpha, \beta, \gamma \in \Aut(G)$, composition satisfies $\alpha(\beta\gamma) = (\alpha\beta)\gamma$.

            \textit{Identity element}: The identity map $\text{id}_G : G \to G$, given by $\text{id}_G(g) = g$, is an automorphism and serves as the identity.

            \textit{Inverses}: If $\alpha \in \Aut(G)$, then $\alpha^{-1}$ exists and is also an automorphism, ensuring that each element has an inverse.
            
            Thus, $\Aut(G)$ is a group.

        \end{solution}

        \item Let $a \in G$. Define a map $\phi_a$ from $G$ to $G$ by the formula
              \[
                  \phi_a : G \longrightarrow G, \quad \phi_a(g) = aga^{-1}
              \]
              Prove that $\phi_a \in \Aut(G)$ and that the map (6.23 below)
              \[
                  G \longrightarrow \Aut(G), \quad a \longmapsto \phi_a
              \]
              is a group homomorphism.

        \begin{solution}
            We check that $\phi_a(g) = aga^{-1}$ is an automorphism:
            
            \textit{Homomorphism property}: For any $g_1, g_2 \in G$,
            \[
                \phi_a(g_1 g_2) = a g_1 g_2 a^{-1} = (a g_1 a^{-1})(a g_2 a^{-1}) = \phi_a(g_1)\phi_a(g_2).
            \]
            \textit{Invertibility}: The inverse of $\phi_a$ is $\phi_{a^{-1}}$, since $\phi_{a^{-1}}(\phi_a(g)) = a^{-1} (aga^{-1}) a = g$.
            
            The mapping $a \mapsto \phi_a$ respects group operation:
            \[
                \phi_{ab}(g) = (ab) g (ab)^{-1} = a(b g b^{-1}) a^{-1} = \phi_a(\phi_b(g)),
            \]
            so it is a homomorphism.
        \end{solution}

        \item Prove that the kernel of homomorphism (6.23) is the center $Z(G)$ of $G$.
        
        \begin{solution}
            The kernel consists of elements $a \in G$ such that $\phi_a$ is the identity map, meaning $aga^{-1} = g$ for all $g \in G$. This holds precisely when $a$ commutes with all elements of $G$, i.e., $a \in Z(G)$. Hence, $\ker(\phi) = Z(G)$.
        \end{solution}

        \item Elements of $\Aut(G)$ that are equal to $\phi_a$ for some $a \in G$ are called \textit{inner automorphisms},
              and all other elements of $\Aut(G)$ are called \textit{outer automorphisms}. Prove that $G$ is abelian if
              and only if its only inner automorphism is the identity map.

        \begin{solution}
            If $G$ is abelian, then $aga^{-1} = g$ for all $a, g \in G$, implying that all $\phi_a$ are the identity map, making all inner automorphisms trivial. Conversely, if the only inner automorphism is the identity, then $aga^{-1} = g$, meaning $G$ is abelian.
        \end{solution}

        \item More generally, if $H$ is a normal subgroup of $G$, prove that there is a well-defined group homomorphism
              \[
                  G \longrightarrow \Aut(H), \quad a \mapsto \phi_a, \quad \text{where } \phi_a(h) = aha^{-1}, 
              \]
              and that the kernel of this homomorphism is the centralizer of $H$ in $G$.

              \begin{solution}
                The function $G \to \Aut(H)$ given by $a \mapsto \phi_a$ is well-defined because $H$ is normal, ensuring that conjugation preserves $H$. The kernel consists of elements commuting with all of $H$, which defines the centralizer $C_G(H)$ of $H$ in $G$.
              \end{solution}
    \end{enumerate}
\end{problem}


\begin{problem}[6.30]
    Let $p$ and $q$ be odd primes with $q < p$, and let $G$ be a finite group with $\#G = p^nq$ for some $n \geq 1$. 
    Let $H_p$ be a $p$-Sylow subgroup, and let $H_q$ be a $q$-Sylow subgroup.
    \begin{enumerate}[label=(\alph*)]
        \item If $q \not\equiv 1 \hspace{3px} (\text{mod } p)$, prove that $H_p$ is a normal subgroup of $G$.
        
        \begin{solution}
            By Sylow’s theorems, the number of Sylow $p$-subgroups, denoted $n_p$, satisfies $n_p \equiv 1 \pmod{p}$ and divides $q$. Since $q$ is a prime, the divisors of $q$ are $1$ and $q$. If $n_p = 1$, then $H_p$ is unique and thus normal in $G$. Suppose for contradiction that $n_p = q$. Then $q \equiv 1 \pmod{p}$, contradicting the assumption that $q \not\equiv 1 \pmod{p}$. Thus, $n_p = 1$, and $H_p$ is normal in $G$.
        \end{solution}

        \item If $n = 3$ and $q \equiv 2 \hspace{3px} (\text{mod } 3)$, prove that $H_q$ is a normal subgroup of $G$.
              (\textit{Hint.} You may need to use the fact that $-3$ is not a square modulo $q$, which is a special
              case of quadratic reciprocity).

        \begin{solution}
            By Sylow’s theorems, the number of Sylow $q$-subgroups, $n_q$, satisfies $n_q \equiv 1 \pmod{q}$ and divides $p^3$. Thus, $n_q$ is either $1$, $p$, $p^2$, or $p^3$. If $n_q = 1$, then $H_q$ is normal. Suppose for contradiction that $n_q > 1$. Then $n_q \equiv 1 \pmod{q}$ implies that $p^k \equiv 1 \pmod{q}$ for some $k \in \{1,2,3\}$. Since $q \equiv 2 \pmod{3}$, the claim follows from the given hint that $-3$ is not a quadratic residue modulo $q$. This forces $n_q = 1$, implying that $H_q$ is normal in $G$.
        \end{solution}
    \end{enumerate}
\end{problem}

\begin{problem}[6.31]
    Let $p$ and $q$ be distinct primes, and let $G$ be a group of order $p^2q$. Let $H_p$ be a $p$-Sylow
    subgroup, and let $H_q$ be a $q$-Sylow subgroup. Prove that at least one of $H_p$ and $H_q$ is a normal
    subgroup of $G$.

    \begin{solution}
        By Sylow’s theorems, $n_p$ (the number of Sylow $p$-subgroups) satisfies $n_p \equiv 1 \pmod{p}$ and divides $q$, so $n_p \in \{1,q\}$. Similarly, $n_q$ satisfies $n_q \equiv 1 \pmod{q}$ and divides $p^2$, so $n_q \in \{1,p,p^2\}$. If either $n_p = 1$ or $n_q = 1$, then the corresponding Sylow subgroup is normal. If $n_p = q$ and $n_q > 1$, then $n_q \equiv 1 \pmod{q}$ forces $p^k \equiv 1 \pmod{q}$, which contradicts the assumption that $p$ and $q$ are distinct primes. Hence, at least one of $H_p$ or $H_q$ must be normal in $G$.
    \end{solution}
\end{problem}


\end{document}