\documentclass[12pt]{article}
\usepackage[margin=1in]{geometry}

\usepackage{libertine}
\usepackage{parskip}
\usepackage{graphicx}
\graphicspath{ {./images/} }

\usepackage{amsthm,amsmath,amssymb}
\usepackage{tikz}
\usetikzlibrary{arrows,automata,shapes.geometric}

\usepackage{pgfplots}
\pgfplotsset{compat=1.18}

\pagestyle{plain}
\thispagestyle{empty}

\definecolor{carnellian}{RGB}{190,20,20}

\theoremstyle{definition}
\newtheorem{problem}{Problem}

\newcounter{subq}[problem]
\newenvironment{subproblem}
{\refstepcounter{subq} \begin{itemize} \item[(\alph{subq})]}
{\end{itemize} \medskip}
\DeclareMathOperator{\ima}{im}
\DeclareMathOperator{\rank}{rank}
\DeclareMathOperator{\tr}{tr}
\DeclareMathOperator{\der}{Der}
\DeclareMathOperator{\id}{id}
\DeclareMathOperator{\Hom}{Hom}
\DeclareMathOperator{\Char}{char}
\DeclareMathOperator{\Cen}{Cen}
\DeclareMathOperator{\ged}{ged}
\DeclareMathOperator{\tor}{tor}
% \DeclareMathOperator{\span}{span}



\usepackage{environ}
\NewEnviron{solution}[1][\vfill]{
    \textcolor{blue}{\BODY}
}

\newcommand{\hwnum}{9}
\newcommand{\duedate}{11/17/2024}
\renewcommand{\title}{}

\begin{document}

\hspace{-10px}
\begin{tabular*}{\textwidth}{l @{\extracolsep{\fill}} r}
    \textbf{Honors Linear Algebra} 
        & \textbf{Fall 2024} \\
    \textbf{HW\hwnum: \title} &  \textbf{Due: \duedate}
\end{tabular*}

\vspace{1cm}

\begin{problem}[TensorProducts 4]
    Let $R$ be any commutative domain with field of fractions $\mathbb{F} = \{a/b | a,b \in R, b \neq 0\}$ (recall your earlier exercises). Show that:

    \begin{subproblem}
        $\mathbb{F} \otimes_R \mathbb{F} \approx \mathbb{F}$

        \begin{solution}
            % ----------- solution here ----------- %
            To show $\mathbb{F} \otimes_R \mathbb{F} \approx \mathbb{F}$, we use the universal property of the tensor product. 

            First, note that $\mathbb{F} = R[S^{-1}]$, where $S = R \setminus \{0\}$. The elements of $\mathbb{F}$ can be written as $a/b$ with $a, b \in R$ and $b \neq 0$. The tensor product $\mathbb{F} \otimes_R \mathbb{F}$ consists of finite sums of elements of the form $(a/b) \otimes (c/d)$ with $a, b, c, d \in R$ and $b, d \neq 0$.

            Define a map $\phi: \mathbb{F} \otimes_R \mathbb{F} \to \mathbb{F}$ by
            \[
            \phi((a/b) \otimes (c/d)) = \frac{a \cdot c}{b \cdot d}.
            \]
            This map is well-defined, as it respects the relations in $\mathbb{F} \otimes_R \mathbb{F}$. For example:
            \[
            \phi((a/b) \otimes (rc/d)) = \phi((ar/b) \otimes (c/d)) = \frac{a \cdot rc}{b \cdot d} = \frac{(ar)c}{b \cdot d}.
            \]

            Next, $\phi$ is clearly bilinear. To check bijectivity:\\
            \textbf{Injectivity}: If $\phi\left(\sum (a_i/b_i) \otimes (c_i/d_i)\right) = 0$, then $\sum \frac{a_i \cdot c_i}{b_i \cdot d_i} = 0$, which implies the original tensor sum is zero.\\
            \textbf{Surjectivity}: For any $\frac{e}{f} \in \mathbb{F}$, choose $\frac{e}{f} = \phi((e/1) \otimes (1/f))$.

            Thus, $\phi$ is an isomorphism, and we conclude $\mathbb{F} \otimes_R \mathbb{F} \approx \mathbb{F}$.
        
        \end{solution}
    \end{subproblem}
    \begin{subproblem}
        $\mathbb{F} \otimes_R \mathbb{F}/I \approx 0$ for each non-zero ideal $I \subseteq R$.

        \begin{solution}
            % ----------- solution here ----------- %
            To show $\mathbb{F} \otimes_R \mathbb{F}/I \approx 0$, consider a non-zero ideal $I \subseteq R$. Since $R$ is a domain, $I$ contains a non-zero element $r \neq 0$. 

            In $\mathbb{F}$, any element can be written as $a/b$ with $a, b \in R$ and $b \neq 0$. The ideal $I$ induces elements in $\mathbb{F}/I$ of the form $(a + I)/b$. Consider the tensor product $\mathbb{F} \otimes_R (\mathbb{F}/I)$. 

            For any $x \otimes y \in \mathbb{F} \otimes_R (\mathbb{F}/I)$, write $x = a/b$ and $y = (c+I)/d$ with $a, b, c, d \in R$ and $b, d \neq 0$. Then
            \[
            x \otimes y = \frac{a}{b} \otimes \frac{c+I}{d}.
            \]
            Multiply by $r \in I$:
            \[
            r \cdot x \otimes y = \frac{ra}{b} \otimes \frac{c+I}{d}.
            \]
            In $\mathbb{F}/I$, $r \in I$ implies $rc + I = 0$. Thus, $x \otimes y = 0$. Since $x \otimes y$ was arbitrary, $\mathbb{F} \otimes_R (\mathbb{F}/I) = 0$.

            Therefore, $\mathbb{F} \otimes_R (\mathbb{F}/I) \approx 0$ for any non-zero ideal $I \subseteq R$.
        
        \end{solution}
    \end{subproblem}
\end{problem}

\begin{problem}[TensorProducts 6]\phantom{text}

    \begin{subproblem}
        Let $I$ and $J$ be two ideals in the ring $R$. Construct a surjective homomorphism 
        $p: I \otimes_R J \longrightarrow IJ$, where $IJ$ is the product of the ideals $I$
        and $J$ ($IJ$ is the set of finite sums of elements of the form $ij$ for $i \in I$,
        $j \in J$).

        \begin{solution}
            % ----------- solution here ----------- %
            Define the map $p: I \otimes_R J \to IJ$ by
            \[
            p(i \otimes j) = ij \quad \text{for all } i \in I \text{ and } j \in J.
            \]
            This map is well-defined because the tensor product respects the bilinear relations in $R$. For instance, if $r \in R$, then:
            \[
            p((ri) \otimes j) = p((i \otimes rj)) = rij.
            \]
            Clearly, $p$ is linear in both arguments.

            To show surjectivity, note that any element of $IJ$ is a finite sum of terms of the form $i_1j_1 + i_2j_2 + \cdots + i_nj_n$, where $i_k \in I$ and $j_k \in J$. For each such term, $i_k \otimes j_k \mapsto i_kj_k$ under $p$. Therefore, the image of $p$ contains all elements of $IJ$.

            Hence, $p$ is a surjective homomorphism.
        \end{solution}
    \end{subproblem}

    \begin{subproblem}
        Prove that if $I$ (or $J$) is a principal ideal and $R$ is a domain, then $p$ is an 
        isomorphism. 

        \begin{solution}
            % ----------- solution here ----------- %
            Assume $I = (a)$ is a principal ideal, where $a \in R$. Then every element of $I$ can be written as $i = ra$ for some $r \in R$. Similarly, let $j \in J$.

            The tensor product $I \otimes_R J$ can be expressed as:
            \[
            I \otimes_R J = (a) \otimes_R J \cong J \quad \text{(as $R$-modules)}.
            \]
            Now, under the map $p$, we have
            \[
            p(i \otimes j) = p((ra) \otimes j) = ra \cdot j = r(aj) \in IJ.
            \]
            Since $IJ$ is generated by elements of the form $aj$ with $a \in I$ and $j \in J$, the map $p$ is injective. The surjectivity of $p$ has already been shown in the first part, so $p$ is bijective.

            Therefore, when $I$ is principal, $p$ is an isomorphism. The same argument applies if $J$ is principal instead.
        
        \end{solution}
    \end{subproblem}

    \begin{subproblem}
        Show that if $R = \mathbb{Z}[x]$ and $I = J = (2, x)$,  then $p$ is NOT and isomorphism.
        Compute the kernel of $p$.

        \begin{solution}
            % ----------- solution here ----------- %
            Let $R = \mathbb{Z}[x]$ and $I = J = (2, x)$. Then $I$ and $J$ are generated by $\{2, x\}$. In the tensor product $I \otimes_R J$, elements are linear combinations of the form:
            \[
            r_1(2 \otimes 2) + r_2(2 \otimes x) + r_3(x \otimes 2) + r_4(x \otimes x), \quad r_1, r_2, r_3, r_4 \in R.
            \]

            Under the map $p$, we have:
            \[
            p(2 \otimes 2) = 4, \quad p(2 \otimes x) = 2x, \quad p(x \otimes 2) = 2x, \quad p(x \otimes x) = x^2.
            \]
            The image of $p$ is $IJ = (4, 2x, x^2)$, but $I \otimes_R J$ contains more relations than $IJ$. For example, consider the relation:
            \[
            2 \otimes x - x \otimes 2 \in \ker(p),
            \]
            because $p(2 \otimes x - x \otimes 2) = 2x - 2x = 0$. This indicates that $p$ is not injective.

            To compute $\ker(p)$, observe that $\ker(p)$ is generated by all elements of the form:
            \[
            i \otimes j - j \otimes i \quad \text{for } i \in I, j \in J.
            \]
            In this case, $\ker(p)$ is generated by:
            \[
            2 \otimes x - x \otimes 2.
            \]

            Therefore, $p$ is not an isomorphism, and the kernel is:
            \[
            \ker(p) = \langle 2 \otimes x - x \otimes 2 \rangle.
            \]
        \end{solution}
    \end{subproblem}
\end{problem}

\begin{problem}[TensorProducts 14]
    Let $A$ and $B$ be two square matrices. Prove that the Kronecker products $A \otimes B$
    and $B \otimes A$ are similar matrices.

    \begin{solution}
        % ----------- solution here ----------- %
        To show that $A \otimes B$ and $B \otimes A$ are similar matrices, we construct an invertible matrix $P$ such that:
        \[
        P (A \otimes B) P^{-1} = B \otimes A.
        \]

        Let $A$ be an $m \times m$ matrix, and $B$ be an $n \times n$ matrix. The Kronecker product $A \otimes B$ is an $mn \times mn$ matrix. Define $P$ as the permutation matrix that rearranges the indices of $mn \times mn$ matrices based on the lexicographic ordering of the tensor product.

        Specifically, let $e_{i,j}$ denote the standard basis of $m \times n$ matrices. The permutation matrix $P$ is defined such that:
        \[
        P(e_{i,j} \otimes e_{k,l}) = e_{i,k} \otimes e_{j,l},
        \]
        for all $1 \leq i, j \leq m$ and $1 \leq k, l \leq n$. Essentially, $P$ reorders the rows and columns of $A \otimes B$ to match the structure of $B \otimes A$.

        Proof of Similarity:\\
        Consider $A \otimes B$ with entries indexed by pairs of indices $(i,k)$ and $(j,l)$:
        \[
        (A \otimes B)_{(i,k),(j,l)} = A_{i,j} B_{k,l}.
        \]
        Applying the permutation matrix $P$ to $A \otimes B$ results in:
        \[
        P (A \otimes B) P^{-1} = B \otimes A,
        \]
        because the reordering induced by $P$ swaps the role of indices from $(i,k)$ and $(j,l)$ to $(k,i)$ and $(l,j)$, effectively transforming $A \otimes B$ into $B \otimes A$.
        Thus, $A \otimes B$ and $B \otimes A$ are similar matrices.
        
    \end{solution}
\end{problem}

\begin{problem}[Problem 1]
    Let $U$ and $V$ be vector spaces over the complex numbers $\mathbb{C}$.
    Then $U \otimes_\mathbb{C} V$ is also a complex vector space. Note $U, V$
    and $U \otimes_\mathbb{C} V$ may also be regarded as vector spaces over the 
    real numbers $\mathbb{R}$, and we can form $U \otimes_\mathbb{C} V$ and 
    $U \otimes_\mathbb{C} V$ isomorphic as real vector spaces? Prove your answer.

    \begin{solution}
        % ----------- solution here ----------- %
        To address the question, let's first recall a few key points:\\
        1. If $U$ and $V$ are vector spaces over $\mathbb{C}$, then $U \otimes_\mathbb{C} V$ is a vector space over $\mathbb{C}$ with complex dimension:
        \[
        \dim_\mathbb{C}(U \otimes_\mathbb{C} V) = (\dim_\mathbb{C} U) \cdot (\dim_\mathbb{C} V).
        \]
        2. When $U$ and $V$ are regarded as vector spaces over $\mathbb{R}$, their real dimensions are:
        \[
        \dim_\mathbb{R} U = 2 \dim_\mathbb{C} U, \quad \dim_\mathbb{R} V = 2 \dim_\mathbb{C} V.
        \]

        Step 1: Real dimension of $U \otimes_\mathbb{C} V$\\
        Over $\mathbb{R}$, $U \otimes_\mathbb{C} V$ can be expressed as:
        \[
        U \otimes_\mathbb{C} V \cong (U \otimes_\mathbb{R} V) / I,
        \]
        where $I$ is the subspace generated by elements of the form:
        \[
        (au) \otimes v - u \otimes (av) \quad \text{for } a \in \mathbb{C}, u \in U, v \in V.
        \]
        Using the fact that $\dim_\mathbb{R} \mathbb{C} = 2$, we can infer that:
        \[
        \dim_\mathbb{R}(U \otimes_\mathbb{C} V) = 2 (\dim_\mathbb{C} U) \cdot (\dim_\mathbb{C} V).
        \]

        Step 2: Real dimension of $U \otimes_\mathbb{R} V$\\
        When $U$ and $V$ are regarded as real vector spaces, their tensor product over $\mathbb{R}$ is:
        \[
        U \otimes_\mathbb{R} V,
        \]
        which has real dimension:
        \[
        \dim_\mathbb{R}(U \otimes_\mathbb{R} V) = (\dim_\mathbb{R} U) \cdot (\dim_\mathbb{R} V) = [2 (\dim_\mathbb{C} U)] \cdot [2 (\dim_\mathbb{C} V)].
        \]
        Simplifying this gives:
        \[
        \dim_\mathbb{R}(U \otimes_\mathbb{R} V) = 4 (\dim_\mathbb{C} U) \cdot (\dim_\mathbb{C} V).
        \]

        Step 3: Comparing dimensions\\
        From the above calculations:
        - $\dim_\mathbb{R}(U \otimes_\mathbb{C} V) = 2 (\dim_\mathbb{C} U) \cdot (\dim_\mathbb{C} V)$,
        - $\dim_\mathbb{R}(U \otimes_\mathbb{R} V) = 4 (\dim_\mathbb{C} U) \cdot (\dim_\mathbb{C} V)$.

        These dimensions are not equal, so $U \otimes_\mathbb{C} V$ and $U \otimes_\mathbb{R} V$ are **not isomorphic** as real vector spaces.

        Conclusion:
        $U \otimes_\mathbb{C} V$ and $U \otimes_\mathbb{R} V$ are not isomorphic as real vector spaces because their real dimensions differ.
    
    \end{solution}
\end{problem}

\begin{problem}[Problem 2]
    Let $V$ and $W$ be vector spaces over the field $\mathbb{F}$. Using the universal
    mapping property of the tensor product, show that there is a linear transformation
    \[
        H: V^* \otimes W^* \longrightarrow (V \otimes W)^*
    \]
    satisfying
    \[
        [H(f \otimes g)](v \otimes w) = f(v) \cdot g(w)
    \]
    In case both $V$ and $W$ have finite dimension, prove that $H$ is an isomorphism.
    
    \begin{solution}
        % ----------- solution here ----------- %
        Step 1: Constructing the map \( H \)\\
        Let \( f \in V^* \) and \( g \in W^* \), where \( V^* \) and \( W^* \) are the dual spaces of \( V \) and \( W \), respectively. Define \( H: V^* \otimes W^* \to (V \otimes W)^* \) by specifying its action on elementary tensors:
        \[
        [H(f \otimes g)](v \otimes w) = f(v) \cdot g(w),
        \]
        where \( v \in V \) and \( w \in W \).

        \textbf{Well-definedness:}\\
        The universal property of the tensor product ensures that \( H(f \otimes g) \) extends uniquely to a well-defined linear map on the entire tensor product \( V \otimes W \). Thus, \( H \) is a well-defined linear transformation from \( V^* \otimes W^* \) to \( (V \otimes W)^* \).

        Step 2: Linear transformation \( H \)\\
        To check linearity, consider:\\
        \[
        H\left(\sum_i f_i \otimes g_i\right)(v \otimes w) = \sum_i [H(f_i \otimes g_i)](v \otimes w) = \sum_i f_i(v) g_i(w).
        \]
        Since both \( V^* \otimes W^* \) and \( (V \otimes W)^* \) are vector spaces, this confirms \( H \) is linear.

        Step 3: Isomorphism when \( V \) and \( W \) have finite dimension\\
        Assume \( V \) and \( W \) have finite dimensions \( \dim(V) = n \) and \( \dim(W) = m \), respectively.\\ The dimensions of the relevant spaces are:
        \[
        \dim(V^*) = n, \quad \dim(W^*) = m, \quad \dim(V^* \otimes W^*) = n \cdot m,
        \]
        and
        \[
        \dim(V \otimes W) = n \cdot m, \quad \dim((V \otimes W)^*) = \dim(V \otimes W) = n \cdot m.
        \]

        Since \( \dim(V^* \otimes W^*) = \dim((V \otimes W)^*) \), it suffices to show \( H \) is injective and surjective:\\
        \textbf{Injectivity}: Suppose \( H(f \otimes g) = 0 \). This implies:
          \[
          [H(f \otimes g)](v \otimes w) = f(v) g(w) = 0 \quad \forall v \in V, w \in W.
          \]
          Since \( f \) and \( g \) are linear, this implies \( f = 0 \) or \( g = 0 \). Thus, \( f \otimes g = 0 \), and \( H \) is injective.

        \textbf{Surjectivity}: For any linear functional \( \phi \in (V \otimes W)^* \), define \( f_i \in V^* \) and \( g_j \in W^* \) such\\
        that \( \phi(v \otimes w) = \sum_{i,j} f_i(v) g_j(w) \). Then \( \phi \) is in the image of \( H \), proving surjectivity.
    
    \end{solution}
\end{problem}

\begin{problem}[Problem 3]
    Let $U$ and $V$ be vector spaces over the field $\mathbb{F}$. Show there is a linear 
    transformation $L: U^* \otimes V \longrightarrow \hom_{\mathbb{F}}(U, V)$ defined by the
    formula $L(\sum f_i \otimes v_i)(u) = \sum f_i(u)v_i$. Do this by first defining an appropriate
    bilinear function, and then use the universal mapping property. If $U$ and $V$ are finite dimensional,
    show that this map is an isomorphism. [Hint: first compute the dimensions.]
    
    \begin{solution}
        % ----------- solution here ----------- %
        Step 1: Defining an appropriate bilinear function\\
        Let \( f_i \in U^* \) and \( v_i \in V \). Define a bilinear function \( \Phi: U^* \times V \to \hom_\mathbb{F}(U, V) \) by:
        \[
        \Phi(f, v)(u) = f(u) v \quad \text{for all } f \in U^*, v \in V, u \in U.
        \]
        Linearity in \( f \):\\
        For \( f_1, f_2 \in U^* \) and \( a \in \mathbb{F} \),
        \[
        \Phi(af_1 + f_2, v)(u) = (af_1 + f_2)(u)v = a f_1(u)v + f_2(u)v = a\Phi(f_1, v)(u) + \Phi(f_2, v)(u).
        \]
        Linearity in \( v \):\\
        For \( v_1, v_2 \in V \) and \( b \in \mathbb{F} \),
        \[
        \Phi(f, bv_1 + v_2)(u) = f(u)(bv_1 + v_2) = bf(u)v_1 + f(u)v_2 = b\Phi(f, v_1)(u) + \Phi(f, v_2)(u).
        \]
        Hence, \( \Phi \) is bilinear.

        Step 2: Using UMP:\\
        By the universal property of the tensor product, the bilinear function \( \Phi \) induces a unique linear map:
        \[
        L: U^* \otimes V \to \hom_\mathbb{F}(U, V),
        \]
        such that:
        \[
        L(f \otimes v)(u) = \Phi(f, v)(u) = f(u) v.
        \]
        For a general tensor \( \sum f_i \otimes v_i \in U^* \otimes V \), the action of \( L \) is given by:
        \[
        L\left(\sum f_i \otimes v_i\right)(u) = \sum \Phi(f_i, v_i)(u) = \sum f_i(u)v_i.
        \]
        Thus, \( L \) is well-defined and linear.

        Step 3: Proving \( L \) is an isomorphism for finite-dimensional \( U \) and \( V \)\\
        Assume \( \dim(U) = n \) and \( \dim(V) = m \). Then:\\
        The dimension of \( U^* \) is \( \dim(U^*) = n \),
        The dimension of \( U^* \otimes V \) is:
        \[
        \dim(U^* \otimes V) = \dim(U^*) \cdot \dim(V) = n \cdot m.
        \]
        The space \( \hom_\mathbb{F}(U, V) \) consists of all linear maps from \( U \) to \( V \), and its dimension is:
        \[
        \dim(\hom_\mathbb{F}(U, V)) = \dim(U) \cdot \dim(V) = n \cdot m.
        \]

        Since \( \dim(U^* \otimes V) = \dim(\hom_\mathbb{F}(U, V)) \), it suffices to show \( L \) is injective and surjective:\\
        \textbf{Injectivity}: Suppose \( L\left(\sum f_i \otimes v_i\right) = 0 \). Then for all \( u \in U \),
        \[
        L\left(\sum f_i \otimes v_i\right)(u) = \sum f_i(u) v_i = 0.
        \]
        Since the \( v_i \)'s are linearly independent in \( V \), this implies \( f_i(u) = 0 \) for all \( u \in U \), meaning \( f_i = 0 \). Hence, \( \sum f_i \otimes v_i = 0 \), and \( L \) is injective.

        \textbf{Surjectivity}: For any \( T \in \hom_\mathbb{F}(U, V) \), define \( f_i \in U^* \) and \( v_i \in V \) such that \( T(u) = \sum f_i(u) v_i \). Then \( T = L\left(\sum f_i \otimes v_i\right) \), proving \( L \) is surjective.

        Thus, \( L: U^* \otimes V \to \hom_\mathbb{F}(U, V) \) is a linear isomorphism when \( U \) and \( V \) are finite-dimensional vector spaces.

    \end{solution}
\end{problem}

\begin{problem}[Problem 4]
    Assume that $U$ and $V$ are finite-dimensional. In the previous problem, you showed that
    $L$ identifies $U^* \otimes V$ with $\hom_{\mathbb{F}}(U, V)$, where $L(f \otimes v)$ is
    the linear map sending $u$ to $f(u)v$.

    \begin{subproblem}
        Show that $L(f \otimes v)$ is a linear transformation of rank 1 if and only if $f$ and $v$ are nonzero.
        
        \begin{solution}
            % ----------- solution here ----------- %
            \textbf{Showing $\implies$}:\\
            Let \( f \in U^* \) and \( v \in V \) with \( f \neq 0 \) and \( v \neq 0 \). \\
            By definition, \( L(f \otimes v)(u) = f(u)v \) for \( u \in U \). \\
            The image of \( L(f \otimes v) \) is spanned by \( v \) because \( f(u) \in \mathbb{F} \), and the output is always a scalar multiple of \( v \).\\
            Since \( v \neq 0 \), the image of \( L(f \otimes v) \) is a one-dimensional subspace of \( V \).\\
            Thus, \( \operatorname{rank}(L(f \otimes v)) = 1 \).\\

            \textbf{Showing $\impliedby$}:\\
            Assume \( L(f \otimes v) \) has rank 1.\\
            This means the image of \( L(f \otimes v) \) is spanned by a single vector \( v' \neq 0 \).\\
            For \( L(f \otimes v)(u) = f(u)v \), this is only possible if \( v \neq 0 \).\\
            Additionally, if \( f = 0 \), then \( f(u) = 0 \) for all \( u \in U \), and \( L(f \otimes v) \) is the zero map, which contradicts \( \operatorname{rank}(L(f \otimes v)) = 1 \).\\
            Hence, \( f \neq 0 \) and \( v \neq 0 \).

            Therefore, \( L(f \otimes v) \) is a rank-1 linear transformation if and only if \( f \neq 0 \) and \( v \neq 0 \).
        
        \end{solution}
    \end{subproblem}

    \begin{subproblem}
        Show that the rank of an arbitrary linear transformation $T: U \longrightarrow V$
        is the smallest integer $r$ such that $T$ can be expressed in the form
        \[
            T = L(\sum_{i = 1}^{r} f_i \otimes v_i)
        \]
        with $f_i \in U$ and $v_i \in V$.
        
        \begin{solution}
            % ----------- solution here ----------- %
            Let \( T \in \hom_\mathbb{F}(U, V) \) be a linear transformation with \( \operatorname{rank}(T) = r \). \\
            By the rank-nullity theorem, \( T(U) \) is an \( r \)-dimensional subspace of \( V \).\\
            There exist \( v_1, \ldots, v_r \in V \) such that \( \{v_1, \ldots, v_r\} \) is a basis for \( \operatorname{Im}(T) \).\\
            For each \( v_i \), there exists \( f_i \in U^* \) such that \( T(u) = \sum_{i=1}^r f_i(u)v_i \) for all \( u \in U \).\\
            This is equivalent to \( T = L\left(\sum_{i=1}^r f_i \otimes v_i\right) \).

            Minimality of \( r \):\\
            Suppose \( T = L\left(\sum_{i=1}^s f_i \otimes v_i\right) \) for \( s < r \).\\
            Then the image of \( T \) would be spanned by fewer than \( r \) vectors, contradicting the fact that \( \operatorname{rank}(T) = r \).\\
            Thus, \( r \) is the smallest integer such that \( T = L\left(\sum_{i=1}^r f_i \otimes v_i\right) \).

            Thus, the rank of \( T \) is the smallest integer \( r \) such that \( T \) can be expressed as \( T = L\left(\sum_{i=1}^r f_i \otimes v_i\right) \) with \( f_i \in U^* \) and \( v_i \in V \).
        
        \end{solution}
    \end{subproblem}
\end{problem}

\begin{problem}[problem 5]
    In this problem, we will investigate what $\otimes_R$ does to surjective and injective maps, for various $R$.

    \begin{subproblem}
        Let $\mathbb{F}$ be a field, let
        \[
            0 \xrightarrow{\makebox[2cm]{}} V' \xrightarrow{\makebox[2cm]{$\iota$}} V \xrightarrow{\makebox[2cm]{$\pi$}} V'' \xrightarrow{\makebox[2cm]{}} 0
        \]
        be a short exact sequence of vector spaces, and let $W$ be a vector space. From the previous problem, one can define maps
        $\iota \otimes \id_W : V \otimes_{\mathbb{F}} W \longrightarrow V'' \otimes_{\mathbb{F}} W$, where $\id_W$ is the identity map of $W$.

        Show that they fit into the following short exact sequence" 
        \[
            0 \xrightarrow{\makebox[2cm]{}} V' \otimes_{\mathbb{F}} W \xrightarrow{\makebox[2cm]{$\iota \otimes \id_W$}} V \otimes_{\mathbb{F}} W \xrightarrow{\makebox[2cm]{$\pi \otimes \id_W$}} V'' \otimes_{\mathbb{F}} W \xrightarrow{\makebox[2cm]{}} 0
        \]
        *Note: Most of the proof should \textit{not} use the fact that $V$, $V'$ or $W$ are vector spaces, but there is a point where it is cruical
        (See the next part for a hint on where to look!)

        \begin{solution}
            % ----------- solution here ----------- %
            To prove the exactness of the sequence:
            Exactness at \( V' \otimes_{\mathbb{F}} W \):\\
                The map \( \iota: V' \to V \) is injective, so \( \iota \otimes \id_W: V' \otimes_{\mathbb{F}} W \to V \otimes_{\mathbb{F}} W \) is also injective. This follows from the fact that the tensor product of an injective map with the identity map remains injective over fields.

            Exactness at \( V \otimes_{\mathbb{F}} W \):\\
                For \( v \otimes w \in V \otimes_{\mathbb{F}} W \), \( (\pi \otimes \id_W)(\iota \otimes \id_W)(v' \otimes w) = (\pi \circ \iota)(v') \otimes w = 0 \) because \( \pi \circ \iota = 0 \). Hence, \( \operatorname{Im}(\iota \otimes \id_W) \subseteq \ker(\pi \otimes \id_W) \).

                Conversely, if \( (\pi \otimes \id_W)(v \otimes w) = 0 \), then \( \pi(v) = 0 \), so \( v \in \ker(\pi) = \operatorname{Im}(\iota) \). Therefore, \( v = \iota(v') \) for some \( v' \in V' \), and \( v \otimes w = (\iota \otimes \id_W)(v' \otimes w) \). Thus, \( \ker(\pi \otimes \id_W) \subseteq \operatorname{Im}(\iota \otimes \id_W) \), proving exactness at \( V \otimes_{\mathbb{F}} W \).

            Exactness at \( V'' \otimes_{\mathbb{F}} W \):\\
                The map \( \pi: V \to V'' \) is surjective, so \( \pi \otimes \id_W: V \otimes_{\mathbb{F}} W \to V'' \otimes_{\mathbb{F}} W \) is surjective as well. Thus, \( \operatorname{Im}(\pi \otimes \id_W) = V'' \otimes_{\mathbb{F}} W \).

            Therefore, the sequence is exact:
            \[
                0 \to V' \otimes_{\mathbb{F}} W \xrightarrow{\iota \otimes \id_W} V \otimes_{\mathbb{F}} W \xrightarrow{\pi \otimes \id_W} V'' \otimes_{\mathbb{F}} W \to 0.
            \]
        \end{solution}
    \end{subproblem}

    \begin{subproblem}
        Let $R$ be a commutative domain, let $a \in R$ be a nonzero nonunit. Let $M$ be an $R$-module. 
        One then has an exact sequence
        \[
            0 \xrightarrow{\makebox[1cm]{}} R \xrightarrow{\makebox[1cm]{$\iota$}} R \xrightarrow{\makebox[1cm]{}} R/(a) \xrightarrow{\makebox[1cm]{}} 0
        \]
        where $\iota(r) = ar$ is multiplication by $a$.

        \begin{enumerate}
            \item[i] Show that one always has an exact sequence
            \[
                R \otimes_R M \xrightarrow{\makebox[2cm]{$\iota \otimes \id_M$}} R \otimes_R M \xrightarrow{\makebox[2cm]{}} R/(a) \otimes_R M \xrightarrow{\makebox[2cm]{}} 0
            \]
            Note the lack of zero on the left hand side! (See the next part.)

            \begin{solution}
                % ----------- solution here ----------- %
                Tensor the given exact sequence with \( M \):
                \[
                    R \otimes_R M \xrightarrow{\iota \otimes \id_M} R \otimes_R M \xrightarrow{} R/(a) \otimes_R M \to 0.
                \]

                The map \( \iota \otimes \id_M \) is induced by multiplication by \( a \), sending \( r \otimes m \) to \( ar \otimes m \). 

                Surjectivity of \( R \to R/(a) \) implies surjectivity of \( R \otimes_R M \to R/(a) \otimes_R M \), giving the exact sequence. The lack of \( 0 \) on the left arises because \( R \otimes_R M \to R \otimes_R M \) need not be injective.
            \end{solution}

            \item[ii] Show by example that the map
            \[
                R \otimes_R M \xrightarrow{\makebox[2cm]{$\iota \otimes \id_M$}} R \otimes_R M
            \]
            need not be injective (for certain $M$), so that one cannot keep the zero on the left in general when tensoring with modules over
            general rings. Find a condition on $M$ that ensures that the map $\iota \otimes_R M$ is injective in the case we are considering 
            from part i. (Hint: Try $M = R/(a)$).

            \begin{solution}
                % ----------- solution here ----------- %
                Example:\\
                Take \( R = \mathbb{Z} \), \( a = 2 \), and \( M = \mathbb{Z}/2\mathbb{Z} \). Then \( \iota \otimes \id_M \) maps \( r \otimes m \) to \( 2r \otimes m \). Since \( 2m = 0 \) in \( M \), this map sends all elements of \( R \otimes M \) to 0, hence it is not injective.

                Condition for injectivity:\\
                The map \( \iota \otimes \id_M \) is injective if \( M \) is torsion-free. For example, if \( M = R/(a) \), then \( a \cdot m = 0 \) in \( M \), ensuring injectivity.
            
            \end{solution}
        \end{enumerate}
    \end{subproblem}
\end{problem}

\end{document}


