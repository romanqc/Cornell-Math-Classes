\documentclass[12pt]{article}
\documentclass{article}
\usepackage{amsmath, amssymb, amsthm}
\usepackage{amsfonts}
\usepackage{tikz}
\usetikzlibrary{matrix,arrows}

\newtheorem{theorem}{Theorem}[section]
\newtheorem{definition}[theorem]{Definition}
\newtheorem{corollary}[theorem]{Corollary}

\begin{document}

\title{De Rham Cohomology, Hodge Decomposition, and Vector Field Analysis}
\author{Roman Q. Chavez}
\date{December 1, 2023}
\maketitle

\section{Introduction}
In differential geometry, the study of cohomology is essential for understanding the topology of smooth manifolds. De Rham cohomology provides a powerful tool for this purpose, and its connection to the Hodge decomposition is particularly insightful.

\section{De Rham Cohomology}
Let $M$ be a smooth manifold. The De Rham cohomology groups $H^k_{\text{dR}}(M)$ are defined as the cohomology groups of the de Rham complex.

\begin{definition}[De Rham Complex]
The De Rham complex on $M$ is given by
\[
0 \rightarrow \Omega^0(M) \xrightarrow{d} \Omega^1(M) \xrightarrow{d} \Omega^2(M) \xrightarrow{d} \cdots,
\]
where $\Omega^k(M)$ is the space of differential $k$-forms on $M$ and $d$ is the exterior derivative.
\end{definition}

The cohomology groups $H^k_{\text{dR}}(M)$ represent the equivalence classes of closed and cohomologous $k$-forms on $M$.

\section{Hodge Decomposition}
The Hodge decomposition on a compact, oriented Riemannian manifold $M$ states that any differential form on $M$ can be uniquely decomposed into a sum of three orthogonal components:
\[
\Omega^k(M) = \mathcal{H}^k(M) \oplus d\Omega^{k-1}(M) \oplus \delta\Omega^{k+1}(M),
\]
where $\mathcal{H}^k(M)$ is the space of harmonic forms, $d$ is the exterior derivative, and $\delta$ is the codifferential.

\begin{theorem}[Hodge Decomposition]
For any differential form $\alpha \in \Omega^k(M)$, there exist unique forms $\omega \in \mathcal{H}^k(M)$, $\beta \in \Omega^{k-1}(M)$, and $\gamma \in \Omega^{k+1}(M)$ such that
\[
\alpha = \omega + d\beta + \delta\gamma.
\]
\end{theorem}

\section{Connection to Vector Field Analysis}
Now, let's explore the connection between the Hodge decomposition and the analysis of vector fields.

Consider a vector field $V$ on $M$. We can decompose $V$ into its gradient, vorticity, and flux components.

\begin{definition}[Vector Field Decomposition]
Let $V$ be a vector field on $M$. Then, there exist unique vector fields $W_{\text{grad}}, W_{\text{vort}}, W_{\text{div}}$ such that
\[
V = W_{\text{grad}} + W_{\text{vort}} + W_{\text{div}},
\]
where
\begin{align*}
W_{\text{grad}} &= \nabla f \quad \text{(gradient)}, \\
W_{\text{vort}} &= \nabla \times A \quad \text{(vorticity)}, \\
W_{\text{div}} &= \nabla \cdot B \quad \text{(flux)},
\end{align*}
for suitable functions $f$ and vector fields $A, B$.
\end{definition}

This decomposition is analogous to the Hodge decomposition, where the gradient corresponds to the harmonic forms, the vorticity corresponds to the coexact forms, and the flux corresponds to the exact forms.

% ... (previous content)

\section{Approximation of Closed Forms}

In this section, we consider the problem of finding the closest exact form to a given closed form on a manifold $M$. We will utilize the $L^2$ inner product on differential forms to quantify the distance between forms.

\begin{definition}[$L^2$ Inner Product]
For a compact Riemannian manifold $M$ with volume form $\mathrm{Vol}$, the $L^2$ inner product of two $k$-forms $\alpha, \beta$ is defined as
\[
\langle \alpha, \beta \rangle = \int_M \alpha \wedge \ast \beta,
\]
where $\ast$ denotes the Hodge star operator.
\end{definition}

For a closed form $\alpha$ that is not exact, we aim to find the closest exact form to $\alpha$. Specifically, we want to find $\phi_0$ such that $\|\alpha - d\phi_0\| = \inf_{\phi}\|\alpha - d\phi\|$, where $\|\cdot\|$ denotes the $L^2$ norm induced by the inner product.

\begin{theorem}[Approximation of Closed Forms]
Let $M$ be a compact Riemannian manifold, and let $\alpha$ be a closed $k$-form on $M$ that is not exact. Then, there exists a unique $(k-1)$-form $\phi_0$ such that
\[
\|\alpha - d\phi_0\| \leq \|\alpha - d\phi\|
\]
for all $(k-1)$-forms $\phi$.
\end{theorem}

\begin{proof}
Consider the space $\Omega^{k-1}(M)$ of smooth $(k-1)$-forms on $M$. Define the set
\[
\mathcal{A} = \{\phi \in \Omega^{k-1}(M) : d\phi \text{ is closed}\}.
\]
For any $\phi \in \mathcal{A}$, the form $\alpha - d\phi$ is closed, and by Poincaré's lemma, there exists a $(k-2)$-form $\psi$ such that $\alpha - d\phi = d\psi$. Now, let $\phi_0$ be any element in $\mathcal{A}$ such that
\[
\|\alpha - d\phi_0\|^2 = \langle \alpha - d\phi_0, \alpha - d\phi_0 \rangle
\]
is minimized.

Since $\alpha - d\phi_0$ is exact, we have
\[
\alpha - d\phi_0 = d\psi_0
\]
for some $(k-2)$-form $\psi_0$. Then, for any $(k-1)$-form $\phi$,
\begin{align*}
\|\alpha - d\phi_0\|^2 &\leq \langle \alpha - d\phi, \alpha - d\phi \rangle \\
&= \langle d\psi_0, \alpha - d\phi \rangle \\
&= \langle d\psi_0, \alpha \rangle - \langle d\psi_0, d\phi \rangle \\
&= \langle d\psi_0, \alpha \rangle - \langle d\ast d\psi_0, \phi \rangle,
\end{align*}
where we used the fact that $d\phi$ is exact, and thus, $d\ast d\phi = 0$. The equality holds when $\phi = \phi_0$, implying that $\phi_0$ is the desired form.

This establishes the existence of $\phi_0$. Uniqueness follows from the fact that if $\alpha - d\phi_0 = d\psi_0$, then $\phi_0$ is unique up to the addition of an exact form.
\end{proof}

This theorem provides a constructive way to find the closest exact form to a closed form, which is crucial in applications involving differential forms on manifolds.

% ... (continue with the rest of the document)

% ... (previous content)

\section{Harmonic Forms on Manifolds with Boundary}

Now, let's consider a manifold $M^n$ embedded in $\mathbb{R}^n$ with boundary $\partial M = \varnothing$. We are interested in harmonic forms on $M$.

\begin{definition}[Harmonic Forms]
Let $M$ be a compact Riemannian manifold with boundary $\partial M = \varnothing$. A differential $k$-form $\alpha$ on $M$ is called \textit{harmonic} if it satisfies the following conditions:
\[
\begin{aligned}
    &\text{(1)} && d\alpha = 0, \\
    &\text{(2)} && \delta\alpha = 0,
\end{aligned}
\]
where $d$ is the exterior derivative and $\delta$ is the codifferential.
\end{definition}

The space of harmonic $k$-forms is denoted by $\mathcal{H}^k(M)$. We are interested in the following result:

\begin{theorem}[Existence of Harmonic Representatives]
Let $M^n$ be a compact Riemannian manifold with boundary $\partial M = \varnothing$. For any $k$-form $\alpha$ on $M$, there exists a unique $(k-1)$-form $\omega_0$ such that
\[
\int_M |\alpha - d\omega_0|^2 \leq \int_M |\alpha - d\omega'|^2
\]
for all $(k-1)$-forms $\omega'$ if and only if $d\ast(\alpha - d\omega_0) = 0$.
\end{theorem}

\begin{proof}
Let $\mathcal{A} = \{\omega \in \Omega^{k-1}(M) : d\omega \text{ is closed}\}$, and consider the set
\[
\mathcal{B} = \{\omega \in \mathcal{A} : \delta(\alpha - d\omega) = 0\}.
\]

Similar to the previous proof, we can show that there exists a unique element $\omega_0$ in $\mathcal{B}$ that minimizes $\|\alpha - d\omega_0\|^2$. This implies that $\omega_0$ is the desired form.

The condition $d\ast(\alpha - d\omega_0) = 0$ ensures that $\delta(\alpha - d\omega_0) = 0$, and thus, $\omega_0$ lies in $\mathcal{B}$. The proof of uniqueness is analogous to the previous theorem.

Therefore, the existence of $\omega_0$ is guaranteed if and only if $d\ast(\alpha - d\omega_0) = 0$.
\end{proof}

Here are some images of 3-dimensional manifolds I created in Java Processing.

\begin{figure}[ht]
    \centering
    \includegraphics[width=0.6\textwidth]{manifold.png}
    \caption{Here is a closed manifold in $\mathbb{R}^{3}$ in the shape of a morphed bean.}
\end{figure}

\begin{figure}[ht]
    \centering
    \includegraphics[width=0.6\textwidth]{sphere.png}
    \caption{This is an image of a sphere.}
\end{figure}

% ... (continue with the rest of the document)

\section{Conclusion}
The interplay between De Rham cohomology, Hodge decomposition, and vector field analysis provides a rich framework for understanding the geometric and topological properties of smooth manifolds. The decomposition of vector fields into gradient, vorticity, and flux components reflects the underlying structure revealed by De Rham cohomology.

\end{document}

% \DeclareMathOperator{\span}{span}



\usepackage{environ}
\NewEnviron{solution}[1][\vfill]{
    \textcolor{blue}{\BODY}
}

\newcommand{\hwnum}{9}
\newcommand{\duedate}{12/4/2024}
\renewcommand{\title}{}

\begin{document}

\hspace{-10px}
\begin{tabular*}{\textwidth}{l @{\extracolsep{\fill}} r}
    \textbf{Honors Linear Algebra} 
        & \textbf{Fall 2024} \\
    \textbf{HW\hwnum: \title} &  \textbf{Due: \duedate}
\end{tabular*}

\vspace{1cm}

\begin{problem}[Determinants 2]
    Prove that if $M \neq 0$ is a free module then $M^{\otimes p} \neq 0$ for any $p \geq 1$.

    \begin{solution}
        1. Free Module and Tensor Product:\\
        Since \( M \) is a free module, there exists a basis \( \{e_i\}_{i \in I} \) for \( M \). The tensor product \( M^{\otimes p} \) is the \( p \)-fold tensor power of \( M \), which is also a module over the same ring \( R \).

        2. Basis of Tensor Product:\\
        The elements of \( M^{\otimes p} \) are linear combinations of tensors of the form \( e_{i_1} \otimes e_{i_2} \otimes \cdots \otimes e_{i_p} \), where \( i_k \in I \) for \( k = 1, \dots, p \). Thus, the basis of \( M^{\otimes p} \) is given by all possible such products of basis elements of \( M \).

        3. Non-Zero Property:\\
        Since \( M \neq 0 \), its basis \( \{e_i\} \) is non-empty, implying \( I \neq \emptyset \). Therefore, the basis of \( M^{\otimes p} \) is also non-empty, as it consists of all \( p \)-fold tensors of the non-empty basis of \( M \). Hence, \( M^{\otimes p} \neq 0 \).

        Thus, the assumption that \( M \neq 0 \) ensures the existence of a non-empty basis for \( M \), which in turn guarantees that \( M^{\otimes p} \) is non-zero for any \( p \geq 1 \).
    \end{solution}
\end{problem}

\begin{problem}[Determinants 5]
    Prove that if $M$ is a free module with basis $\mathcal{B} = \{m_1, m_2, \ldots, m_n\}$
    then the elements $m_{i_1} \wedge \cdots \wedge m_{i_p}$ where $i_1 < i_2 < \cdots < i_p$
    is a strictly ascending sequence are linearly independent by constructing many alternating
    maps for $M^{\times p} \longrightarrow R$.

    \begin{solution}
        1. Definition of Alternating Map:\\  
        An alternating map \( f: M^{\times p} \to R \) is a multilinear map satisfying the property that\\
        \( f(m_{\sigma(1)}, \ldots, m_{\sigma(p)}) = \operatorname{sgn}(\sigma) f(m_1, \ldots, m_p) \), where \( \sigma \) is a permutation of \( \{1, \ldots, p\} \), and \( \operatorname{sgn}(\sigma) \) is the sign of the permutation. If \( m_i = m_j \) for some \( i \neq j \), then \( f(m_1, \ldots, m_p) = 0 \).

        2. Construction of Maps:\\
        Let \( I = \{i_1, i_2, \ldots, i_p\} \), where \( i_1 < i_2 < \cdots < i_p \), be a strictly ascending sequence of indices. Define \( f_I: M^{\times p} \to R \) by:
        \[
        f_I(m_{j_1}, \ldots, m_{j_p}) =
        \begin{cases}
            \det(A) & \text{if } (j_1, \ldots, j_p) = (i_1, \ldots, i_p), \\
            0       & \text{otherwise},
        \end{cases}
        \]
        where \( A \) is the \( p \times p \) matrix whose columns are the coordinates of \( m_{i_1}, \ldots, m_{i_p} \) with respect to the basis \( \mathcal{B} \).

        3. Linearly Independent Elements:\\  
        Each \( f_I \) is an alternating map that vanishes on any wedge product \( m_{j_1} \wedge \cdots \wedge m_{j_p} \) unless \( \{j_1, \ldots, j_p\} = \{i_1, \ldots, i_p\} \). Thus, applying \( f_I \) to \( m_{i_1} \wedge \cdots \wedge m_{i_p} \) gives:
        \[
        f_I(m_{i_1}, \ldots, m_{i_p}) = \det(A) \neq 0,
        \]
        while \( f_I(m_{j_1}, \ldots, m_{j_p}) = 0 \) for any other strictly ascending sequence \( \{j_1, \ldots, j_p\} \neq \{i_1, \ldots, i_p\} \).
 
        Since we can construct one alternating map \( f_I \) for each strictly ascending sequence \( \{i_1, \ldots, i_p\} \), and each \( f_I \) uniquely distinguishes \( m_{i_1} \wedge \cdots \wedge m_{i_p} \), the set \( \{m_{i_1} \wedge \cdots \wedge m_{i_p} : i_1 < i_2 < \cdots < i_p\} \) is linearly independent. 
        Thus, we have proven that \( m_{i_1} \wedge \cdots \wedge m_{i_p} \) are linearly independent by constructing alternating maps.

    \end{solution}
\end{problem}

\begin{problem}[Determinants 8]
    An associative algebra $A$ is called $\mathbb{Z}/2\mathbb{Z}$ graded if $A = A^(0) \oplus A^(1)$
    and the multiplication preserves the grading, i.e.,
    \[
        A^{(i)} \cdot A^{(j)} \subset A^{(i+j)}
    \]
    for any $i, j$, where the addition is mod 2. The algebra is called supper commutative if
    \[
        a \cdot b = (-1)ijba \text{for any} a \in A^{(i)}, b \in A^{(j)}
    \]

    \begin{subproblem}
        Verify the exterior algebra is super commutative. How do you define the grading?

        \begin{solution}
            1. Definition of Grading:  
            The exterior algebra \( \Lambda^*(M) \) over an \( R \)-module \( M \) is graded by degree, where the grading is given as:
            \[
            \Lambda^*(M) = \bigoplus_{k \geq 0} \Lambda^k(M),
            \]
            with \( \Lambda^k(M) \) denoting the \( k \)-th exterior power of \( M \). To define a \( \mathbb{Z}/2\mathbb{Z} \)-grading, set:
            \[
            \Lambda^{(0)}(M) = \bigoplus_{k \text{ even}} \Lambda^k(M), \quad \Lambda^{(1)}(M) = \bigoplus_{k \text{ odd}} \Lambda^k(M).
            \]

            2. Super Commutativity:  
            For \( x \in \Lambda^i(M) \) and \( y \in \Lambda^j(M) \), the wedge product satisfies:
            \[
            x \wedge y = (-1)^{ij} y \wedge x.
            \]
            This follows from the alternating property of the wedge product, which imposes the sign rule when swapping two elements. This property ensures that \( \Lambda^*(M) \) is super commutative under the \( \mathbb{Z}/2\mathbb{Z} \)-grading.

            Thus, the exterior algebra \( \Lambda^*(M) \) is super commutative with the grading defined as above.
        \end{solution}
    \end{subproblem}

    \begin{subproblem}
        Show that the exterior algebra satisfies the following universal property 
        (assuming that 2 is invertible in $R$): Let $M$ be an $R$-module. Then for 
        any super commutative associative unitial $R$-algebra $A$ and any $R$-module
        homomorphism $\varphi : M \longrightarrow A^{(1)}$ there exists exactly one
        $R$-algebra map $\bar{\varphi}$ such that the diagram commutes
        \begin{center}
            \begin{tikzcd}[column sep=huge, row sep=huge]
                M \arrow[r, hook, "i"] \arrow[dr, "\varphi"]
                & \Lambda^*(M) \arrow[d, dashrightarrow, "\bar{\varphi}"]\\
                & A
            \end{tikzcd}
        \end{center}
        i.e., $\varphi = \bar{\varphi} \circ i$, where $i : M \longhookrightarrow \Lambda^*M$ 
        comes from the identification $M = M^{\otimes 1}$.

        \begin{solution}
            To show that the exterior algebra satisfies the universal property, assume \( \varphi : M \to A^{(1)} \) is an \( R \)-module homomorphism and \( A \) is a super commutative associative unital \( R \)-algebra. 

            1. Existence of \( \bar{\varphi} \):  
            The universal property of the exterior algebra \( \Lambda^*(M) \) states that there exists a unique \( R \)-algebra map \( \bar{\varphi} : \Lambda^*(M) \to A \) such that \( \bar{\varphi} \circ i = \varphi \), where \( i: M \hookrightarrow \Lambda^*(M) \) is the canonical inclusion.

            2. Construction of \( \bar{\varphi} \):  
            Since \( A \) is super commutative, the image of \( \varphi(M) \subset A^{(1)} \) satisfies:
            \[
            \varphi(m_1) \varphi(m_2) = -\varphi(m_2) \varphi(m_1), \quad \forall m_1, m_2 \in M.
            \]
            This ensures that \( \varphi \) extends uniquely to an \( R \)-algebra homomorphism \( \bar{\varphi} : \Lambda^*(M) \to A \), where the wedge product in \( \Lambda^*(M) \) corresponds to the multiplication in \( A \).

            3. Uniqueness of \( \bar{\varphi} \):  
            Any \( R \)-algebra homomorphism \( \bar{\varphi} \) must satisfy \( \bar{\varphi}(m) = \varphi(m) \) for \( m \in M \) because \( M \) generates \( \Lambda^*(M) \) as an \( R \)-algebra. The super commutativity of \( A \) ensures that \( \bar{\varphi} \) is well-defined and unique.

            4. Commutativity of Diagram:  
            By construction, \( \varphi = \bar{\varphi} \circ i \), as \( i(m) = m \) for \( m \in M \). Hence, the diagram commutes.

            Thus, the exterior algebra satisfies the universal property, with \( \bar{\varphi} \) as the unique \( R \)-algebra map that makes the diagram commute.
        \end{solution}
    \end{subproblem}
\end{problem}

\begin{problem}[Determinants 12]
    Prove that $T(R) \simeq R[x]$. Let $\alpha : R \longrightarrow R$ be the multiplication by 
    $c$, i.e., $\alpha(r) = cr$. What is $\alpha^{\otimes\star}$ as a map from $R[x] \longrightarrow R[x]$

    \begin{solution}
        Tensor Algebra and Polynomial Ring:  
        The tensor algebra \( T(R) \) over a ring \( R \) is given by:
        \[
        T(R) = \bigoplus_{n \geq 0} R^{\otimes n},
        \]
        where \( R^{\otimes 0} = R \) and \( R^{\otimes n} = R \otimes R \otimes \cdots \otimes R \) (with \( n \)-factors). Elements of \( T(R) \) can be written as finite sums of tensors \( r_0 + r_1 \otimes r_2 + \cdots \).

        There is an isomorphism between \( T(R) \) and \( R[x] \), the ring of polynomials in \( x \) with coefficients in \( R \), given by:
        \[
        R^{\otimes n} \ni r_1 \otimes \cdots \otimes r_n \mapsto r_1 x r_2 x \cdots x r_n \in R[x].
        \]
        This identifies \( T(R) \) as the free associative \( R \)-algebra generated by one element \( x \), which corresponds to \( R[x] \).

        Description of \( \alpha \):  
        Let \( \alpha : R \to R \) be defined by \( \alpha(r) = cr \), where \( c \in R \). This is an \( R \)-module homomorphism that scales elements of \( R \) by \( c \).

        Action of \( \alpha^{\otimes \star} \) on \( R[x] \):  
        The map \( \alpha^{\otimes \star} \) extends \( \alpha \) to \( T(R) \) (and thus \( R[x] \)) via its action on the generators:
        \[
        \alpha^{\otimes \star}(r_1 x^n r_2 x^m \cdots r_k x^p) = \alpha(r_1) x^n \alpha(r_2) x^m \cdots \alpha(r_k) x^p.
        \]
        Explicitly, if \( f(x) = \sum_{i=0}^n a_i x^i \in R[x] \), then:
        \[
        \alpha^{\otimes \star}(f(x)) = \sum_{i=0}^n \alpha(a_i) x^i = \sum_{i=0}^n c a_i x^i.
        \]
        Thus, \( \alpha^{\otimes \star} \) acts as multiplication by \( c \) on the coefficients of \( f(x) \).

        We have established that \( T(R) \simeq R[x] \) via the correspondence between tensors and polynomials, and the map \( \alpha^{\otimes \star} \) acts as a scaling map that multiplies each coefficient of a polynomial in \( R[x] \) by \( c \).
    \end{solution}
\end{problem}

\begin{problem}[Determinants 13]
    Prove that if a module $M$ can be generated by $m$ elements then $\wedge^pM = 0$ if $p>m$

    \begin{solution}
        Generating Set of \( M \):\\
        Since \( M \) can be generated by \( m \) elements, there exists a set \( \{x_1, x_2, \ldots, x_m\} \subset M \) such that any element of \( M \) can be expressed as a linear combination of these generators.

        Definition of \( \wedge^p M \):  
        The \( p \)-th exterior power \( \wedge^p M \) is defined as the \( R \)-module generated by wedge products of the form:
        \[
        x_{i_1} \wedge x_{i_2} \wedge \cdots \wedge x_{i_p},
        \]
        where \( x_{i_j} \in M \) and \( i_1 < i_2 < \cdots < i_p \). The alternating property of the wedge product implies that if any two indices \( i_j \) and \( i_k \) are equal, the wedge product is zero.

        Case \( p > m \):\\  
        If \( p > m \), any subset of \( p \) elements chosen from \( \{x_1, x_2, \ldots, x_m\} \) must have at least one repeated index, as there are only \( m \) distinct generators. Due to the alternating property, the wedge product:
        \[
        x_{i_1} \wedge x_{i_2} \wedge \cdots \wedge x_{i_p} = 0
        \]
        for any such choice of \( i_1, i_2, \ldots, i_p \).
        
        Since every wedge product in \( \wedge^p M \) vanishes for \( p > m \), we have:
        \[
        \wedge^p M = 0 \quad \text{for } p > m.
        \]
    \end{solution}
\end{problem}

\begin{problem}[Determinants 15]
    Express the determinants of the following linear transformations in terms of $\det(\alpha)$, 
    $\det(\beta)$, $\dim{V}$, $\dim{W}$ where $\alpha \in \Hom(V, V)$ and $\beta \in \Hom(W, W)$:
    \begin{enumerate}
        \item $\det(\alpha \otimes \beta)$
        \item $\det(\alpha^{\otimes p})$
        \item $\det(\wedge^p\alpha)$
        \item $\det(\wedge^*\alpha)$
    \end{enumerate}

    \begin{solution}
        1. \( \det(\alpha \otimes \beta) \):\\
        The linear transformation \( \alpha \otimes \beta \) acts on \( V \otimes W \), and its determinant is given by:
        \[
        \det(\alpha \otimes \beta) = \det(\alpha)^{\dim W} \cdot \det(\beta)^{\dim V}.
        \]
        This follows from the fact that the action of \( \alpha \otimes \beta \) corresponds to applying \( \alpha \) on \( V \) and \( \beta \) on \( W \), each scaled by the dimension of the other space.

        2. \( \det(\alpha^{\otimes p}) \):\\ 
        The linear transformation \( \alpha^{\otimes p} \) acts on \( V^{\otimes p} \), and its determinant is:
        \[
        \det(\alpha^{\otimes p}) = \det(\alpha)^{\dim V^{\otimes p}} = \det(\alpha)^{(\dim V)^p}.
        \]
        This uses the fact that the determinant of a tensor power scales with the dimension of the tensor space.

        3. \( \det(\wedge^p \alpha) \):\\
        The linear transformation \( \wedge^p \alpha \) acts on \( \wedge^p V \), and its determinant is:
        \[
        \det(\wedge^p \alpha) = \det(\alpha)^{\binom{\dim V}{p}}.
        \]
        Here, \( \binom{\dim V}{p} \) is the dimension of \( \wedge^p V \), as it represents the number of linearly independent wedge products of \( p \) basis vectors.

        4. \( \det(\wedge^* \alpha) \):\\ 
        The transformation \( \wedge^* \alpha \) acts on the full exterior algebra \( \wedge^* V = \bigoplus_{p=0}^{\dim V} \wedge^p V \). The determinant of \( \wedge^* \alpha \) is the product of the determinants over all \( \wedge^p V \):
        \[
        \det(\wedge^* \alpha) = \prod_{p=0}^{\dim V} \det(\wedge^p \alpha) = \prod_{p=0}^{\dim V} \det(\alpha)^{\binom{\dim V}{p}}.
        \]
        Using the binomial theorem, the sum of the binomial coefficients equals \( 2^{\dim V} \), so:
        \[
        \det(\wedge^* \alpha) = \det(\alpha)^{\sum_{p=0}^{\dim V} \binom{\dim V}{p}} = \det(\alpha)^{2^{\dim V}}.
        \]
    \end{solution}
\end{problem}

\begin{problem}[Determinants 19]
    Do this problem according to the outline given. Let $a_1, \ldots, a_n$ be elements of the field $\mathbb{F}$.
    Let $V_n$ denote the matrix whose $i$-th row is $1, \alpha_i, \alpha_i^2, \cdots, a_i^{n-1}$. Prove the formula
    \[  
        \det
        \begin{pmatrix}
            1 & a_1 & a_1^2 & \cdots & a_1^{n-1}\\
            1 & a_2 & a_2^2 & \cdots & a_2^{n-1}\\
            \vdots & \vdots & \vdots & \ddots & \vdots\\
            1 & a_n & a_n^2 & \cdots & a_n^{n-1}\\
        \end{pmatrix}
        = \prod_{i > j}(a_i - a_j)
    \]
    where the product is over all pairs of integers $n \geq i > j \geq 1$. This is to be done as follows:
    Define two polynomials $f(x), g(x) \in \mathbb{F}[x]$ by letting $f(x)$ be the determinant of the matrix
    obtained from $B$ by replacing $a_n$ by $x$, and $g(x)$ the polynomial obtained by replacing $a_n$ by $x$
    in the right-hand side of the above formula. Using properties of determinants and the ring $\mathbb{F}[x]$,
    prove the formula by induction on $n$. Fist conclude that $f(x) = g(x)$ and then evaluate at $a_n$.

    \begin{solution}
        We aim to prove the determinant formula:
        \[
        \det(V_n) = \prod_{i > j}(a_i - a_j),
        \]
        where \( V_n \) is the Vandermonde matrix. This will be done as outlined, using polynomials \( f(x) \) and \( g(x) \) and induction on \( n \).

        Step 1: Define \( f(x) \) and \( g(x) \)\\
        Let \( V_n \) be the Vandermonde matrix:
        \[
        V_n =
        \begin{pmatrix}
        1 & a_1 & a_1^2 & \cdots & a_1^{n-1} \\
        1 & a_2 & a_2^2 & \cdots & a_2^{n-1} \\
        \vdots & \vdots & \vdots & \ddots & \vdots \\
        1 & a_n & a_n^2 & \cdots & a_n^{n-1} \\
        \end{pmatrix}.
        \]

        Define \( f(x) \) as the determinant of the matrix \( V_n \) with \( a_n \) replaced by \( x \):
        \[
        f(x) = 
        \det
        \begin{pmatrix}
        1 & a_1 & a_1^2 & \cdots & a_1^{n-1} \\
        1 & a_2 & a_2^2 & \cdots & a_2^{n-1} \\
        \vdots & \vdots & \vdots & \ddots & \vdots \\
        1 & x & x^2 & \cdots & x^{n-1} \\
        \end{pmatrix}.
        \]

        Define \( g(x) \) as the right-hand side of the formula with \( a_n \) replaced by \( x \):
        \[
        g(x) = \prod_{k=1}^{n-1} (x - a_k) \prod_{i > j} (a_i - a_j).
        \]

        Step 2: Base Case (\( n = 2 \))\\
        For \( n = 2 \), the matrix is:
        \[
        V_2 =
        \begin{pmatrix}
        1 & a_1 \\
        1 & a_2
        \end{pmatrix}.
        \]
        Its determinant is:
        \[
        \det(V_2) = a_2 - a_1.
        \]
        The formula is:
        \[
        \prod_{i > j} (a_i - a_j) = a_2 - a_1.
        \]
        Thus, the formula holds for \( n = 2 \).

        Step 3: Inductive Step\\
        Assume the formula holds for \( n-1 \), i.e., for a Vandermonde matrix \( V_{n-1} \), we have:
        \[
        \det(V_{n-1}) = \prod_{i > j} (a_i - a_j).
        \]

        Now consider \( V_n \). Expanding \( f(x) \) along the last row gives:
        \[
        f(x) = \sum_{k=1}^n (-1)^{n+k} x^{k-1} \det(V_{n-1}'),
        \]
        where \( V_{n-1}' \) is the \((n-1) \times (n-1)\) matrix obtained by removing the \( n \)-th row and \( k \)-th column from \( V_n \). By properties of determinants, this expansion ensures that \( f(x) \) is a polynomial of degree \( n-1 \).

        Similarly, \( g(x) \), as a product of linear terms, is also a polynomial of degree \( n-1 \). Since \( f(x) \) and \( g(x) \) agree at \( n \) distinct points (\( x = a_1, a_2, \ldots, a_{n-1} \)), they must be identical:
        \[
        f(x) = g(x).
        \]

        Step 4: Evaluate at \( x = a_n \)\\
        Substitute \( x = a_n \) into \( f(x) \) and \( g(x) \). For \( f(x) \), this returns \( \det(V_n) \), and for \( g(x) \), this returns:
        \[
        g(a_n) = \prod_{k=1}^{n-1} (a_n - a_k) \prod_{i > j} (a_i - a_j).
        \]
        Thus:
        \[
        \det(V_n) = \prod_{i > j} (a_i - a_j).
        \]

    \end{solution}
\end{problem}

\end{document}


